\newcommand{\dataflow}{data-flow}
\newcommand{\Dataflow}{Data-flow}
\newcommand{\code}[1]{\texttt{\lstinline[basicstyle=\normalsize\ttfamily,identifierstyle={\normalsize},commentstyle={\normalsize\itshape},keywordstyle={\normalsize\bfseries},ndkeywordstyle={\normalsize},stringstyle={\normalsize\ttfamily},numberstyle={\normalsize}]!#1!}}
\newcommand{\CFG}{CFG}
\newcommand{\intraj}{\textsc{Intra}J}
\newcommand{\jastadd}{\textsc{JastAdd}}
\newcommand{\extendj}{\textsc{ExtendJ}}
\newcommand{\jastaddjintraflow}{\textsc{jastaddj-intraflow}}
\newcommand{\jji}{\code{JJI}}
\newcommand{\cG}{\mathcal{G}}
\newcommand{\cV}{\mathcal{V}}
\newcommand{\cE}{\rightarrowtail}
\newcommand{\cP}{\mathcal{P}}
\newcommand{\cM}{\mathcal{M}}

\newcommand{\bnc}{\textsc{BNC}}

\newcommand{\mSyn}{\ensuremath{\uparrow}}
\newcommand{\mInh}{\ensuremath{\downarrow}}
\newcommand{\mHOA}{\ensuremath{\rightarrow}}
\newcommand{\mColl}{\ensuremath{\square}}
\newcommand{\mCirc}{\ensuremath{\circlearrowleft}}

\newcommand{\Abase}[1]{\textcolor{ATGsym}{\mbox{\umlcode{#1}}}}
\newcommand{\Asyn}[2]{\textcolor{ATGsym}{\mbox{\umlcode{\astnode{#1}}.\mSyn{}\umlcode{#2}}}}
\newcommand{\Asynaccess}[2]{\umlcode{get}\textcolor{ATGsym}{\mbox{\umlcode{\astnode{#1}}\textcolor{black}{\umlcode{()}}.\mSyn{}\umlcode{#2}\textcolor{black}{()}}}}
\newcommand{\Ainh}[2]{\textcolor{ATGsym}{\mbox{\umlcode{\astnode{#1}}.\mInh{}\umlcode{#2}}}}
\newcommand{\Ainhdef}[3]{\textcolor{ATGsym}{\mbox{\umlcode{\astnode{#1}}.\umlcode{\astnode{#2}}.\mInh{}\umlcode{#3}}}}
\newcommand{\Ainhindex}[4]{\textcolor{ATGsym}{\mbox{\umlcode{\astnode{#1}}.\umlcode{\astnode{#2}}(#3).\mInh{}\umlcode{#4}}}}
\newcommand{\Ahoa}[2]{\textcolor{ATGsym}{\mbox{\umlcode{\astnode{#1}}.\mHOA{}\umlcode{#2}}}}
\newcommand{\Acoll}[2]{\textcolor{ATGsym}{\mbox{\umlcode{\astnode{#1}}.\mColl{}\umlcode{#2}}}}

\newcommand{\AcircSyn}[2]{\textcolor{ATGsym}{\mbox{\umlcode{\astnode{#1}}.\mSyn{}\umlcode{#2}\mCirc{}}}}



\newcommand{\umlcode}[1]{\textrm{#1}}  % Style of code used in UML fragments
\newcommand{\astnodestyle}{\ttfamily\color{magenta}}
\newcommand{\astnode}[1]{\texttt{\textcolor{magenta}{#1}}}  % Style used for AST node types

\newcommand{\ASTUnrestricted}{AST-unrestricted}
\newcommand{\ParentFirst}{Parent-First}

\newcommand{\project}[1]{\textsc{#1}}
\newcommand{\tool}[1]{\textsc{#1}}

% can't get fbox to work reliably in the UML code, and adjustbox and nested \tikz don't work at all
%\newcommand{\dfapi}{\textsf{\setlength{\fboxsep}{0pt}\fcolorbox{blue}{white}{df-api}}}
\newcommand{\dfapi}{\textbf{\textcolor{black}{[df-api]}}}
\newcommand{\nameapi}{\textbf{\textcolor{black}{[name-api]}}}

\newcommand{\frameworkname}{\textsc{Intra}CFG}
\newcommand{\intracfg}{\textsc{\frameworkname}}

\newcommand{\node}{\mathsf{n}}
\newcommand{\Null}{\mathtt{NULL}}
\newcommand{\Notnull}{\mathtt{NOTNULL}}
\newcommand{\gen}{\mathtt{gen}}
\renewcommand{\kill}{\mathtt{kill}}

\newcommand{\In}{\mathtt{in}}
\newcommand{\Out}{\mathtt{out}}
\newcommand{\Use}{\mathtt{use}}
\newcommand{\Def}{\mathtt{def}}
\newcommand{\tf}{f_t}
\newcommand{\mCi}[1]{ { \textcolor{black!30}{\scriptstyle \pm\text{#1}}}}%Condifdence interval

\newcommand{\CR}[1]{\textbf{[}\textcolor{blue!60!black}{\textbf{CR:} #1}\textbf{]}}
\newcommand{\Ckw}[1]{\texttt{\textbf{#1}}}
\newcommand{\auxlabel}[1]{{\scriptsize{$\textrm{\texttt{#1}}$}}}
\newcommand{\auxlabeli}[2]{{\scriptsize{$\textrm{\texttt{#2}}_{#1}$}}}
\newcommand{\auxlabelbox}[1]{\tikz[baseline=-0.7ex] \node[rectangle, minimum width=0, thin, draw, rounded corners, fill=white, inner sep=2pt, outer sep=0pt] (N) {\auxlabel{#1}};}
\newcommand{\auxlabelboxhoa}[1]{\tikz[baseline=-0.7ex] \node[rectangle, dashed,minimum width=0, thin, draw, rounded corners, fill=white, inner sep=2pt, outer sep=0pt] (N) {\auxlabel{#1}};}

%\newcommand{\auxlabelboxi}[2]{\tikz \node[rectangle, minimum width=0, thin, draw, rounded corners, fill=white] {\auxlabeli{#1}{#2}};}


\newcommand{\Prod}{::=}
\newcommand{\terminal}[1]{\textcolor{green!50!black}{\textit{#1}}}
\newcommand{\vmetavar}[1]{\textcolor{cyan!30!black}{\textsf{\textbf{#1}}}}
\newcommand{\vcode}[1]{\textsf{\textcolor{green!35!black}{{#1}}}}
\newcommand{\vterminal}[1]{\vcode{#1}}
\newcommand{\nta}[1]{\ensuremath{\textit{#1}}}
\newcommand{\tuple}[1]{\ensuremath{\langle #1 \rangle}}
\newcommand{\nt}[1]{\ensuremath{\tuple{\hspace{-0.02cm}\nta{#1}\hspace{0.02cm}}}}
\newcommand{\VB}{\ |\ }
\newcommand{\Gcomment}[1]{\textrm{\textcolor{black!50!white}{({#1})}}}
\newcommand{\sem}[1]{\ensuremath{\llbracket #1 \rrbracket}}
%\newcommand{\semNPA}[1]{\ensuremath{\sem{#1}_{\textit{NPA}}}}
\newcommand{\semNPA}[1]{\ensuremath{\sem{#1}}}
\newcommand{\CFGEdge}[2]{\ensuremath{ #1 \rightarrow #2 }}


\newcommand{\listingsfontsize}{\normalsize}

\newcommand{\NAmark}{\multicolumn{1}{c}{\textcolor{black!40!white}{-}}}
\newcommand{\NAmarkR}{\multicolumn{1}{c|}{\textcolor{black!40!white}{-}}}
\newcommand{\Tcenter}[1]{\multicolumn{1}{c}{#1}}
\newcommand{\TcenterR}[1]{\multicolumn{1}{c|}{#1}}
\newcommand{\succarrow}{\tikz[baseline=-0.7ex] \draw[succarrow, thick, -{Stealth[scale=0.9, inset=0pt, angle'=45]}] (0,0) -- (0.3,0.0);}

\colorlet{hlgreen}{green}
\colorlet{hlorange}{orange}
\colorlet{hlgreenhalf}{green!50!white}
\colorlet{hlorangehalf}{orange!50!white}

\DeclareRobustCommand{\hlgreen}[1]{{\sethlcolor{hlgreenhalf}\hl{#1}}}
\DeclareRobustCommand{\hlorange}[1]{{\sethlcolor{hlorangehalf}\hl{#1}}}

\definecolor{ATGsym}{HTML}{206010}

\definecolor{SQ}{HTML}{0080ff}
\definecolor{JJI}{HTML}{ff0080}
\definecolor{IJnonH}{HTML}{004010}
\definecolor{IJH}{HTML}{00ff20}

\definecolor{succarrow}{HTML}{4e90e2}	% adapted from RunningExample.tex

\definecolor{lightblue}{HTML}{006699}		%#006699
\definecolor{lightgreen}{HTML}{669900}		%#669900
\lstdefinelanguage{JastAdd}{
  %keyword1&2&6
  morekeywords = [1]{aspect, abstract,fun, class, continue, default, enum, extends, false, final, finally, implements, import, instanceof, interface, native, new, null, package, private, protected, public, static, strictfp, throws, transient, true, volatile, length, assert, case, return, super, this, throw, catch, do, if, else, switch, synchronized, while, try, ?},
  %keyword3
  morekeywords = [2]{inh, syn, coll, with, eq, NTA, contributes, to, for, each, circular, with, nta, when, }, %JASTADD keywords
  %keyword4
  morekeywords = [3]{CFGNode, Assignments, Assign,AssignBitwiseExpr, TrueLiteral, Entry,ConstructorDecl,UnceckedException, TryWithResorucesStmt, NTAFinallyBlock,ClassDecl, CloseListNTA, BreakStmt,ThrowStmt,TryStmt,ReturnStmt, ContinueStmt, Exit, Expr, AssignShiftExpr, AssignMultiplicativeExpr, IntegerLiteral, AssignAdditiveExpr, UnaryIncDec,DoStmt,VariableDeclarator, NumericLiteral, ParameterDeclaration, FieldDeclarator, PostfixExpr, PreIncExpr, PreDecExpr, ReachedLVal, Variable,EQOp,CFGSupport, ForStmt, EmptyStmt,FunDecl, MethodDecl, ExprStmt, AssignStmt, IfStmt, AndOp, LessOp, WhileStmt, Block, AssignExpr, PostUnaryInc, PostUnaryDec, Entry, Exit, EQExpr, VarAccess, CFGRoot }, %ASTnode typess
  %keyword5
  morekeywords = [4]{void,Array, ArrayList, Boolean, Byte, Object, BitSet, BufferedReader,Collections, Character, Class, Double, Float, Gamma, AbsDomain, NULL, NOTNULL, TOP,Integer, HashMap, PrintWriter, String, StringBuffer, StringBuilder, Thread, boolean, byte, char, color, double, float, int,Type, long, short, FloatDict, FloatList, IntDict, IntList, JSONArray, JSONObject, PFont, PGraphics, PImage, PShader, PShape, PVector, StringDict, StringList, Table, TableRow, XML, Set, HashSet},
  %keywordstyle = [1]\color{lightblue},
  %keywordstyle = [2]\color{lightgreen},
  keywordstyle = [1]\bfseries,
  keywordstyle = [2]\bfseries,
  keywordstyle = [3]\astnodestyle,
  keywordstyle = [4]\color{lightblue},
  sensitive = true,
  morecomment = [l]{//},
  morecomment = [s]{/*}{*/},
  morecomment = [s]{/**}{*/},
  commentstyle = \color{gray},
  morestring = [b]",
  morestring = [b]',
  stringstyle = \color{purple},
  frame=none
}


\lstdefinelanguage{ASTGrammar}{
  %keyword1&2&6
  morekeywords = [1]{::=,:,<,>, abstract},
  %keyword3
  morekeywords = [2]{IntConstant, Constant, StringConstant,BinExpr, Expr, IdUse, Actual, AddExpr, SubExpr, MulExpr, DivExpr, EQExpr,ModExpr, NEQExpr, LTExpr, GTExpr, LEQExpr, GEQExpr, OrExpr, AndExpr, CallExpr, Null, ArrayLiteralExpr, IndexExpr, NotExpr, TypedExpr, NewExpr, Access, Expr*}, %Expr keywords
  %keyword4
  morekeywords = [3]{Stmt, VarDeclStmt, ExprStmt, AssignStmt, Block, IfStmt, WhileStmt, ReturnStmt, SkipStmt}, %Stmt types
  %keyword5
  morekeywords = [4]{VarDecl, FunDecl, IdDecl, Decl, Program}, %Decls
  morekeywords = [6]{DeclaredType, Initializer, DeclaredReturnType, Formal, Body, Left, Right, Base, Index, DeclType, Value, LValue, RValue, Cond, Then, Else,Identifier},
  morekeywords = [7]{Type, Long, String},
  % keywordstyle = [1]\color{lightblue},
  %keywordstyle = [2]\color{lightgreen},
  keywordstyle = [1]\bfseries\color{lightgreen},
  keywordstyle = [2]\bfseries\color{lightblue},
  keywordstyle = [3]\astnodestyle,
  keywordstyle = [4]\color{orange},
  keywordstyle = [6]\color{CadetBlue},
  keywordstyle = [7]\color{Blue},
  sensitive = true,
  morecomment = [l]{//},
  morecomment = [s]{/*}{*/},
  morecomment = [s]{/**}{*/},
  commentstyle = \color{gray},
  morestring = [b]",
  morestring = [b]',
  stringstyle = \color{purple},
  frame=none,
  literate     =
  {::=}{{{\color{lightblue}::= }}}1
  % {:}{{{\color{lightblue}:}}}2
}

\lstset{
  basicstyle=\listingsfontsize\ttfamily,
  identifierstyle={\listingsfontsize},
  commentstyle={\listingsfontsize\itshape},
  keywordstyle={\listingsfontsize\bfseries},
  ndkeywordstyle={\listingsfontsize},
  stringstyle={\listingsfontsize\ttfamily},
  frame={tb},
  breaklines=true,
  breakatwhitespace=true, %To avoid linebreaks between \code{} and comma.
  columns=[l]{fullflexible},
  numbers=none,
  numberstyle={\listingsfontsize},
  stepnumber=1,
  mathescape
%	escapeinside     = {@}{@}, %General escape does not seem to work in lstinline/GH.
}

\makeatletter
\pgfdeclareshape{topbottombox}{
  \inheritsavedanchors[from=rectangle]
  \inheritanchorborder[from=rectangle]
  \inheritanchor[from=rectangle]{center}
  \inheritanchor[from=rectangle]{base}
  \inheritanchor[from=rectangle]{north}
  \inheritanchor[from=rectangle]{north east}
  \inheritanchor[from=rectangle]{east}
  \inheritanchor[from=rectangle]{south east}
  \inheritanchor[from=rectangle]{south}
  \inheritanchor[from=rectangle]{south west}
  \inheritanchor[from=rectangle]{west}
  \inheritanchor[from=rectangle]{north west}
  \backgroundpath{
    %  store lower right in xa/ya and upper right in xb/yb
    \southwest \pgf@xa=\pgf@x \pgf@ya=\pgf@y
    \northeast \pgf@xb=\pgf@x \pgf@yb=\pgf@y
    \pgfpathmoveto{\pgfpoint{\pgf@xa}{\pgf@ya}}
    \pgfpathlineto{\pgfpoint{\pgf@xb}{\pgf@ya}}
    \pgfpathmoveto{\pgfpoint{\pgf@xa}{\pgf@yb}}
    \pgfpathlineto{\pgfpoint{\pgf@xb}{\pgf@yb}}
 }
}
\makeatother


% Patch UML package pgf-umlcd to be able to write abstract grammar as class name.

\newbool{ANON}
%\booltrue{ANON}
\boolfalse{ANON}
\newcommand{\anon}[2]
{\ifbool{ANON}{
   #1
}{
   #2
}}


\newcommand{\paperIref}{%
Idriss Riouak, Christoph Reichenbach, G\"{o}rel Hedin and Niklas Fors
``A Precise Framework for Source-Level Control-Flow Analysis'.
In \emph{21st International Working Conference on Source Code Analysis and Manipulation (SCAM),
pp. 1--11}. Virtual, 2021.
}
\newcommand{\paperIIref}{%
Idriss Riouak, G\"{o}rel Hedin, Christoph Reichenbach and Niklas Fors
``JFeature: Know Your Corpus!''.
In \emph{22st International Working Conference on Source Code Analysis and Manipulation (SCAM),
pp. 236--241s}. Limassol, Cyprus, 2022.
}




\lstdefinelanguage{bytecode}{
  basicstyle=\ttfamily,
  commentstyle=\color{green},
  numbers=none,
  numberstyle=\color{blue},
  keywordstyle=\bfseries
   morekeywords = [1]{return,iconst\_m1,goto,istore\_1,iconst\_1,ifle,iload\_1,iconst\_0},
   keywordstyle = [1]\bfseries
}



% From here on, the macros.tex file contains settings about the layout of the
% Thesis itself. Do not change anything below this line unless you know what
% you are doing.


% Defines the layout settings for the thesis
\newcommand{\pageSettings}{
 \setlength{\voffset}{3pt}
 \setlength{\hoffset}{-1in}
 \setlength{\textheight}{520pt}
 \setlength{\textwidth}{327pt}
 \setlength{\oddsidemargin}{93pt}
 \setlength{\evensidemargin}{195pt}
\setlength{\marginparsep}{0pt}
}


% ------------------ Change Chapter title row layout
\makeatletter
\def\thickhrulefill{\leavevmode \leaders \hrule height 1ex \hfill \kern \z@}
\def\@makechapterhead#1{%
  \markmargin%
  \addtolength{\spiff}{3cm}%
  \vspace*{10\p@}%
  {\parindent \z@ \raggedleft \reset@font
            \sffamily \bfseries \scshape %\@chapapp{} \thechapter%
        \par\nobreak%
        \interlinepenalty\@M%
    \Huge #1\par\nobreak
    \hrulefill
    \par\nobreak
    \vskip 80\p@
  }}
\def\@makeschapterhead#1{%
  \vspace*{10\p@}%
  {\parindent \z@ \raggedleft \reset@font
            \sffamily \bfseries \scshape \vphantom{\@chapapp{} \thechapter}
        \par\nobreak
        \interlinepenalty\@M
    \Huge  #1\par\nobreak
    \hrulefill
    \par\nobreak
    \vskip 80\p@
  }}

%------------------------ part page layout
%% from book.cls:
\renewcommand\part{%
   \if@openright
      \cleardoublepage
   \else
      \clearpage
   \fi
   \thispagestyle{plain}%
  \if@twocolumn
    \onecolumn
    \@tempswatrue
  \else
    \@tempswafalse
  \fi
  \null\vfil
  \secdef\@part\@spart}

\def\@part[#1]#2{%
    \ifnum \c@secnumdepth >-2\relax
      \refstepcounter{part}%
      \addcontentsline{toc}{part}{#1}%
    \else
      \addcontentsline{toc}{part}{#1}%
    \fi
    \markboth{}{}%
    {\centering
     \interlinepenalty \@M
     \normalfont
     \Huge \reset@font \hrulefill \\ \raggedleft \vspace{5mm} \sffamily%
            \bfseries \scshape {#2} \hspace{15mm}\ \\ \hrulefill \par}%
    \@endpart}
\def\@spart#1{%
    {\centering
     \interlinepenalty \@M
     \normalfont
     \Huge \sffamily \bfseries scshape #1\par}%
    \@endpart}
\def\@endpart{\vfil    %% \newpage -- HWL: no pagebreak with part env
              \if@twoside
                \null
                \thispagestyle{empty}%
                %%\newpage
              \fi
              \if@tempswa
                \twocolumn
              \fi}
\makeatother
%-----------------------------------------------



%Make fancy header and no footer
\pagestyle{fancy}
\renewcommand{\chaptermark}[1]{\markboth{#1}{}}
\renewcommand{\sectionmark}[1]{\markright{\thesection\ #1}}
\fancyhf{}
\fancyhead[LE, RO]{\bfseries\thepage}
\fancyhead[LO]{\itshape \rightmark}
\fancyhead[RE]{\truncate{.95\headwidth}{\itshape \leftmark}}
\renewcommand{\headrulewidth}{0.5pt}
\renewcommand{\footrulewidth}{0pt}%
\setlength{\headheight}{75pt}
\fancypagestyle{plain}{
  \fancyfoot[LE, RO]{\chapterFoot}
  \fancyhead{}
  \renewcommand{\headrulewidth}{0pt}
}

\newcommand{\chapterFoot}{}
\newcommand{\markmargin}{}
\newlength{\spiff}

\pageSettings % -------------------------> Applying the style to the thesis. See macros.tex for more details
\allsectionsfont{\sffamily} % -----------> All sections in sans serif


\newtheorem{definition}{Definition}
\newcommand{\Chapter}[1]{Chapter~\ref{#1}}


\newfloat{paperfoot}{b}{paper}
\newcommand{\paperRemark}[1]{
  \begin{paperfoot}%
  \hrulefill \flushleft \footnotesize #1 \end{paperfoot}
}


% patch for uml
% Patch UML package pgf-umlcd to be able to write abstract grammar as class name.
\ExplSyntaxOn
\NewDocumentCommand{\defineclassname}{m}
 {
  \tl_set:Nn \umlcdClassName { #1 }
  \tl_set_eq:NN \umlcdClassNameString \umlcdClassName
  \tl_replace_all:Nfn \umlcdClassName { \char_generate:nn { `_ } { 8 } } { \_\kern1pt }
 }
\cs_generate_variant:Nn \tl_replace_all:Nnn { Nf }
\ExplSyntaxOff
\xpatchcmd{\classAndInterfaceCommon}
 {\def\umlcdClassName}
 {\defineclassname}
 {}{}
\xpatchcmd{\endclass}
 {(\umlcdClassName)}
 {(\umlcdClassNameString)}
 {}{\ddt}
\xpatchcmd{\endinterface}
 {(\umlcdClassName)}
 {(\umlcdClassNameString)}
 {}{\ddt}
\xpatchcmd{\endabstractclass}
 {(\umlcdClassName)}
 {(\umlcdClassNameString)}
 {}{\ddt}
\xpatchcmd{\endobject}
 {(\umlcdClassName)}
 {(\umlcdClassNameString)}
 {}{\ddt}
\xpatchcmd{\endclassAndInterfaceCommon}
 {(\umlcdClassName)}
 {(\umlcdClassNameString)}
 {}{\ddt}
\xpatchcmd{\endclassAndInterfaceCommon}
 {(\umlcdClassName)}
 {(\umlcdClassNameString)}
 {}{\ddt}
\xpatchcmd{\endclassAndInterfaceCommon}
 {(\umlcdClassName)}
 {(\umlcdClassNameString)}
 {}{\ddt}




\newcommand{\ntaX}[1]{\ensuremath{\textcolor{cyan!20!black}{\textit{#1}}}}
\newcommand{\ntam}[1]{\ensuremath{{#1}}}
\newcommand{\ntstar}[1]{\ensuremath{\tuple{\hspace{-0.02cm}\ntaX{#1}\hspace{0.02cm}}}\star}
\newcommand{\ntq}[1]{\ensuremath{\tuple{\hspace{-0.02cm}\ntaX{#1}\hspace{0.02cm}}}?}
\newcommand{\ntm}[1]{\ensuremath{\tuple{\hspace{-0.02cm}{#1}\hspace{0.02cm}}}}
\newcommand{\astmap}[1]{\textrm{\textcolor{green!20!grey}{#1}}}
\newcommand{\Cty}[1]{\textcolor{dblue}{\textsf{#1}}}
\chapter{Control-Flow Analysis}
\section{Control-Flow Analysis}
\label{sec:controlflow}
Control-Flow Analysis is a static analysis technique used to determin the control-flow graph (CFG) of a program.
The CFG is a directed graph representing all the possible path that a program can
take during its execution. The CFG of a program is used to perform many static analyses, such as
data-flow analysis, and to perform program transformations, e.g., loop unrolling.


Let us consider the graph $\mathcal{G}=(B,F)$ where $B$ is the set of nodes 
of the graph and $F$ is the set of edges. Each node $b$ of $B$ represents a
basic block of the program. A basic block is a sequence of instructions with one and only one
entry point and exit point. Meaning that there is no jump instruction in the middle of a basic block.
The edges of $F$ represent the possible transitions between basic blocks. Given two
basic blocks $b_1$ and $b_2$, there is an edge \CFGEdge{b_1}{b_2} in $F$ if $b_2$ is an immediate
successor of $b_1$.
\begin{definition}
    Given a basic block $b$, we define the set of successor of $b$ as 
    $$ \text{succ(b)} = \{b_2 \mid \CFGEdge{b}{b_2} \in F\}.$$
\end{definition}
The predecessor relation is defined in a similar way:
\begin{definition}
    Given a basic block $b$, we define the set of predecessor of $b$ as the 
    inverse of the successor relation:
    $$ \text{pred(b)} = \{b_1 \mid \CFGEdge{b_1}{b} \in F\} = \text{succ}^{-1}\text{(b)}.$$
\end{definition}
Among $B$, we can distinguish two special nodes: the entry node $b_{\text{entry}}$ and 
the exit node $b_{\text{exit}}$. The entry node is the node from which the program starts its execution.
The exit node is the node to which the program returns after its execution. The entry node is the only node
with no predecessor, and the exit node is the only node with no successor. More 
formally, we can define the entry node as the unique node $b_{\text{entry}}$ such that
$$ \forall b \in B, \text{pred}(b) \neq \emptyset \Rightarrow b = b_{\text{entry}}.$$
Similarly, we can define the exit node as the unique node $b_{\text{exit}}$ such that
$$ \forall b \in B, \text{succ}(b) \neq \emptyset \Rightarrow b = b_{\text{exit}}.$$
\begin{definition}
    A path in $\mathcal{G}$ is a sequence of basic blocks $b_1, \ldots, b_n$ such that
    $$ \forall i \in \{1, \ldots, n-1\},  b_{i+1} \in \text{succ}(b_i).$$
\end{definition}


Most of the time, the CFG is constructed on top of an intermediate representation
(IR) of the program. The IR is a language that is easier to analyze than the original
source language. The IR is usually a low-level language, such as LLVM IR or Java Bytecode, that are
closer to the machine code than the source language. 
Moreover, since the IR is usually targeted by serveral different langauges, 
it requires less engineering effort to implement a control-flow analysis for the IR than for each
source language.
A less common approach is to construct the CFG on top of the source language.
This method fits well for imperative languages, where the control-flow is 
explicit in the source code. For functional languages, the
control-flow is implicit in the source code, and the CFG is usually constructed
on top of the IR.




\section{Control-Flow Analysis Using RAGs}
\label{sec:cfarag}
In this section, we discuss IntraCFG, a declarative and language independent framework for 
constructing CFGs using RAGs. IntraCFG is the second attempt to implement a control-flow analysis
on top of the AST using RAGs. The first attempt was the \texttt{JastAddJ-intraflow}, which was
implemented on the \texttt{JastAddJ} compiler. IntraCFG tries to overcome the limitations of
\texttt{JastAddJ-intraflow} by providing a more general and flexible framework.
Before discussing IntraCFG and JastAddJ-Intraflow, we first introduce the 
SimpliC language, a simple imperative language, that we use to illustrate the use of IntraCFG.

\subsection{SimpliC}
SimpliC is a simple imperative language subset of the C language.
The syntax of SimpliC is defined by the following abstract syntax:
\begin{lstlisting}[caption={Syntax of the SimpliC language}]
% Program ::= Function*
% Function ::= Type id Param* Block
% Param ::= Type id
% Block ::= Stmt*
Stmt ::= Block | IfStmt | WhileStmt | ReturnStmt | ExprStmt | DeclStmt | AssignStmt
IfStmt ::= Cond:Expr Then:Expr Else:Expr
WhileStmt ::= Cond:Expr Body:Expr
ReturnStmt ::= Expr
ExprStmt ::= Expr
DeclStmt ::= Type id Init:Expr
AssignStmt ::= Dest:Expr Source:Expr
Expr ::= id | Numeral | true | false | Add | Less | Not | Call
Add ::= Left:Expr Right:Expr
Less ::= Left:Expr Right:Expr
Not ::= Expr
Call ::= id Arg:Expr*
\end{lstlisting}
The starting symbol of the SimpliC language is the \texttt{Program} non-terminal.
A \texttt{Program} is a list of \texttt{Function}s. A \texttt{Function} is a function
declaration, which is composed of a return type, a name, a list of parameters, and a body.
The body of a function is a \texttt{Block}, which is a list of statements. A \texttt{Block}
is a sequence of statements, which can be nested. The statements of SimpliC are
\texttt{IfStmt}, \texttt{WhileStmt}, \texttt{ReturnStmt}, \texttt{ExprStmt}, \texttt{DeclStmt},
and \texttt{AssignStmt}. Reguarding expressions, we decided to include only the most basic
expressions, such as variable reference (\texttt{Id}),  \texttt{Numeral}, \texttt{Bool}
and some binary operators (\texttt{Add}, \texttt{Less} and \texttt{Not}).
For the sake of simplicity, we did not include \texttt{BreakStmt} as the control-flow Analysis
solution is similar to the one of \texttt{ReturnStmt}. 

\subsection{JastAddJ-Intraflow}
\label{sec:jastaddj-intraflow}
The \texttt{JastAddJ-intraflow} is the first framework designed for the construction of CFGs
using RAGs. It is not langauge agnostic as it targets the Java language. 



\chapter{Abstract}


Static program analysis plays a crucial role in ensuring the quality and security of
software applications by detecting and fixing bugs,
and potential security vulnerabilities in the code. The utilisation of declarative
paradigms in dataflow analysis as part of static program analysis has become 
increasingly popular in recent years due to its enhance expressivity 
and modularity, allowing for a higher-level programming approach, resulting in
easy and efficient development.\\[5pt]

The aim of this thesis is to explore the design and implementation of \emph{control-flow} and 
\emph{dataflow} analyses using the declarative \emph{Reference Attribute Grammars} formalism.
Specifically, we focus on the construction of analyses directly on the source code
rather than on an intermediate representation.\\[5pt]

The main result, summarised in this thesis, is our language-agnostic framework, called \textsc{IntraCFG}.
\textsc{IntraCFG} enables efficient and effective dataflow analysis by allowing the construction of precise and
lightweight source-level control-flow graphs. The framework superimposes control-flow
graphs on top of the abstract syntax tree of the program. 
The effectiveness of \textsc{IntraCFG} is demonstrated through two case studies, 
\textsc{IntraJ} and \textsc{IntraTeal}. These case studies showcase the potential and 
flexibility of \textsc{IntraCFG} in diverse contexts, such as bug detection and education. 
\textsc{IntraJ} specifically supports the Java programming language, while 
\textsc{IntraTeal} is an educational language designed for teaching program analysis.\\[5pt]

\textsc{IntraJ}, has proven to be faster than and precise as well-known
industrial tools. 
The combination of precision, performance, and on-demand evaluation in \textsc{IntraJ} 
leads to a low latency in querying the analysis results. This makes \textsc{IntraJ} a
suitable tool for use in interactive tools. Preliminary experiments have also 
been conducted to demonstrate how \textsc{IntraJ} can be used to support interactive
bug detection and fixing.\\[5pt]

Additionally, this thesis presents \textsc{JFeature}, a tool for automatically extracting
and summarising the key features of a Java corpus. \textsc{JFeature} provides 
researchers and developers a deeper understanding of the characteristics of 
a codebase, enabling them to identify suitable corpus for the evaluation of their 
tools and methodologies.

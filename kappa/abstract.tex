\chapter{Abstract}


Static program analysis plays a crucial role in ensuring the quality and security of 
software applications by detecting and fixing bugs, vulnerabilities, 
and potential security risks in the code. The utilization of declarative paradigms
in dataflow analysis as part of this process has become increasingly popular in recent years.

The aim of this thesis is to examine the benefits and limitations of implementing 
declarative paradigms in static program analysis through dataflow analysis, specifically 
by computing the analysis directly on the source code rather than on an intermediate representation using
Reference Attibute Grammars.

The focus of this work is on our language-agnostic framework, called \textsc{IntraCFG}, 
that enables more efficient and effective dataflow analysis by constructing precise,
lightweight source-level control-flow graphs. This framework superimposes control-flow 
graphs on the abstract syntax tree of the program, allowing for more accurate analysis 
on the source code level. The effectiveness of the \textsc{IntraCFG} framework is 
demonstrated through two case studies, \textsc{IntraJ} and \textsc{IntraTeal}, that 
show the potential and the flexibility of \textsc{IntraCFG} in different contexts,
such as, bug detection and teaching. 
\textsc{IntraJ}, has proven to be both faster and precise as well-known 
industrial tools. Its precision and performance, combined with 
on-demand evaluation, make \textsc{IntraJ} an ideal tool for use in 
Integrated Development Environments. 
Additionally, the thesis presents \textsc{JFeature}, a tool for automatically extracting 
and summarising the key features of a corpus of Java programs. \textsc{JFeature} allows 
researcher and developers to gain a deeper understanding of the characteristics of a codebase and 
can be used to evaluate the effectiveness of their tools and techniques.







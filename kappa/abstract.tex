\chapter{Abstract}


Static program analysis plays a crucial role in ensuring the quality and security of
software applications by detecting and fixing bugs,
and potential security vulnerabilities in the code. The utilization of declarative paradigms
in dataflow analysis as part of this process has become increasingly popular in recent years.\\[5pt]

The aim of this thesis is to explore the design and implementation of \emph{control-flow} and 
\emph{dataflow} analyses using the \emph{Reference Attribute Grammars} formalism.
Specifically, we focus on the construction of analyses directly on the source code
rather than on an intermediate representation.\\[5pt]

The main result, summarised in this thesis, is our language-agnostic framework, called \textsc{IntraCFG}.
\textsc{IntraCFG} enables efficient and effective dataflow analysis by allowing the construction of precise and
lightweight source-level control-flow graphs. The framework superimposes control-flow
graphs on top of the abstract syntax tree of the program. The effectiveness of the \textsc{IntraCFG} framework is
demonstrated through two case studies, \textsc{IntraJ} and \textsc{IntraTeal}, that
show the potential and the flexibility of \textsc{IntraCFG} in different contexts,
such as, bug detection and teaching.
\textsc{IntraJ}, has proven to be faster and precise as well-known
industrial tools. Its precision and performance, combined with
on-demand evaluation, make \textsc{IntraJ} an ideal tool for use in
Integrated Development Environments.
Additionally, this thesis presents \textsc{JFeature}, a tool for automatically extracting
and summarising the key features of a corpus of Java programs. \textsc{JFeature} allows
researcher and developers to gain a deeper understanding of the characteristics of a codebase and
can be used to evaluate the effectiveness of their tools and techniques.

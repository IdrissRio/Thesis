\chapter{Abstract}

\vspace{-0.5cm}
Static program analysis plays a crucial role in ensuring the quality and security of
software applications by detecting and fixing bugs,
and potential security vulnerabilities in the code. The use of declarative
paradigms in dataflow analysis as part of static program analysis has become
increasingly popular in recent years. This is due to its enhanced expressivity
and modularity, allowing for a higher-level programming approach, resulting in
easy and efficient development.

The aim of this thesis is to explore the design and implementation of \emph{control-flow} and
\emph{dataflow} analyses using the declarative \emph{Reference Attribute Grammars} formalism.
Specifically, we focus on the construction of analyses directly on the source code
rather than on an intermediate representation.

The main result of this thesis is our language-agnostic framework, called \textsc{IntraCFG}.
\textsc{IntraCFG} enables efficient and effective dataflow analysis by allowing the construction of precise and
source-level control-flow graphs. The framework superimposes control-flow
graphs on top of the abstract syntax tree of the program.
The effectiveness of \textsc{IntraCFG} is demonstrated through two case studies,
\textsc{IntraJ} and \textsc{IntraTeal}. These case studies showcase the potential and
flexibility of \textsc{IntraCFG} in diverse contexts, such as bug detection and education.
\textsc{IntraJ} supports the Java programming language, while
\textsc{IntraTeal} is a tool designed for teaching program analysis for an educational language, TEAL.

\textsc{IntraJ} has proven to be faster than and as precise as well-known
industrial tools.
The combination of precision, performance, and on-demand evaluation in \textsc{IntraJ}
leads to low latency in querying the analysis results. This makes \textsc{IntraJ} a
suitable tool for use in interactive tools. Preliminary experiments have also
been conducted to demonstrate how \textsc{IntraJ} can be used to support interactive
bug detection and fixing.

Additionally, this thesis presents \textsc{JFeature}, a tool for automatically extracting
and summarising the features of a Java corpus, including the use of different Java features (e.g., use of \emph{Lambda Expressions}) across different
Java versions. \textsc{JFeature} provides
researchers and developers with a deeper understanding of the characteristics of
corpora, enabling them to identify suitable benchmarks for the evaluation of their
tools and methodologies.

\section{Conclusions and Future Work}%
\label{sec:kappa:conclusions}%
In this thesis, we have explored the use of declarative paradigms for
dataflow analysis in static program analysis. Our main contributions include the
development of a new framework for precise and light-weight construction of source-level
control-flow graphs, called \textsc{IntraCFG}, and two case studies that demonstrate
its effectiveness, namely \textsc{IntraJ} and \textsc{IntraTeal}. 
We have demonstrated the potential of IntraJ as a precise and light-weight tool
for detecting complex dataflow analysis, such as \textsc{NullPointerExceptions} and \textsc{DeadAssignements},
through its integration in IDEs. Additionally, we have shown the use of \textsc{IntraTeal}
for educational purposes, in combination with the visualisation tool \textsc{CodeProber},
resulting in an effective tool for students to learn about dataflow analysis. 
Furthermore, we presented  \textsc{JFeature}, an extensible tool for automatically 
extracting and summarising the key features of a corpus of Java programs. 
\textsc{JFeature} allows developers to gain a deeper understanding of the composition and 
suitability of software corpora for their particular research or development needs. 
By applying \textsc{JFeature} to four widely-used corpora in the program analysis area, 
we demonstrated its potential for use in corpus evaluation, the creation of new 
corpora, and longitudinal studies of individual Java projects. 
Together, these contributions provide frameworks and practiva tools practical tools 
for improving the development and maintenance of software systems.


In future, we plan to investigate the application of interprocedural analysis
to \textsc{IntraCFG}. Interprocedural analysis refers to the process of analyzing 
the interactions and dependencies between different procedures or functions within a
program. One of the main challenges in this area is the computation of call graphs,
which are diagrams that depict the relationships between the various functions and 
procedures in a program.
To achieve this, we plan to investigate the use of point-to analysis, which is a 
technique used to determine the memory locations that a given variable may point to. 
This is a crucial step in interprocedural analysis, as it allows us to determine 
the flow of data between different procedures and functions. However, since we aim 
to run these analyses within IDEs, we want to investigate how to compute and refine 
the results of point-to analysis incrementally, rather than over the entire codebase.

Additionally, we plan to explore the use of on-demand evaluation, which is becoming a popular 
technique in compilers, e.g., \texttt{rustc}~\cite{Rust_Query_Guide}. This approach allows for the 
evaluation of code only when it is needed, rather than statically analyzing the 
entire program. Furthermore, we plan to add native support for widening and narrowing 
operators in Jastadd's circular attributes, which would enable an easier implementation
of analysis over infinite lattices. Finally, we aim to explore the application of \textsc{IntraCFG} to 
dynamically typed languages such as JavaScript and Python.
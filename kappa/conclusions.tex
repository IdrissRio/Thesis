\section{Conclusions and Future Work}%
\label{sec:kappa:conclusions}%
In this thesis, we have explored the use of RAGs declarative paradigms for
intraprocedural control-flow and dataflow analysis in static program analysis. Our main contributions is the
development of a new framework for precise construction of source-level
control-flow graphs, called \textsc{IntraCFG}.
\textsc{IntraCFG} is language-agnostic and is a flexible framework that can be used to
construct precise control-flow graphs for a wide range of programming languages.
We have demonstrated the effectiveness of \textsc{IntraCFG} through two case studies,
namely \textsc{IntraJ} and \textsc{IntraTeal}.
\textsc{IntraJ} is an instance of \textsc{IntraCFG} for Java, 
built upon the \textsc{ExtendJ} compiler.
We have demonstrated the potential of \textsc{IntraJ} as a precise tool
for detecting complex dataflow analysis, such as \textsc{NullPointerExceptions} and \textsc{DeadAssignements}.
We compared its performance and precision with the industrial tool \textsc{SonarQube}. 
The results showed that \textsc{IntraJ} is more efficient and as precise as \textsc{SonarQube}.
Additionally, we have shown the use of \textsc{IntraTeal}
for educational purposes, in combination with the visualisation tool \textsc{CodeProber},
resulting in an effective tool for students to learn about dataflow analysis.
We also presented  \textsc{JFeature}, an extensible tool for automatically
extracting and summarising the key features of a corpus of Java programs.
\textsc{JFeature} allows researchers and developers to gain a deeper understanding of the composition and
suitability of software corpora for their particular research or development needs.
By applying \textsc{JFeature} to four widely-used corpora in the program analysis area,
we demonstrated its potential for use in corpus evaluation, the creation of new
corpora, and longitudinal studies of individual Java projects.
Together, these contributions provide frameworks and practical tools
for improving the development and maintenance of software systems.


In the future, we plan to investigate using RAGs to support interprocedural analysis in \textsc{IntraCFG}.
Interprocedural analysis refers to the process of analysing
the interactions and dependencies between different procedures or functions within a
program. One of the main challenges in this area is the computation of precise \emph{call graphs},
which are graphs representing the dependencies between procedures and functions.
Call graphs are crucial for interprocedural dataflow analysis, as they allow to determine
the flow of data between different procedures and functions.
Static analysis techniques can be used to compute call graphs, including \emph{Rapid
Type Analysis}~\cite{bacon1996fast} which is fast but imprecise, or \emph{call graph} analysis~\cite{SootJavaBytecode2010} in combination with \emph{points-to}~\cite{steensgaard1996points} analysis 
which is computationally expensive but more precise. Points-to analysis involves determining 
the memory locations that a variable may point to. 
In order to achieve accurate and efficient program analysis within IDEs,
we aim to explore a combination of static analysis techniques, including
Rapid Type Analysis and Points-to Analysis. 
Performing a precise analysis typically requires starting from the main entry 
point of the program. However, in order to run the analysis within IDEs, we plan 
to start with the methods that are currently being edited. This presents a significant 
challenge for the analysis, as it must be able to accurately track data flow from 
these partial starting points.

Furthermore, we plan to push even further the use of the on-demand evaluation
feature in \textsc{IntraJ}. Currently, quick-fixes are computed when the analyses are triggered.
We plan to investigate the feasibility of computing quick-fixes on-demand,
i.e., when they are requested by the user.
The proposed change could help optimize the use of \textsc{IntraJ} in IDEs.
Finally, we plan to add native support for widening and narrowing
operators in \textsc{JastAdd}'s circular attributes, which would enable an easier implementation
of analysis over infinite lattices.
% Finally, we aim to explore the application of \textsc{IntraCFG} to
% dynamically typed languages such as JavaScript and Python.
\section{Contribution}
\label{sec:contribution}
In this Thesis, we present two major contributions to the field of programming 
language analysis and software engineering. 
The first significant contribution of this thesis is the formalisation of \textsc{IntraCFG},
a language-agnostic RAG-based framework for constructing control-flow graphs (CFGs) 
on top of the abstract syntax tree (AST). \textsc{IntraCFG}, is a highly flexible
framework that enables the construction of light-weight and precise the control-flow graphs
of a program. To exemplify the applicability of \textsc{IntraCFG}, we instantiate it for the TEAL language.

The second contribution is a Java feature extractor called \textsc{JFeature}.
This tool is designed to assist researchers in selecting the appropriate corpora
for their experiments by extracting relevant features from Java programs. 
\textsc{JFeature} is able to extract a wide range of features, 
including syntactic, e.g., use of keywords, and semantic features, e.g.,
the number of method calls.


\subsection{\textsc{IntraCFG}: A Precise Framework for Source-level Control-flow Analysis}
The field of control-flow graph (CFG) construction has seen significant advancements
in recent years, with various frameworks being proposed to aid in the construction 
of precise intraprocedural CFGs~\cite{smits2020flowspec,10.1016/j.scico.2012.02.002}.
We contributed to the state-of-the-art introducing \textsc{IntraCFG}, a declarative, RAG-based,
and language-independent framework for constructing precise intraprocedural CFGs.

Unlike most other frameworks, which build CFGs on an Intermediate Representation (IR) level,
such as bytecode, \textsc{IntraCFG}'s approach is unique in that it superimposes the CFGs 
on the Abstract Syntax Tree (AST). This allows for a more accurate client analysis,
as the CFGs are constructed directly on the source code level, rather than an
intermediate representation. Additionally, this approach also enables the construction 
of \textsc{AST-Unrestricted} CFGs, which are CFGs whose shape is not restricted to the AST structure.

\subsubsection*{Design and Implementation}
\textsc{IntraCFG} is written within the \textsc{JastAdd} ecosystem.

The framework provides three interfaces, namely \code{CFGRoot}, \code{CFGNode}, and \code{CFGSupport},
to be implemented by AST nodes. Each interface provides a set of attribute that are used to construct
the CFGs.

The \code{CFGRoot} interface is the starting point of the CFG construction process.
It is implemented by AST nodes that identify the subtree containing the entire method 
or function. This interface is responsible for defining the entry and exit nodes of the CFG and 
forwards a reference of the entry node to every \code{CFGNode}.


The \code{CFGSupport} interface is implemented by nodes that are not part of the CFG,
but define its shape. These nodes are used to provide additional information 
about the structure of the CFG, such as the presence of loops or branches. 
They are designed to be used as auxiliary nodes to support the CFG construction 
process, and they do not participate directly in the control-flow.
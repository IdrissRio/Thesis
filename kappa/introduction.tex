\chapter{Introduction}
\section{Introduction}

Static program analysis (\emph{static analysis} for short) is
\emph{``the art of reasoning about a program's behaviour without executing it''}~\cite{spa}.
It is an essential technique for improving the quality and reliability of software
systems and has been widely used in various applications such
as safety~\cite{cousot2005astree,Blanchet2002} and security~\cite{piskachev2021secucheck,flowDroid,ayewah2008using,dura2021javadl,fink2012wala},
performance optimization~\cite{aho2007compilers,appel2004modern}, and software maintenance.
Static analysis aims to identify potential errors, bugs, or vulnerabilities
in a program before it is executed.
By examining the source code of a program, static
analysis can provide a detailed and precise understanding of its behaviour, including
its control flow~\cite{allen1970control}, dataflow~\cite{kam1977monotone},
and potential interactions with other system components.



One of the key techniques used in static analysis is \emph{dataflow analysis},
which focuses on the flow of data through a program. Dataflow analysis applications are used to identify
potential sources of errors, such as uninitialised variables or null dereference~\cite{riouak2021precise,10.1016/j.scico.2012.02.002},
and to optimise the program's performance by identifying opportunities for
parallelisation or other forms of optimisation~\cite{aho2007compilers}.
Traditionally, dataflow analysis has been performed using imperative paradigms,
which are based on the idea of explicitly specifying \emph{how} the analysis should be
performed.
However, more recently, there has been a growing interest in using
declarative paradigms for dataflow analysis, which are based on specifying \emph{what}
the analysis should compute rather than \emph{how} it should be performed.
In this Thesis, we will investigate the utilization of declarative paradigms in
dataflow analysis~\cite{smits2020flowspec,madsen2016programming}. Our focus is to
create a novel framework that constructs control-flow graphs (CFGs), which represent the
sequence of executed instructions in a program, to perform dataflow analyses in a
more effective and efficient manner. This approach, combined with on-demand evaluation,
enables execution of complex dataflow analysis in Integrated Development Environments,
improving the efficiency of the development process.

This Thesis presents two contributions.
In Paper~\ref{paper:intraj},  we introduce \textsc{IntraCFG}, a language-agnostic framework for building
precise, lightweight source-level CFGs. Unlike other frameworks that
build CFGs on the intermediate representation level, \textsc{IntraCFG} superimposes
CFGs on the abstract syntax tree (AST) of the program, allowing for more accurate
analysis on the source code level and the construction of CFGs whose shape is not
restricted to the AST structure. We demonstrate the effectiveness
of \textsc{IntraCFG} through the implementation of \textsc{IntraJ}, a Java language
instance. We show that \textsc{IntraJ} is as precise and efficient as existing
static analysis tools. We also show the teaching potential of \textsc{IntraCFG}
through the implementation of \textsc{IntraTeal}, a Teal instance, which
is a language designed for teaching \emph{Program Analysis} concepts.
Secondly, we introduce \textsc{JFeature}, a tool for automatically extracting and summarising
the key features of a corpus (i.e., collection) of Java programs. \textsc{JFeature}
allows researchers to gain an understanding of the characteristics, for example,
use of different Java feature for different Java versions, of a codebase,
which can be used to evaluate the effectiveness of their tools and techniques.
In addition, in Paper~\ref{paper:jfeature}, we present a case study of using \textsc{JFeature} to
identify the key features of four popular open-source corpora, providing a
baseline for future research.

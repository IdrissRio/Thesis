\chapter{Introduction}
\section{Introduction}

Static program analysis (static analysis for short) is 
\emph{the art of reasoning about a program's behaviour without executing it}~\cite{spa}.
It is an essential tool for improving the quality and reliability of software
systems and has been widely used in various applications such
as safety~\cite{cousot2005astree,Blanchet2002} and security~\cite{piskachev2021secucheck,flowDroid,ayewah2008using,dura2021javadl,fink2012wala},
performance optimization~\cite{aho2007compilers,appel2004modern}, and software maintenance.
Static program analysis aims to identify potential errors, bugs, or vulnerabilities
in a program before it is executed, thus avoiding costly and time-consuming 
debugging processes. By examining the source code of a program, static program 
analysis can provide a detailed and precise understanding of its behaviour, including
its control flow~\cite{allen1970control}, dataflow~\cite{kam1977monotone}, and potential interactions with other system components.



One of the key techniques used in static program analysis is dataflow analysis,
which focuses on the flow of data through a program. Dataflow analysis applications are used to identify
potential sources of errors, such as uninitialised variables or null pointers dereference, 
and to optimise the program's performance by identifying opportunities for 
parallelisation or other forms of optimisation.
Traditionally, dataflow analysis has been performed using imperative paradigms, 
which are based on the idea of explicitly specifying how the analysis should be 
performed. 

However, more recently, there has been a growing interest in using 
declarative paradigms for dataflow analysis, which are based on specifying what 
the analysis should accomplish rather than how it should be performed.

In this thesis, we will explore the use of declarative paradigms for dataflow 
analysis, focusing on developing new underlying framework for constructing 
efficient control-flow graphs for performing dataflow 
analysis more effectively and efficiently.

This thesis aims to contribute to the field of static program analysis 
by enhancing the state-of-the-art in dataflow analysis using the declarative paradigm. 
By doing so, we hope to enable researchers and developers to understand and 
analyse complex programs' behaviour and develop more effective tools for 
identifying and fixing errors in software.



\subsection*{Thesis Contribution}

The contributions of this thesis are summarised by the following two papers:

\begin{itemize}
	\item \paperIref.
	This paper presents a new framework for precise source-level control-flow 
	analysis, called IntraCFG. This framework is based on the idea of 
	representing a program's control flow at the source code level, instead of 
	the traditional intermediate representation level. This allows for more 
	precise analysis, as it captures the exact control flow of the original program.
    We also implemented IntraJ, an instance of IntraCFG for the Java programming 
	language, and demonstrated its effectiveness (performance and precision) in 
	two study cases.

	\item \paperIIref.
	This paper introduces JFeature, an extensible tool for automatically extracting and summarising
	the key features of a corpus (i.e., collection) of Java programs. JFeature 
	allows researchers to quickly gain an understanding of the characteristics of a codebase, 
	which can be used to evaluate the effectiveness of their tools and techniques.
	In addition, in this paper, we also present a case study of using JFeature to
	identify the key features of four popular open-source corpora, providing a
	baseline for future research.
\end{itemize}

Together, these contributions advance the state of the art in program analysis and software engineering, and provide practical tools for improving the development and maintenance of software systems.

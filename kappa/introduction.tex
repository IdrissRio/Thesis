\chapter{Introduction}
\section{Introduction}
In the past few decades, software has become increasingly important in all systems.
As our reliance on software increases, the repercussions of bugs can become more severe,
resulting in substantial financial losses and even loss of life.
Well-known examples of software bugs include the \emph{Therac-25} radiation therapy machine~\cite{leveson1993investigation},
the \emph{Mars Climate Orbiter} crash~\cite{Sawyer1999}, resulting in a 327 million dollars loss, and the \emph{Toyota unintended acceleration}~\cite{kane2010toyota}.
Not only can software bugs result in financial losses and harm people, but they can also negatively impact the environment,
as seen in the \emph{Deepwater Horizon} oil spill~\cite{Shafer2010Oil}.
\emph{Program analysis} is a branch of computer science that aims to study the behaviour
and properties of computer programs. Program analysis plays a crucial role in software
development and maintenance, as it helps to ensure the expected functioning of software
systems and to identify potential bugs or security vulnerabilities such as the one mentioned above.
In this thesis, we focus on \emph{static program analysis}. 
Static program analysis (\emph{static analysis} for short) is
\emph{``the art of reasoning about a program's behaviour without executing it''}~\cite{spa}.
It is an essential technique for improving the quality and reliability of software
systems and has been widely used in various applications such
as safety~\cite{cousot2005astree,Blanchet2002} and security~\cite{piskachev2021secucheck,flowDroid,ayewah2008using,Sayar_2022,fink2012wala},
performance optimisation~\cite{aho2007compilers,appel2004modern}, and software maintenance.
Static analysis aims to identify potential errors, bugs, or vulnerabilities
in a program before it is executed.
By examining the source code of a program, static
analysis can provide a detailed and precise understanding of its behaviour, including
its control flow~\cite{allen1970control}, dataflow~\cite{kam1977monotone},
and potential interactions with other system components.



One of the foundamental techniques used in static analysis is \emph{dataflow analysis},
which focuses on the flow of data through a program. Dataflow analysis applications are used to identify
potential sources of errors, such as uninitialised variables or null dereference~\cite{khedker2017data},
and to optimise the program's performance by identifying opportunities for
parallelisation or other forms of optimisation~\cite{aho2007compilers}.
Traditionally, dataflow analysis has been implemented using imperative paradigms,
which are based on the idea of explicitly specifying \emph{how} the analysis should be
performed.
However, more recently, there has been a growing interest in using
declarative paradigms for dataflow analysis, which are based on specifying \emph{what}
the analysis should compute rather than \emph{how}.
The declarative approach leads to a higher-level specification, resulting in improved modularity as
the emphasis is on the desired outcome, rather than the specific steps required to 
achieve it.
There are several declarative languages for specifying dataflow analysis, 
such as \emph{Flowspec}~\cite{smits2020flowspec}, a domain-specific language,
or the functional language \emph{Flix}~\cite{madsen2016programming}. In this thesis, we use \emph{Reference 
Attribute Grammars}~\cite{hedin2000rags} as a declarative approach for implementing dataflow analysis. 
Our implementation is performed upon the \textsc{ExtendJ}~\cite{ekman2007jastadd} Java compiler,
which is written in the \textsc{JastAdd}~\cite{DBLP:journals/entcs/HedinM01} ecosystem.
The \textsc{JastAdd} system supports RAGs, a powerful and flexible
formalism for defining compilers and static analysis tools. 
\textsc{JastAdd} implements demand-driven evaluation algorithms, ensuring that attributes are evaluated only 
when necessary, reducing the overall evaluation time by avoiding redundant
computations.

The primary goal of this thesis is to explore the definition of static analysis 
frameworks utilising RAGs.
We aim to exploit the declarative nature of RAGs, 
combined with the demand-driven evaluation, to study the feasibility of using static 
analysis directly in the development process, i.e., in the Integrated Development Environment (IDE).
Our attention is mainly directed towards intraprocedural analyses, 
including control-flow and dataflow analysis, which involve examining a single procedure.

We started using an older RAG-based framework, \textsc{JastAddJ-intraflow}~\cite{10.1016/j.scico.2012.02.002},
but we encountered some limitations. The most critical issue was that the computed 
control-flow analysis was imprecise, meaning that the 
control-flow graph, which represents all possible paths in a program, did not 
accurately reflect the actual evaluation order of the program.

Therefore, we developed a more general framework called \textsc{IntraCFG}.
We evaluated the precision of \textsc{IntraCFG} with two instances of the framework, 
\textsc{IntraJ} and \textsc{IntraTeal}. 
\textsc{IntraJ} is an instance of \textsc{IntraCFG} for the Java programming language,
and we compared its performance and precision with the industrial tool \textsc{SonarQube}~\cite{sonarqube}. Our results 
showed that \textsc{IntraJ} is more efficient and as precise as \textsc{SonarQube}. 
Additionally, we implemented \textsc{IntraTeal}, an instance of our framework for the teaching 
language TEAL, which has proven to be a supportive tool for teaching control-flow and 
dataflow analyses. Finally, we initiated an exploratory study on integrating
\textsc{IntraJ} and \textsc{IntraTeal} in interactive tools, e.g., IDEs.

While evaluating \textsc{IntraJ}, we found it challenging to locate suitable 
benchmark suites. To overcome this issue, we designed and implemented \textsc{JFeature}, 
a static analysis tool for automatically extracting features from a Java codebase.
\textsc{JFeature} allows researchers and developers to explore the characteristics of a 
codebase, including the use of different Java features across different 
versions, facilitating the identification of suitable corpora for evaluating
their tools and methodologies.

The following sections of this thesis provide a detailed background on the 
underlying concepts and techniques used in this work. We then present our main contributions:
\begin{itemize}
	\item \textsc{IntraCFG}: a language-agnostic framework for intraprocedural analysis.
	 An overview of this contribution can be found in Section~\ref{sec:IntraCFG}, and more details are presented in Paper~\ref{paper:intraj}.
	\item \textsc{IntraJ}: an instance of \textsc{IntraCFG} for the Java programming language. 
	We evaluate \textsc{IntraJ} compared to industry standards, showcasing its 
	performance and precision. An overview of this contribution can be found in Section~\ref{sec:IntraJ},
	with more details provided in Paper~\ref{paper:intraj}.
	\item Integration into IDEs: we have conducted exploratory experiments on integrating
	\textsc{IntraJ} into IDEs. The preliminary results of this contribution are described in Section~\ref{sec:IDEIntegration}.
	\item \textsc{IntraTEAL}: an instance of \textsc{IntraCFG} for teaching control-flow and  dataflow analysis using the TEAL language.
	This contribution is described in detail in Section~\ref{sec:intraTeal}.
	\item \textsc{JFeature}: a tool for automatically extracting and summarising key 
	features of a Java codebase. An overview of this contribution can be found 
	in Section~\ref{sec:JFeature}, with more details presented in Paper~\ref{paper:jfeature}.

\end{itemize}

To conclude, we will present our conclusions and possible future work in Section~\ref{sec:kappa:conclusions}.


% This thesis presents two main contributions.
% \begin{itemize}
% 	\item In Paper~\ref{paper:intraj},  we describe \textsc{IntraCFG}, a language-agnostic framework for building
% 	precise, lightweight source-level CFGs. Unlike other frameworks that
% 	build CFGs on the intermediate representation level, \textsc{IntraCFG} superimposes
% 	CFGs on the program's abstract syntax tree (AST), allowing for more accurate
% 	analysis on the source code level and the construction of CFGs whose shape is not
% 	restricted to the AST structure. We demonstrate the effectiveness
% 	of \textsc{IntraCFG} through the implementation of \textsc{IntraJ}, a Java language
% 	instance. We show that \textsc{IntraJ} is as precise and efficient as existing
% 	static analysis tools. We also show the teaching potential of \textsc{IntraCFG}
% 	through the implementation of \textsc{IntraTeal}, a Teal instance,
% 	 a language designed for teaching \emph{Program Analysis} concepts.
% 	\item In Paper~\ref{paper:jfeature}, we introduce \textsc{JFeature}, a tool for automatically extracting and summarising
% 	the key features of a corpus (i.e., collection) of Java programs. \textsc{JFeature}
% 	allows researchers to understand the characteristics, e.g.,
% 	the use of different Java feature for different Java versions, of a codebase,
% 	which can be used to evaluate the effectiveness of their tools and techniques.
% 	In addition, in Paper~\ref{paper:jfeature}, we present a case study using \textsc{JFeature} to
% 	identify the key features of four popular open-source corpora, providing a
% 	baseline for future research.
% \end{itemize}



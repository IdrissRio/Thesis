\chapter{Introduction}
\section{Introduction}
In this Section, we  introduce the reader of this Thesis to the topic of Static Program Analysis and
the challenges and opportunities associated with the development of precise and fast static analysers.
We present an overview of the design and implementation context of this Thesis, i.e., 
declarative specification of on-demand static analyses meant to be run in the the developer's 
integrated development environment (IDE).
We will also examine and summarise the thesis's contributions and results.



\section{Static Program Analysis}

Program Analysis is the branch of Computer Science that investigates the behaviour
and characteristics of a computer program. Program analysis applications are widely
used in numerous areas of Computer Science, including compiler construction~\cite{aho2007compilers,appel2004modern},
e.g., to perform optimisations; software engineering~\cite{ayewah2008using,dura2021javadl,fink2012wala}, e.g., to detect bugs or
code smells; security~\cite{piskachev2021secucheck,flowDroid}, e.g., to detect vulnerabilities; safety
critical systems~\cite{cousot2005astree,Blanchet2002}, e.g., avionic, automotive, and medical, to detect safety
violations, and many others. In this thesis we will focus on a specific type of program analysis, called
\emph{Static Program Analysis}, or \emph{Static Analysis} for short.


In~\cite{spa}, M\o{}ller et al.~define static program analysis,
as \emph{the art of reasoning about a program's behaviour without executing it}. 
The analysis is performed by processing the program's source code or its 
intermediate representation, to infer and prove properties about the program's behaviour.
The mathematical specification of a static analysis is implemented in a program analysis tool, which
takes as input a program and produces a set of reports for each property that is violated by the program.
Nowadays, static program analysis tools are widely used in industry~\cite{piskachev2022far} and are an
essential part of the software development process in many companies. Static Analysis
is used to detect bugs, code smells and vulnerabilities, but more in general, it is used to
reduce the cost of software development and maintenance by shifting-left the analses execution, i.e., by
performing the analysis in earlier stages in the software development process allowing
to detect and fix subtle and hard to detect misbehaviours of the analysed program.
Often, static analyses are executed as part of a Continuous Integration (CI) pipeline, to
detect and fix bugs before they are deployed to production. For huge projects, such as
the Linux kernel~\cite{rusling1999linux}, static analyses are executed on a daily basis to detect
and fix bugs before they are merged into the main branch of the project.
In order to find and solve defects before they are submitted to version control,
it is necessary to execute analyses even earlier in the software development lifecycle, i.e., during the implementation.
In this thesis, we focused on the implementation of on-demand static analyses that are 
intended to be run in the developer's Integrated Development Environment (IDE).
The goal of these analyses is to detect problems in the code as soon as they are written,
and to provide the developer an explanation of the problem and a set of suggestions to fix it.

Our focus in the work presented in this Thesis is on the specification and implementation of
precise and fast static analyses. We are interested in providing local feedback 
to the developer, i.e., feedback that is provided for a specific piece of code, and not for the whole program, avoiding to 
provide the developer with a list of problems\footnote{also known as \emph{wall of warnings.}} 
that are not related to the code that is currently being edited.




\subsection{The Need of Precise and Fast Static Analyses}
Static analyses are used to detect bugs, code smells, and vulnerabilities, but more in general, they are used to
reduce the cost of software development and maintenance by shifting-left the analses execution. 
In order to be useful, static analysers must be able to detect
most of the problems in the code and to provide the developer with a fast feedback.
Unfortunately, the precision and the speed of static analyses are often in conflict with each other.
In fact, the precision of a static analysis is often inversely proportional to its speed: the more precise
the analysis is, the slower it is. This is due to the fact that the precision of an analysis is often
achieved by performing a more complex analysis, which is more expensive to execute.
Moreover, due to the implications of the Halting Problem, it is not possible to
perform a sound and complete analysis in a finite amount of time leading to the
introduction of false positives and false negatives in the analysis results. 
However, the trade-off between precision and performance can be reduced by 
optimising certain aspects of the analysis, e.g., by using more efficient underlying
data structures, or by performing analyses only when and where necessary. 

In this thesis, we focused on the design and implementation of precise and fast
Control Flow Analysis (CFA)~\cite{spa}.  CFA is a static analysis that computes
the control-flow graph of a program, i.e., a graph overapproximation of all the possible
execution paths of the program. CFA is the underlying analysis for most of the dataflow-based analyses, such as
constant propagation, dead-code elimination, and null-pointer dereferencing detection.
We compute CFA on the source code level of the program avoiding the step of
generating an intermediate representation (IR) of the program. The 
main advantage of this approach is that it allows us to perform CFA on the whole program
in a single pass, without the need to perform a second pass to compute the control-flow graph.
This approach is also more precise than the traditional approach, which is based on the
construction of the control-flow graph on the IR of the program. In fact, the IR is often
simpler (number of instructions) than the source code, and it is not always possible to reconstruct the source code
from the IR. This is due to the fact that the IR is often a simplified representation of the source code,
which is used to perform optimisations on the program. For instance, the IR is often
constructed by removing the syntactic sugar of the source code, such as the \emph{for} loop, 
or the \emph{switch} statement.
On the other hand, designing and implementing CFA (more generally, any static analysis) on the source code level
requires a higher engineering effort than designing and implementing the same analysis on the IR level.
In fact, IRs, such as LLVM-IR~\cite{LLVM}, or Java Bytecode~\cite{javaBytecode}, are 
the target of several compiler frontends. Meaning that the analysis can be implemented once
and then used on any program that is compiled to the IR.




\subsection{Shifting-left Static Analyses}
Shifting left is a generally recommended practice in the software development process.
It is the concept of pushing operations such as testing earlier in the development process.
In the context of static analysis, shifting left refers to the practice of performing
static analysis earlier in the development process, ideally as part of the coding process.
This has numerous significant advantages.


Moving left the execution of static analysers can increase the quality and robustness of the code. 
This enables it to uncover potential flaws within the code, e.g., logical errors,
and other issues that could compromise the software's operation and
dependability. By using static analysis early in the development process, developers
can discover and correct these issues before they become more challenging and time-consuming.
This can increase the chanse to delivery a high-quality (mostly) defect-free product.

Moreover, shifting left static analysis can lower the development process's
time and expense. By detecting and addressing flaws early in the development process,
developers can prevent costly and time-consuming rework later on. This can assist 
in enhancing the development team's overall productivity, allowing them to create high-quality 
software faster and at a lesser cost.

Finally, shifting left static analysis can prevent the introduction of 
potentially severe security flaws in the code. Potential security flaws in the code,
such as SQL injection attacks and cross-site scripting attacks, can be identified 
by static analysis. By using static analysis early in the development process,
developers can patch these vulnerabilities before their exploitation by attackers.
This can aid in protecting the software and its users from the repercussions of
these vulnerabilities.
 
Progress have been made and it is good practice to execute static analyses as part
of a Continuous Integration (CI) pipeline.
Platfoms such as GitHub~\footnote{\url{www.github.com}}, GitLab~\footnote{\url{www.gitlab.com}},
and BitBucket~\footnote{\url{www.bitbucket.com}} provide the possibility to execute static
analyses as part of the CI pipeline. 
However, it is possible to execute static analyses even earlier in the software development lifecycle, 
i.e., during the implementation phase.The natural place to execute static analyses is directly into 
the developer's Integrated Development Environment (IDE). 
Running static analysis in the IDE allows developers to receive feedback on their
code in real-time, as they are writing it.  Moreover, developers to customize the analysis
to their specific needs. Different projects and teams have different requirements for
their code, and different static analysis tools have different strengths and weaknesses.
By running static analysis in the IDE, developers can choose the tools and settings
that are most appropriate for their project, allowing them to get the most value
from the analysis.

But running static analysis in the IDE can be resource-intensive, especially for
large code bases or for analysis that is performed frequently. This can slow down
the IDE and make it less responsive, which can be frustrating for developers and
can impact their productivity. 

% TODO: review this paragraph
In this Thesis, we present the design and implementation of a IDE plugin that
allows developers to run static analysis on their code in real-time, as they are writing it.
The plugin uses the \emph{Language Server Protocol} (LSP)~\cite{LSP} to communicate with the IDE.
We exploit on-demand evaluation to analyse only the code that is currently being edited 
by the developer.
This allows us to reduce the analysis time and the resource consumption of the analysis.







\subsection{Thesis Contributions}




\subsection{Dataflow Analysis}
Among the numerous approaches to static program analysis, dataflow analysis is one of the most used.
Dataflow analysis is utilised to examine the data dependencies between programme entities
(such as variables, expressions, and statements) in order to infer and collect constraints
on the values of the program's entities. Dataflow applications include, for instance,
constant propagation, registers optimisations, and, null-pointer exception detection.
Dataflow analysis is performed on the program's control-flow graph~\cite{allen1970control} (CFG), which is a
directed graph where the nodes represent the program statements or expressions,
and the edges reflect the program control-flow. 
In most cases, the CFG is constructed on a simpler representation of the program, called the intermediate representation (IR).
Different IRs are employed for different purposes. For instance, the Abstract Syntax Tree (AST), which
is the tree representation of the program constructed by the parser, is used to perform semantics checks, e.g.,
name analysis and type analysis.



Like any other software project, the development of a static analysis tool
it is a complex process that involves several steps, including the design of 
flexible and extensible architecture, i.e., a software architecture that is 
easy to extend and maintain. In this thesis, we employ the Declarative Programming
paradigm to overcome this issue. Declarative Programming is a programming 
paradigm that allows the programmer to specify what are the properties of a
program without specifying how to compute them (i.e., without specifying the 
algorithm). We used as declarative formalism an extension of Attribute Grammars~\cite{knuth1968semantics}. Attribute Grammars
were first introduced in 1968 by Donald Knuth and are a formalism for the specification
of the semantics of programming languages. Attribute Grammars are a generalisation
of Context-Free Grammars, which are a formalism for the specification of the syntax
of programming languages. 
Attribute grammars allow to decorate the nodes of an Abstract Syntax Tree (AST) with attributes. 
The attributes are computed by the attribute grammar rules, i.e., the values are specified
through equations. Reference Attribute Grammars~\cite{hedin2000rags} extend Knuth's Attribute Grammars
with reference attributes whose values are references to other AST nodes. 
There are many implementations of Reference Attribute Grammars systems, including
the metacompiler and langauge JastAdd~\cite{ekman2007jastadd}, used to implement large-scale compilers, such as,
ExtendJ~\cite{DBLP:conf/oopsla/EkmanH07}.
Other famous implementations of Reference Attribute Grammars systems are Kiama~\cite{SLOANE2010205} and Silver~\cite{VANWYK201039}, 
the latter being used to implement a C and Java compiler.


In Paper I, we present IntraCFG, a declarative and language-independent framework for 
construcing control-flow graph superimposed on top of the AST. In the same paper, we
present the implementation of a static analysis tool, called IntraJ, that uses IntraCFG
to perform static analysis on Java programs. IntraJ supports two different dataflow analysis:
null-pointer exception analysis and dead assignments detection, that are, a forward and 
backward analysis, respectively. 
In Paper II, we discuss the lack of autmated tools that helps the researcher 
selecting appropriate and relevant corpora for the evaluation of static analysis tools, and more
generally, for the evaluation of any software engineering tool. We present a tool, called
JFeature, that automatically extracts a set of features, divided in different categories,
from Java programs. We used JFeature to extract relevant features introduced in different 
Java versions, and we conducted an empirical study on four well-known Java corpora analysing 
their modernity. We figured out that only one of the four corpora is modern enough to be used
for evaluating Java 8 applications.

 \subsection{Thesis Outline}









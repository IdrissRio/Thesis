We demonstrate the utility of {\intracfg} and {\intraj}\footnote{Based on ExtendJ commit \texttt{a56a2c2} and JastAdd commit \texttt{faf36d2}} by evaluating the client analyses that we describe in Section~\ref{sec:analysis} against similar source-level analyses
from the {\ParentFirst} framework {\jastaddjintraflow}\footnote{
  Using JastAdd2 release~\texttt{2.1.4-36-g18008bb} and JastAddJ-intraflow commit~\texttt{b0b7c00}, restored with the original authors' generous help
} (\tool{JJI}) and the commercial static analyser \tool{SonarQube}, version 8.9.0.43852 (\tool{SQ}).

Our evaluation targets DaCapo benchmarks \project{ANTLR}, \project{FOP}, and \project{PMD}~\cite{DaCapo:paper}, as well as
JFreeChart (\project{JFC}), which is a superset of the \project{Chart} benchmark.
These benchmarks mostly subsume the ones used by JJI~\cite{10.1016/j.scico.2012.02.002},
except for replacing  \project{Bloat} by the more readily available and larger \project{PMD}.
%
Table~\ref{tbl:projectsMetrics} summarise key metrics for the benchmarks and compares {\CFG}s against JJI.
Here, {\intraj}'s AST-unrestricted strategy for building {\CFG}s
reduces the number of nodes and edges by more than 30\%.

% Moreover, SQ performs an interprocedural analysis. Therefore, we filtered-out from the comparison all the reports involving more than one methods.
% We selected all the projects


\begin{table}
\centering
\begin{tabular}{llrrrr}
& LOC & \textsc{Qty} &\textsc{ \intraj}   & \textsc{JJI} & \%  \\
\bottomrule
\multirow{3}{*}{\begin{tabular}[c]{@{}l@{}}\project{ANTLR}\\{v. 2.7.2}\end{tabular}} & \multirow{3}{*}{$33^\cdot737$}
 & \textsc{Roots}& $2^\cdot667$   & $2^\cdot329$& +14.5   \\
\cline{3-6}
&  & \textsc{Nodes}& $76^\cdot925$  & $116^\cdot523$ & -39.9  \\
\cline{3-6}
  &   & \textsc{Edges}& $85^\cdot028$  & $136^\cdot528$ & -37.7   \\
\midrule
\multirow{3}{*}{\begin{tabular}[c]{@{}l@{}}\project{PMD}\\{v. 4.2}\end{tabular}}   &\multirow{3}{*}{$49^\cdot610$}
     & \textsc{Roots}& $6^\cdot215$   & $5^\cdot960$& +4.26   \\
\cline{3-6}
 &    & \textsc{Nodes}& $103^\cdot739$ & $182^\cdot864$ & -43.2   \\
\cline{3-6}
 &    & \textsc{Edges}& $108^\cdot639$ & $202^\cdot842$ & -46.4  \\
\midrule
\multirow{3}{*}{\begin{tabular}[c]{@{}l@{}}\project{JFC}\\{v 1.0.0}\end{tabular}}  &  \multirow{3}{*}{$95^\cdot664$}  &  \textsc{Roots} & $9^\cdot271$   & $7^\cdot889$&  +17.5   \\
\cline{3-6}
  &   & \textsc{Nodes}& $219^\cdot419$ & $331^\cdot368$ &   -33.7 \\
\cline{3-6}
  &   & \textsc{Edges}& $220^\cdot256$ & $363^\cdot642$ & -39.4   \\
\midrule
\multirow{3}{*}{\begin{tabular}[c]{@{}l@{}}\project{FOP}\\{v 0.95}\end{tabular}} & \multirow{3}{*}{$97^\cdot288$}  &  \textsc{Roots} & $11^\cdot327$  & $8^\cdot921$& +26.9   \\
\cline{3-6}
 &    &  \textsc{Nodes} & $239^\cdot096$ & $347^\cdot125$ & -31.1   \\
\cline{3-6}
&     & \textsc{Edges}& $240^\cdot068$ & $379^\cdot269$ & -36.6   \\
\bottomrule
\end{tabular}
\caption{Benchmark size metrics, \textsc{LOC} from \texttt{cloc}. The rest are {\CFG} sizes. \textsc{Roots} is the number of intraprocedural {\CFG}s.  For {\intraj}, this includes static and instance initialisers.}
\label{tbl:projectsMetrics}\end{table}


\subsection{Precision}
To ensure that our analyses yield useful results, we compared them against the results that \tool{JJI} and \tool{SQ} report.

\paragraph{Dead Assignment Analysis}
\tool{JJI} and \tool{SQ} provide subtly different DAA variants.
\tool{JJI}'s DAA corresponds largely to our LVA (Section~\ref{sec:lva}) with minimal filtering, while
\tool{SQ} additionally applies the default value filtering heuristic from Section~\ref{sec:daa}.
We therefore ran two variants of our DAA, the \tool{JJI}-style \tool{IntraJ-NH} (\emph{non-heuristic}), and
the \tool{SQ}-style \tool{IntraJ-H} (\emph{heuristic}).
For \tool{SQ}'s reports, we filtered reports that involved multiple methods (FOP: 24; JFC: 5; PMD: 8), since \tool{SQ} can use interprocedural analysis within one file.


The Venn diagrams in the upper part of Figure~\ref{fig:sets} show the number of DAA reports for each project, categorised by their overlap among the different checkers.
For each category with 20 or fewer reports, we manually inspected all reports.
For other categories, we sampled and manually inspected at least 20 reports or 20\% of the reports (whichever was higher).

The Venn diagrams are dominated by two bug report categories:
reports from the intersection of \tool{IntraJ-NH} and \tool{JJI}, which are initialisations of variables with default values,
and reports from the intersection of all tools.
For these two categories, we found all inspected reports to be true positives, modulo the DAA heuristic (Section~\ref{sec:daa}).
The remaining cases are often false positives:
\project{SQ} reports 8~and 44~false positives in \project{PMD} and \project{FOP}
that seem to largely stem from  imprecision in handling \code{try}-\code{catch} blocks.
Meanwhile, \tool{JJI} reports 9~false positives in \project{PMD} while handling \code{break} statements.
{\intraj} reports two false positives, due to missing two exceptional flow edges for unchecked exceptions (Section~\ref{sec:exceptions}).
These do not affect \tool{JJI} (and possibly \tool{SQ}), since JJI conservatively merges exceptional and regular control flow.

\begin{figure}[H]
  \centering
  % \scalebox{0.8}{
    %% \newcommand{\venncell}[3][100]{%
%%   \node[venncellstylebg] at ($(#2) + (0.5, -0.5)$) {};%
%%   \ifnum#1=100\relax%
%%   \node[] at ($(#2) + (0.5, -0.5)$) {\textcolor{green!20!white}{#3}};%
%%   \else%
%%   \node[] at ($(#2) + (0.5, -0.5)$) {%
%%     \small \textcolor{white}{#3} \\[-0.6em] \footnotesize %
%%     \ifnum#1=0\relax%
%%       \textcolor{red!20!white}{#1\%}%\\%
%%     \else%
%%       \textcolor{white!80!black}{#1\%}%\\%
%%     \fi%
%%   };%
%%   \fi%
%% }
\newcommand{\venncell}[3][100]{%
  \ifnum#1=100\relax%
  \node[venncellstyle] at ($(#2) + (0.5, -0.5)$) {\textcolor{green!50!black}{#3}};%
  \else%
  \node[venncellstyle] at ($(#2) + (0.5, -0.5)$) {%
    \small #3 \\[-0.6em] \footnotesize %
    \ifnum#1=0\relax%
      \textcolor{red!50!black}{#1\%}%\\%
    \else%
      \textcolor{white!40!black}{#1\%}%\\%
    \fi%
  };%
  \fi%
}
\newcommand{\Xclear}[3]{\fill[vennfill, fill=white, opacity=100] #1}
\newcommand{\Xfill}[3]{\fill[vennfill, fill=#2] #1}
\newcommand{\Xdraw}[3]{\draw[venndraw] #1}
\newcommand{\Xdash}[3]{\draw[venndraw, #3, color=#2!80!black] #1}
\newcommand{\Xfilldraw}[3]{\Xfill{#1}{#2}{#3};\Xdraw{#1}{#2}{#3};}


\begin{tikzpicture}[xscale=0.8, yscale=0.55,
    venncellstyle/.style={fill=white, inner sep=1.2pt, rounded corners=1, minimum width=0.6cm, minimum height=0.45cm},
    venncellstylebg/.style={fill=black, opacity=0.5, inner sep=1.2pt, rounded corners=1, minimum width=0.6cm, minimum height=0.45cm},
      vennfill/.style={rounded corners=4, opacity=0.5},
      venndraw/.style={rounded corners=4, black, thick, opacity=0.8},
      legend/.style={draw, rounded corners=0, minimum height=0.35cm, minimum width=2cm, inner sep=0},
      every text node part/.style={align=center}
  ]

  % Draw the background for the legend connectors
  \begin{scope}[xshift=0.25cm, yshift=0.3cm]
    \draw[line width=0.08cm, -, black]
    	(1.25, 0.1) -- ++(0, -0.6);
    %% \draw[line width=0.08cm, -, black]
    %% 	(4.8, 0.1) -- (5, -1.5) -- (5, -2.9) -- (4.8, -3.1);
    %% \draw[line width=0.08cm, -, black]
    %%     (5.35, 0.1) -- (5.15, -0.5) -- (5.15, -1.2) -- (5.35, -1.4);
    \draw[line width=0.08cm, -, black]
	(5, 0.1) -- (5, -1.5) -- (5, -2.4) -- (4.8, -2.6);
    \draw[line width=0.08cm, -, black]
        (5.15, 0.1) -- (5.15, -0.5) -- (5.15, -1.2) -- (5.35, -1.4);
    \draw[line width=0.08cm, -, black]
        (8.9, 0.1) -- ++(0, -0.6);
  \end{scope}

  % For each benchmark
  \foreach \benchname/\x/\y/\results in { % individual venn intersection numbers are {x/y/count/percentage}
		ANTLR/0/0/{    2/-2/112/100, 4/-2/195/100, 1/-2/9/100},
		PMD/5.5/0/{    2/-2/19/100,  3/0/8/0,    2/0/1/100,  4/-1/9/0,   1/-2/4/100,   4/-2/14/100, 4/-3/1/100},
		JFC/0/-4.8/{   2/-2/87/100,  2/0/1/0,    4/-1/2/100, 1/-2/2/100, 4/-2/160/100},
		FOP/5.5/-4.8/{ 2/-2/153/100, 2/0/3/100,  3/0/44/0,   4/-1/5/80,  1/-2/95/100,  4/-2/95/100, 4/-3/4/75}}
  {
    % shift figure to the right coordinates
    \begin{scope}[xshift=\x cm, yshift=\y cm]

    \fill[black, opacity=0.1] (-0.1,0.25) rectangle (5.3,-4.45);

    % First draw transparent fill, then draw hard boundaries w/o transparency
    \foreach \styledraw in {\Xclear, \Xfilldraw}{ % , \Xdash
      % % SonarQube
      \styledraw{(2, 0.2) rectangle (4, -4.2)}{SQ}{dash pattern=on 0.5pt off 0.5pt};
      % % JJI
      \styledraw{(0, -1) rectangle (5, -3)}{JJI}{dash pattern=on 0.4pt off 1pt};
      % IntraJ-N
      \styledraw{(1, 0) rectangle (3, -4.4)}{IJH}{dash pattern=on 0.4pt off 3pt};
      % IntraJ-NN
      \styledraw{(0.2, -2) rectangle (5.2, -4)}{IJnonH}{black};
    }

    % Draw the individual markers
    \foreach \vennx/\venny/\count/\percent in \results {
      \venncell[\percent]{\vennx, \venny}{\count};
    }

    % Draw benchmark name
    \node at (0, -0.55)
          [right, draw, fill=white, minimum height=0.4cm, inner sep=1.5pt] {\textbf{\large \textsf{\benchname}}};
    {}

    \end{scope}
  }

  \fill[opacity=0.1] (-0.1,0.3) rectangle (10.8, 1.15);

  % Draw the legend
  \begin{scope}[xshift=0.25cm, yshift=0.3cm]

    \node at (1.25, 0.4)
          (IJH)
          [legend, fill=white!50!IJH] {\textcolor{black!80!IJH}{{\small \tool{IntraJ-H}}}};

    \node at (3.80, 0.4)
          (IJnonH)
          [legend, fill=white!50!IJnonH] {\textcolor{black!80!IJnonH}{{\small \tool{IntraJ-NH}}}};

    \node at (6.35, 0.4)
          (JJI)
          [legend, fill=white!50!JJI] {\textcolor{black!80!JJI}{{\small \tool{JJI}}}};

    \node at (8.9, 0.4)
          (SQ)
          [legend, fill=white!50!SQ] {\textcolor{black!80!SQ}{{\small \tool{SonarQube}}}};

    {}
    % Manual coordinates to syncronise with the background color bit from before all else is drawn
    \draw[ultra thick, -, white!50!IJH]
    	(1.25, 0.15) -- ++(0, -0.5);
    %% \draw[ultra thick, -, white!50!IJnonH]
    %% 	(4.8, 0.1) -- (5, -1.5) -- (5, -2.9) -- (4.8, -3.1);
    %% \draw[ultra thick, -, white!50!JJI]
    %%     (5.35, 0.1) -- (5.15, -0.5) -- (5.15, -1.2) -- (5.35, -1.4);
    \draw[ultra thick, -, white!50!IJnonH]
	(5, 0.15) -- (5, -1.5) -- (5, -2.4) -- (4.8, -2.6);
    \draw[ultra thick, -, white!50!JJI]
        (5.15, 0.15) -- (5.15, -0.5) -- (5.15, -1.2) -- (5.35, -1.4);
    \draw[ultra thick, -, white!50!SQ]
        (8.9, 0.15) -- ++(0, -0.5);

  \end{scope}

  % top label
  \node[fill=black!70!white, minimum width=8.85cm, inner sep=1pt]
        at (5.35, 1.5)
        {\textcolor{white}{Dead Assignment Analysis}};

  %% bounding box
  \draw (current bounding box.north east) -- (current bounding box.north west) -- (current bounding box.south west) -- (current bounding box.south east) -- cycle;

\end{tikzpicture}

    \newcommand{\venncellB}[3][100]{%
  \ifnum#1=100\relax%
  \node[venncellstyle] at ($(#2) + (0.5, -0.5)$) {\textcolor{green!50!black}{#3}};%
  \else%
  \node[venncellstyle] at ($(#2) + (0.5, -0.5)$) {%
    \small #3 \\[-0.6em] \footnotesize %
    \ifnum#1=0\relax%
      \textcolor{red!50!black}{#1\%}%\\%
    \else%
      \textcolor{white!40!black}{#1\%}%\\%
    \fi%
  };%
  \fi%
}
%% \newcommand{\Xclear}[3]{\fill[vennfill, fill=white, opacity=100] #1}
%% \newcommand{\Xfill}[3]{\fill[vennfill, fill=#2] #1}
%% \newcommand{\Xdraw}[3]{\draw[venndraw] #1}
%% \newcommand{\Xdash}[3]{\draw[venndraw, #3, color=#2!80!black] #1}
%% \newcommand{\Xfilldraw}[3]{\Xfill{#1}{#2}{#3};\Xdraw{#1}{#2}{#3};}


\begin{tikzpicture}[xscale=0.66, yscale=0.58,
    venncellstyle/.style={fill=white, inner sep=1.2pt, rounded corners=1, minimum width=0.5cm, minimum height=0.4cm},
    venncellstylebg/.style={fill=white, opacity=0.5, inner sep=1.2pt, rounded corners=1, minimum width=0.6cm, minimum height=0.45cm},
      benchlabel/.style={fill=white, draw, inner sep=1pt},
      vennfill/.style={rounded corners=4, opacity=0.5},
      venndraw/.style={rounded corners=4, black, thick, opacity=0.8},
      legend/.style={draw, rounded corners=0, minimum height=0.35cm, minimum width=1.8cm, inner sep=0},
      every text node part/.style={align=center}
  ]

  % Draw the background for the legend connectors
  \begin{scope}[xshift=0cm, yshift=0.3cm]
    \draw[line width=0.08cm, -, black]
    	(0.8, 0.1) -- ++(0, -0.6);
    \draw[line width=0.08cm, -, black]
        (12.4, 0.1) -- ++(0, -0.6);
  \end{scope}

  % For each benchmark
  \foreach \benchname/\x/\y/\results in { % individual venn intersection numbers are {x/y/count/percentage}
		ANTLR/0/0/{    0/0/11/57, 1/0/4/100},
		PMD/3.4/0/{    0/0/6/100, 1/0/7/100,  2/0/3/100},
		JFC/6.8/0/{    0/0/17/82,1/0/188/92},
		FOP/10.2/0/{   0/0/29/55, 1/0/24/91}}
  {
    % shift figure to the right coordinates
    \begin{scope}[xshift=\x cm, yshift=\y cm]

    \fill[black, opacity=0.1] (-0.1,0.1) rectangle (3.1,-1.1);

    % First draw transparent fill, then draw hard boundaries w/o transparency
    \foreach \styledraw in {\Xclear, \Xfilldraw}{ % , \Xdash
      % % SonarQube
      \styledraw{(1, 0.1) rectangle (3, -1)}{SQ}{dash pattern=on 0.5pt off 0.5pt};

      % IntraJ-N
      \styledraw{(0.0, 0) rectangle (2, -1.1)}{IJH}{dash pattern=on 0.4pt off 3pt};

    }

    % Draw the individual markers
    \foreach \vennx/\venny/\count/\percent in \results {
      \venncellB[\percent]{\vennx, \venny}{\count};
    }

    % Draw benchmark name
    \node at (1.5, -1.4)
          [minimum height=0.4cm, inner sep=1.5pt] {\textbf{\textsf{\benchname}}};
    {}

    \end{scope}
  }

  \fill[opacity=0.1] (-0.1,0.3) rectangle (13.3, 1.15);

  % Draw the legend
  \begin{scope}[xshift=0cm, yshift=0.3cm]

    \node at (1.45, 0.4)
          (IJH)
          [legend, fill=white!50!IJH] {\textcolor{black!80!IJH}{{\small \tool{IntraJ}}}};

    \node at (11.75, 0.4)
          (SQ)
          [legend, fill=white!50!SQ] {\textcolor{black!80!SQ}{{\small \tool{SonarQube}}}};

    {}
    % Manual coordinates to syncronise with the background color bit from before all else is drawn
    \draw[ultra thick, -, white!50!IJH]
    	(0.8, 0.15) -- ++(0, -0.5);
    \draw[ultra thick, -, white!50!SQ]
        (12.4, 0.15) -- ++(0, -0.5);

    %% % Draw benchmark labels
    %% %[3.39, 5.53, 7.67, 9.81]
    %% \node at (3.39, 0.4)
    %%       (ANTLR)
    %%       [benchlabel] {\textbf{\textsf{ANTLR}}};
    %% \node at (5.53, 0.4)
    %%       (PMD)
    %%       [benchlabel] {\textbf{\textsf{PMD}}};
    %% \node at (7.67, 0.4)
    %%       (JFC)
    %%       [benchlabel] {\textbf{\textsf{JFC}}};
    %% \node at (9.81, 0.4)
    %%       (FOP)
    %%       [benchlabel] {\textbf{\textsf{FOP}}};
  \end{scope}

  % top label
  %% \node[draw, minimum width=13.4cm, inner sep=0]
  \node[fill=black!70!white,  minimum width=8.85cm, inner sep=1pt]
        at (6.6, 1.5)
        {\textcolor{white}{Null Pointer Analysis}};

  %% bounding box
  \draw (current bounding box.north east) -- (current bounding box.north west) -- (current bounding box.south west) -- (current bounding box.south east) -- cycle;

\end{tikzpicture}

  % }
  \caption{Venn diagram: number of reports shared across checkers, and percentage of true positives (unless 100\%).}
\label{fig:sets}
\end{figure}

\paragraph{Null Pointer Analysis}
For NPA (lower part of Figure~\ref{fig:sets}), \tool{IntraJ} detects at least as many reports as \tool{SQ}, except for PMD,
where \tool{SQ} is able to exploit path sensitivity to identify three additional true positives.
Similarly, the false positives reported only by \tool{IntraJ} are mostly due to the lack of path-sensitivity.
Listing~\ref{listing:npaTruePositiveIntraJ} shows a simplified example.

We found that most of the false positives in the intersection of \tool{IntraJ} and \tool{SQ} are due to the lack of interprocedural knowledge.
Listing~\ref{listing:npaTruePositiveSQ} gives a simplified example.
The code here checks if \code{rs} is \code{null} and, if so, calls \code{panic()} to halt execution.
\tool{IntraJ} and \tool{SQ} treat \code{panic()} as a regular method call and infer that \code{rs} may be \code{null} when dereferenced.

\noindent
\begin{minipage}{0.45\textwidth}
\begin{lstlisting}[caption={Simplified false positive reported by {\intraj} }, language=JastAdd, captionpos=b, label=listing:npaTruePositiveIntraJ]
void bar(boolean flag){
  Object o = null;
  if (flag)
    o = new Object();
  if (flag)
    println(o.toString());
}
\end{lstlisting}
\end{minipage}%
\hfill
\begin{minipage}{0.45\textwidth}
\begin{lstlisting}[caption={False positive due to intraprocedural limitations}, language=JastAdd, captionpos=b, label=listing:npaTruePositiveSQ]
void foo(){
  Object rs = getRS();
  if(rs==null)
     // rs can be null
     panic(); //exit(1)
  println(rs.toString());
}
\end{lstlisting}
\end{minipage}%

\subsection{Performance}
We evaluated {\intraj}'s runtime performance with the above benchmarks
on an octa-core Intel i7-11700K 3.6 GHz CPU with 128 GiB DDR4-3200 RAM, running Ubuntu 20.04.2 with Linux 5.8.0-55-generic
and the OpenJDK Runtime Environment Zulu 7.44.0.11-CA-linux build 1.7.0\_292-b07.

We separately measured both \emph{start-up} performance on a cold JVM (restarting the JVM for each run)
and \emph{steady-state} performance (for a single measurement after 49 warmup runs).
We measured each for 50 iterations (i.e., 2500 analysis runs for steady-state) and report median and 95\% confidence intervals
for {\intraj}, JJI, and \tool{SQ}, where applicable.

~\ref{tbl:baseline} and Table~\ref{tbl:startup} summarise our results.
The Baseline column in Table~\ref{tbl:baseline}, gives the times for each tool to load each benchmark, without data flow analyses.
For \tool{SQ}, we report the command line tool run time, with checkers disabled.
For {\intraj} and JJI, this time includes parsing, name, and type analysis.
As JJI uses old versions of JastAdd and ExtendJ (formerly JastAddJ) from 2013, it reports different baselines.
We speculate that the delta is due to bug fixes and other changes to JastAdd and ExtendJ.

We measured DAA and NPA, as well as {\CFG} construction time, on separate runs (column An.sys).
Table~\ref{tbl:startup} has some missing values since \tool{JJI} does not provide an implementation for NPA analysis, and since for \tool{SQ}, we were unable to trigger the construction of the \CFG{} only.
Further, we could~not measure steady state for \tool{SQ}, since we ran it out of the box.

For start-up measurements, we then subtracted the baseline timings.
DAA and NPA timings include on-demand {\CFG} construction time.
For the {\CFG} measurements, we iterated over the entire AST and computed the \Asyn{succ} attribute.

The \%$_{\text{\tool{JJI}}}$ and \%$_{\text{SQ}}$ columns summarise {\intraj}'s performance against JJI and \tool{SQ}
as slowdown (in percent), i.e.\@ {\intraj} was faster whenever we report less than $100$.

We see that {\intraj} is often slower than JJI for small benchmarks, but outperforms it as the benchmarks grow in size, especially in steady-state.
% The trend becomes more visible when we consider the executions in a pre-heated VM. In some circumstances, IntraJ demands 36\% less time to perform DAA compared to JJI.
% This trend is justified by the fact that IntraJ puts more effort into creating smaller but more accurate graphs (w.r.t. JJI).
This trend mirrors the additional overhead that {\intraj} expends on computing smaller, more accurate CFGs:
the difference between the {\CFG} and DAA timings is consistently smaller for {\intraj} than it is for {JJI}, and becomes more significant for larger benchmarks.

For the industrial-strength \tool{SQ}, we observe that its baseline is longer than {\intraj}'s, and an explanation might be that it includes computations that for {\intraj} would be attributed to the analyses.
A strict comparison to \tool{SQ} is therefore difficult, but we observe that {\intraj} is considerably faster including the baseline, at most 3.12 times slower for DAA only, and considerably faster for NPA only, though the latter is likely due to \tool{SQ}'s more expensive interprocedural analysis.

%\subsection{Summary}
Overall, our results support that {\intraj} enables practical data flow analyses, with run-times and precision similar to state-of-the-art tools.
Moreover, the results support that the overhead that {\intraj} invests in refining {\CFG} construction over {\jastaddjintraflow}
pays off: client analyses can amortise this cost,
and we expect this benefit to grow for analyses on taller lattices (e.g., interval or typestate analyses).


\begin{table*}
  \setlength{\tabcolsep}{3.6pt}
  \centering
  \caption{Measure the baseline execution time and 95\% confidence intervals using 50 data points per reported number.}
  \label{tbl:baseline}
  \begin{adjustbox}{angle=0}
  \begin{tabular}{|l|rrr|}
  \hline
  Benchmark & \multicolumn{3}{c|}{Baseline(s)}\\
  \hline
  \multirow{3}{*}{ANTLR} & \Tcenter{\tool{\intraj}} & \Tcenter{JJI}   &	\TcenterR{\tool{SQ}}\\
   & \multirow{2}{*}{2.14$\mCi{0.01}$} &  \multirow{2}{*}{1.34$\mCi{0.01}$}&  \multirow{2}{*}{4.91$\mCi{0.05}$} \\
   &&&\\
  \hline
  \multirow{3}{*}{PMD}    & \Tcenter{\intraj}  & \Tcenter{JJI}   	&	\TcenterR{\tool{SQ}}\\
  &  \multirow{2}{*}{3.56$\mCi{0.01}$}   &  \multirow{2}{*}{2.34$\mCi{0.02}$}  &  \multirow{2}{*}{10.76$\mCi{0.09}$}\\
  &&&\\
  \hline
  \multirow{3}{*}{JFC}   & \Tcenter{\intraj}  & \Tcenter{JJI} 	&	\TcenterR{\tool{SQ}}\\
  &  \multirow{2}{*}{4.29$\mCi{0.01}$} &  \multirow{2}{*}{3.14$\mCi{0.02}$}  &  \multirow{2}{*}{10.81$\mCi{0.11}$} \\
  &&&\\
  \hline
  \multirow{3}{*}{FOP}   & \Tcenter{\intraj}   & \Tcenter{JJI}    				&	\TcenterR{\tool{SQ}} \\
  &  \multirow{2}{*}{4.42$\mCi{0.00}$}  &  \multirow{2}{*}{3.32$\mCi{0.00}$}  &  \multirow{2}{*}{17.20$\mCi{0.12}$}  \\
  &&&\\
  \hline
  \end{tabular}
  \end{adjustbox}
  \end{table*}



  \begin{table*}
    \setlength{\tabcolsep}{3.2pt}
    \centering
    \caption{Benchmark mean execution time (seconds) and 95\% confidence intervals over 50 data points per reported number. We are reporting 
    only confidence intervals greater than 0.02.}
    \label{tbl:startup}
    \begin{adjustbox}{angle=0}
    \begin{tabular}{|l|cccccc|ccc|}
    \hline
    \multirow{2}{*}{Benchmark} &  \multicolumn{6}{c}{Start-up} & \multicolumn{3}{|c|}{Steady state}\\
    \cline{2-10}
    & An.sys & \Tcenter{\tool{IntraJ}}  & \Tcenter{\tool{JJI}}   & \Tcenter{\tool{SQ}}  & \Tcenter{\%$_{\text{\tool{JJI}}}$} & \TcenterR{\%$_{\text{SQ}}$} & \Tcenter{\tool{IntraJ}} & \Tcenter{\tool{JJI}}   &  \TcenterR{\%$_{\text{JJI}}$}    \\
    \hline
    \multirow{3}{*}{ANTLR} &  CFG& 0.29& 0.16&\NAmark&181 &\NAmarkR  & 0.05  & 0.04 & 125  \\
    &DAA& 0.53& 0.43&   0.24$\mCi{0.05}$ &123  & 220 & 0.12  & 0.13 & 92  \\
    &NPA& 0.90 & \NAmark 			    &12.35$\mCi{0.10}$ & \NAmark &7  &  0.27 & \NAmark & \NAmarkR  \\
    \hline
    \multirow{3}{*}{PMD}   	 &CFG& 0.28  & 0.11  &\NAmark & 120 & \NAmarkR&0.07  & 0.06& 116  \\
    &  DAA& 0.47  & 0.39 &0.18$\mCi{0.08}$& 120 &261 & 0.12  & 0.16&  75\\
    & NPA& 0.80& \NAmark 	&12.40$\mCi{0.13}$		 & \NAmark &6&  0.26 & \NAmark & \NAmarkR   \\
    \hline
    \multirow{3}{*}{JFC}  	& CFG& 0.45 & 0.45$\mCi{0.04}$&\NAmark  & 100 &\NAmarkR&  0.12 &  0.12&100   \\
    &  DAA& 0.75& 1.07$\mCi{0.03}$&0.24$\mCi{0.11}$ & 70   &312&0.25& 0.34&  73 \\
    & NPA& 1.62& \NAmark     				  &10.71$\mCi{0.12}$  & \NAmark  &13 &  0.60 & \NAmark & \NAmarkR   \\
    \hline
    \multirow{3}{*}{FOP} 	& CFG     & 0.36 & 0.33&\NAmark & 109  &\NAmarkR  & 0.14  & 0.17 & 82  \\
      &  DAA     & 0.67 & 0.74 & 0.34$\mCi{0.12}$& 90  &197  & 0.26  & 0.39 & 66 \\
     & NPA& 1.42 & \NAmark&19.25$\mCi{0.14}$ & \NAmark  &7 &  0.67& \NAmark & \NAmarkR   \\
  
    \hline
    \end{tabular}
    \end{adjustbox}
    \end{table*}

%Overall context: Static program analysis in development tools.
%And why it is important: Detecting bugs and security vulnerabilities.
Static program analysis plays an important role in software
development, and may help developers
detect subtle bugs such as null pointer exceptions~\cite{hovemeyer2005evaluating}
or security vulnerabilities~\cite{smith2015questions}.

%Zoom in on what this paper is about: Intraprocedural controlflow.
% And the particular challenges involved for development tools
% - want precise analysis results in terms of source code
% - want generic reusable components
% - want high-level code, and possibility to deal with complex flow, like object initialization and exception handling
Many client analyses make use of intraprocedural control-flow analysis, and are dependent on its precision and efficiency for useful results.
Bug checkers and other clients that report to the user must be able to link their results to the source code, so the control-flow analysis itself must also connect to a representation close to the source code, such as an abstract syntax tree (AST).
Current mainstream program analysis tools and IDEs, like SonarQube, ErrorProne, and Eclipse JDT, take this exact approach.

However, building analyses at the AST level typically ties the analysis closely to a particular language and thereby reduces opportunities for reuse.
Furthermore, language semantics can require highly intricate control flow, e.g.\@ for object initialisation and exception handling.

% What this paper does, and what is new
In this paper, we present an approach for developing control-flow analyses and client analyses at the AST level that is based on reference attribute grammars (RAGs) \cite{hedin2000rags} and addresses these challenges.
We build on an earlier approach that also used RAGs~\cite{10.1016/j.scico.2012.02.002} and remove its two main limitations: imprecision and bloat, both caused by limited flexibility in the shape of control-flow graphs ({\CFG}s) that could be built.
Our approach introduces a new generalised framework, \intracfg, that is unrestricted in the shape of the {\CFG}s that it can build.
This improves precision as well as conciseness, in that {\intracfg} connects only AST nodes of interest in the \CFG.
As a case study, we applied {\intracfg} to the Java language, implementing \intraj, a \CFG{} constructor for Java, as an extension of the Java compiler \extendj~\cite{ekman2007jastadd}.
To evaluate the precision and performance of \intraj, we implemented two client {data flow} analyses, one forward and one backward, namely
\emph{Null Pointer Exception} analysis and \emph{Dead Assignment} analysis, respectively.

% Contributions
More precisely, our contributions are as follows:
\begin{itemize}
\item We present \intracfg, a modular and precise language-independent framework
  for intraprocedural \CFG{} construction, implemented using RAGs.
\item We present \intraj, an application of the framework to construct concise and precise CFGs for Java 7.
  We discuss design decisions for what facts to include, and how to reify implicit facts that the AST does not expose directly.
%\item We provide a systematic benchmark test suite of 151 tests for the Java \CFG{}
%  (126 tests for Java 4; 5 for Java 5;  20 for Java 7) with execution time results.
%  To the best of our knowledge, there is no previous work that provides microbenchmarks for Java {\CFG}s.
\item We provide two different client analyses to validate and evaluate
  the framework: \emph{Dead Assignment} analysis, which detects unnecessary assignments,
  and \emph{Null Pointer Exception} analysis, which detects if there exists a path
  in which a \code{NullPointerException} can be thrown.
\item We provide an evaluation of performance and precision for a number of Java subject applications, and compare performance and precision both to the earlier RAGs-based approach and to SonarQube, a current mainstream program analysis tool.
\end{itemize}

% Roadmap
% The rest of this paper is organised as follows: Section~\ref{sec:framework} reviews the basic
% notions related to RAGs and introduces the \intracfg\ framework.
% Section~\ref{sec:implementation} presents \intraj, the application of \intracfg\ for Java, including design decisions and some implementation details.
% We present our client analyses in Section~\ref{sec:analysis} and discuss the results from our evaluation in
% Section~\ref{sec:evaluation_results}.
% Section~\ref{sec:related_works} covers related work, and Section~\ref{sec:conclusions} concludes.



In the rest of this paper, we review RAGs and introduce {\intracfg} (Section~\ref{sec:framework}) and
{\intraj}, along with underlying design decisions and implementation details (Section~\ref{sec:implementation}), present our client analyses (Section~\ref{sec:analysis})
and evaluation (Section~\ref{sec:evaluation_results}), discuss related work
(Section~\ref{sec:related_works}) and conclude (Section~\ref{sec:conclusions}).










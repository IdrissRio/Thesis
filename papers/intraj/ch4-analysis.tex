%%SECTION INTRODUCTION
We demonstrate our framework with two representative data flow analyses:
\emph{Null Pointer Exception} Analysis (NPA), a forward analysis, and \emph{Live Variable} Analysis (LVA), a backward analysis that helps detect useless (`dead') assignments.
These analyses are significant for bug checking and therefore benefit from a close connection to the AST.
%% which exploits the \code{pred} attribute. Conversely, LVA is a \emph{backward} analysis,
%% which uses the \code{succ} attribute. As an application of LVA, we show how we implemented
%% \emph{Dead Assignment} Analysis (DAA), which detects unnecessary assignments.
% \begin{figure}

We first recall the essence of these algorithms on a minimal language that corresponds to the relevant subset of Java: % (Figure~\ref{fig:miniJ}).
\[
\begin{array}{r@{\ }c@{\ }lcll}
  % \vmetavar{s}& \in &
  % \nta{S}	& \Prod	& \vterminal{var}\ \terminal{id}\ \vterminal{=}\ \nt{E}	& \hspace*{-0.5cm}\Gcomment{Declaration} \\
%  &&		& \VB	& \terminal{id}\ \vterminal{=}\ \nt{E}			& \hspace*{-0.5cm}\Gcomment{Assignment} \\
  \vmetavar{e}& \in &
  \nta{E}	& \Prod	& \vterminal{new()}
  			  \VB{}
        		  \vterminal{null}
			  \VB{}
        		  \terminal{id}
			  \VB{}
        		  \terminal{id} \vterminal{.f}
                          \VB{}
                          \terminal{id}\ \vterminal{=}\ \nta{E}
			  % \VB{}
    			  % \terminal{id}\ \vterminal{++}
                           \\
  \vmetavar{v}& \in &
  \terminal{id}	& \Prod	& \vterminal{x}, \ldots
\end{array}
\]
% \caption{Syntax of a simplified subset of Java.}\label{fig:miniJ}
% \end{figure}
%
An expression $e$ can be a \code{new()} object,
\code{null}, the contents of another variable, the result of a field dereference (\code{x.f}), or an assignment \code{x = e}. %, or a post-increment \code{x++}.
The values in our language are an unbounded set of objects $O$ and the distinct \code{null}.  Expressions have the usual Java semantics.
% For simplicity, we assume that the post-increment operator works on all non-null objects; in practice, the ExtendJ type checker catches violations.
Since {\intraj} already captures control flow (on top of {\intracfg}) and name analysis (via ExtendJ), we can ignore
statements and declarations, and safely assume that each \terminal{id} is globally unique.

%Environment defined in the macro file.
%% \begin{align*}
%% &M ::=\ T\ m([T\ x]^*)\{ S \} &\text{Method Declaration}\\
%% &E::=\ Int \mid null \mid id \mid id.f \mid m(E^*) \mid true \mid false \mid E + E \mid \text{++}id \mid id \text{+=} E&\text{Expressions}\\
%% &S::=\ T\ id = E \mid id = E \mid if(E)\ S\ [S] \mid while(E)\ S \mid S;S \mid \{ S\} &\text{Statements}\\
%% &Int ::=\ ... \mid -1 \mid 0 \mid 1 \mid ...&\text{Integer constants}
%% \end{align*}

% ----------------------------------------
\subsection{Null Pointer Exception Analysis}
In our simplified language, a field access \code{x.f}
fails (in Java: throws a Null Pointer Exception) if
\code{x} is \code{null}.
Null Pointer Exception Analysis (NPA) detects whether a given field
dereference \emph{may} fail (e.g.\@ in the SonarQube NPA variant)
or \emph{must} fail (e.g.\@ in the Eclipse JDT NPA variant) and can
alert programmers to inspect and correct this (likely) bug.

In our framework, writing \emph{may} and \emph{must} analyses requires
the same effort;  we here opt for a \emph{may} analysis
over a binary lattice $\mathcal{L}_2$
in which $\top = \textbf{nully}$ signifies \emph{value may be \code{null}} and
$\bot = \textbf{nonnull}$ signifies \emph{value cannot be \code{null}}.

More precisely, we use a product lattice over $\mathcal{L}_2$ that maps each
\emph{access path} $a \in \mathcal{A}$ (e.g. \code{x}; \code{x.f}; \code{x.f.f}; \ldots)
to an element of $\mathcal{L}_2$.  Our analysis then
follows the usual approach for a join data flow
analysis~\cite{cousot1977ai}. Our monotonic
transfer function
$f_{\textit{NPA}} : (\mathcal{A} \to \mathcal{L}_2) \times \nta{E} \to (\mathcal{A} \to \mathcal{L}_2)$ is straightforward:

\[
\begin{array}{llcl}
\multicolumn{2}{r}{f_{\textit{NPA}}(\Gamma, \vterminal{\vmetavar{v} = \vmetavar{e}})}	&=& \Gamma[\vmetavar{v} \mapsto \semNPA{\vmetavar{e}}^\Gamma] \\
% f(\Gamma, \vterminal{\vmetavar{v} = \vmetavar{e}})	&=& \Gamma[\vmetavar{v} \mapsto \semNPA{\vmetavar{e}}^\Gamma] \\
% \\
\textrm{where}&\semNPA{\vterminal{new()}}^\Gamma			&=& \textbf{nonnull}\\
&\semNPA{\vterminal{null}}^\Gamma			&=& \textbf{nully}\\
&\semNPA{\vmetavar{v}}^\Gamma				&=& \Gamma(\vmetavar{v})\\
&\semNPA{\vterminal{\vmetavar{v}.f}}^\Gamma		&=& \Gamma(\vmetavar{v}\vterminal{.f})\\
&\semNPA{\vterminal{\vmetavar{v} = \vmetavar{e}}}^\Gamma	&=& \semNPA{\vmetavar{e}}^\Gamma\\
\end{array}
\]

We do not need to write a recursive transfer function for assignments
nested in other assignments (e.g., \code{x = y = z}), since the
{\CFG} already visits these in evaluation order.

% NPA is a forward analysis, meaning that we starts at the \emph{Entry} node of each function and follow the CFG \code{succ} edges.
% Initially, we map all variables to $\bot$ (``no information'').
% After computing the fixpoint, our NPA conservatively flags
% each \vterminal{\vmetavar{v}.f} for which locally
% $\Gamma(\vmetavar{v}) = \textbf{null}$.

% ----------------------------------------
% figure: NPA UML
\begin{figure}
  \centering
  \scalebox{1.3}{
  \begin{tikzpicture}\scriptsize

    \begin{interface2}[text width=4.1cm]{CFGNode}{\astnode{CFGNode}}{-2.15, 0}
      \attribute{\Asyn{trFun} : $\umlcode{EnvNPA} \to \umlcode{EnvNPA}$}
      \attribute{\Acirc{$\textrm{in}_\textit{NPA}$} : $\umlcode{EnvNPA}$}
      \attribute{\Acirc{$\textrm{out}_\textit{NPA}$} : $\umlcode{EnvNPA}$}
      \operation{\hl{{\mbox{\Asyn{trFun($\Gamma$)} = $\Gamma$}}}}
      \operation{\Acirc{$\textrm{in}_\textit{NPA}$} = $\{ \begin{array}[t]{@{}l} a \mapsto \bigsqcup{n.\Acirc{$\textrm{out}_\textit{NPA}$}(a)} \\ |\  a \in \mathcal{A}, n \in \Asyn{pred}\} \end{array} $}
      \operation{\Acirc{$\textrm{out}_\textit{NPA}$} = \Asyn{trFun}(\Acirc{$\textrm{in}_\textit{NPA}$})}
    \end{interface2}

    \begin{class2}[text width=4.1cm]{VarAccess}{\astnode{VarAccess}}{-2.15, -3.1}
      \attribute{extends \astnode{Expr}. implements \astnode{CFGNode}}
      \attribute{\Ainh{isDeref} : boolean}
      \attribute{\Asyn{canFail} : boolean}
      \attribute{\Ainh{cu} : \astnode{CompilationUnit} \hfill \textbf{[name-api]}}
      \operation{\Asyn{mayBeNull} = (\Acirc{$\textrm{in}_\textit{NPA}$}(\Asyn{decl}) = \textbf{nully})}
      \operation{\Asyn{canFail} = \Asyn{mayBeNull} $\wedge$ \Ainh{isDeref}}
      \operation{$\Asyn{canFail} \Longrightarrow \umlcode{this} \in \Ainh{cu}.\Acoll{NPA}$}
    \end{class2}

    \begin{class2}[text width=3.8cm]{Expr}{\astnode{Expr}}{2.2, 0}
      \attribute{\Asyn{mayBeNull} : boolean}
      \attribute{\Asyn{decl} : $\mathcal{A}$\hfill\nameapi}
      \operation{\hl{{\mbox{\Asyn{mayBeNull} = false}}}}
    \end{class2}

    \begin{class2}[text width=3.8cm]{AssignExpr}{\astnode{AssignExpr} ::= lhs:\astnode{Expr} rhs:\astnode{Expr}}{2.2, -2}
      \attribute{extends \astnode{Expr}. implements \astnode{CFGNode}}
      \operation{\Asyn{trFun($\Gamma$)} = if rhs.\Asyn{mayBeNull}
         $\begin{array}{@{\hspace{0.4cm}}l@{\ }l}
            \umlcode{then} &\Gamma[\umlcode{lhs.\Asyn{decl}} \mapsto \textbf{nully}]\\
            \umlcode{else} &\Gamma[\umlcode{lhs.\Asyn{decl}} \mapsto \textbf{nonnull}]
          \end{array}$}
      \operation{\Asyn{mayBeNull} = rhs.\Asyn{mayBeNull}}
    \end{class2}

    \begin{class2}[text width=3.8cm]{NullExpr}{\astnode{NullExpr}}{2.2, -4.45}
      \attribute{extends \astnode{Expr}. implements \astnode{CFGNode}}
      \operation{\Asyn{mayBeNull} = true}
    \end{class2}

%     \begin{class}[text width=4cm]{TryStmt ::= Block CatchClause* \ldots}{-2.2, -1.65}  % [Finally:Block] /ExceptionHandler:Block/
% %      \attribute{AndOp ::= left:Expr right:Expr}
%       \operation{block.$\mInh$enclosingTry = this
%       \operation{...}
% %      \operation{left.$\mInh$nextNodes = left.$\mInh$nextNodesTT $\union$ left.$\mInh$nextNodesTT}
% %      \operation{right.$\mInh$nextNodesTT = $\mInh$nextNodesTT}
% %      \operation{right.$\mInh$nextNodesFF = $\mInh$nextNodesFF}
% %      \operation{right.$\mInh$nextNodes = right.$\mInh$nextNodesTT $\union$ right.$\mInh$nextNodesTT}
%     \end{class}

  \end{tikzpicture}
  }
  \caption{Partial implementation of our NPA.  We obtain \Asyn{decl} and \Ainh{cu} from
    ExtendJ's name analysis API.}
  \label{fig:NPAuml}
\end{figure}


% ----------------------------------------
% Details and Limitations for the NPA implementation

Our implementation
is field-sensitive
% (i.e., tracks $\Gamma(\vmetavar{v \vterminal{.f}})$ for each field)%
 and control-sensitive
(i.e., it understands that \code{if (x \!= null)\{x.f=1;\}} is safe)%
, but
  array index-insensitive and alias-insensitive.
% (i.e., \code{a=b; b.f=null;} will not impact $\Gamma(\vmetavar{\vterminal{a.f}})$))%
%
Field sensitivity is reached by considering the entire access path chain, while control sensitivity is given by defining new HOAs representing implicit facts, e.g., \code{x \!= null}.

Figure~\ref{fig:NPAuml} shows how we compute
environments $\Gamma \in \umlcode{EnvNPA} = \mathcal{A} \to \mathcal{L}_2$
that capture access paths that may be \code{null} at runtime.
We extend \astnode{CFGNode} with
\Acirc{$\textrm{in}_\textit{NPA}$}, which merges all
evidence that flows in from control flow predecessors, and \Acirc{$\textrm{out}_\textit{NPA}$}, which applies the local transfer function \Asyn{trFun} to
\Acirc{$\textrm{in}_\textit{NPA}$}.  While NPA is a forward analysis, JastAdd's on-demand semantics mean that we
query \emph{backwards}, following \Asyn{pred} edges, when we compute \Acirc{$\textrm{in}_\textit{NPA}$} on demand.
\Acirc{$\textrm{in}_\textit{NPA}$} and \Acirc{$\textrm{out}_\textit{NPA}$} are circular, i.e., can depend on their own output
and compute a fixpoint.

The attributes for \astnode{VarAccess} show how we use this information.
Each \astnode{VarAccess} contributes to \Ainh{cu}.\Acoll{NPA}, the
compilation unit-wide collection attribute of likely \code{null} pointer
dereferences, whenever \Asyn{mayBeNull} holds and when the \astnode{VarAccess} is also a proper prefix
of an access path and must therefore be dereferenced (\Ainh{isDeref},
not shown here).

Our full Java~7 implementation takes up 142 lines of JastAdd code,
excluding data structures but
including control sensitive analysis handling and reporting.

% % ----------------------------------------
% % figure: NPA fragment code
% \begin{lstlisting}[language=JastAdd,caption={{\intraj} specification for NPA (fragment).}, captionpos=b, label=lst:NPAfragment,float=b]
% syn Gamma CFGNode.trFun(Gamma gamma) {return gamma;}
% eq Assign.trFun(Gamma gamma) {
%   Variable decl = getDeclaration();
%   if (mayBeNull()) gamma.put(decl, AbsDomain.NULL);
%   else gamma.put(decl, AbsDomain.NOTNULL);
%   return gamma;
% }
% \end{lstlisting}


\subsection{Live Variable Analysis}\label{sec:lva}
Given a \astnode{CFGNode} $\node$, a variable is \emph{live} iff there exists at least one path from $\node$ to \astnode{Exit} on which $\node$ is read without first being redefined.
An assignment to a variable that is not live (i.e., \emph{dead}) wastes time and complicates the source code, which generally means that it is a bug~\cite{reichenbach2021ticks}.
We can detect this bug with
\emph{Live Variable}/\emph{Liveness} analysis (LVA), a data flow analysis that computes the live variables for each CFG node.

We express LVA as a Gen/Kill analysis, on the powerset lattice
over the set of \emph{live} (local) variables.
Each transfer function adds variables to the set (marks them \emph{live}) or removes them (marks them \emph{dead}).
LVA is a \emph{backward} analysis, starting at the \emph{Exit} node with the assumption that all variables are dead (i.e., with the set of live variables $L = \emptyset$).
The transfer function thus maps from node exit to entry
% $f_{\textit{LVA}}$
 and has the form:
\[
f_{\textit{LVA}}(L, \vterminal{\vmetavar{e}}) =  (L \setminus \textit{def}(\vmetavar{e}))\ \cup\ \textit{use}(\vmetavar{e})
\]
where
\textit{def}(\vmetavar{e}) is the set of variables that \vmetavar{e} assigns to, and
\textit{use}(\vmetavar{e}) is the set of variables that \vmetavar{e} reads.

We encode the $f_{\textit{LVA}}$ using RAGs in a similar way as done in~\cite{10.1016/j.scico.2012.02.002}:
Figure~\ref{fig:LVAuml} shows
%how we directly encode $f_{\textit{LVA}}$ in
our
computation
where circular attributes \Acirc{$\textrm{in}_\textit{LVA}$} and \Acirc{$\textrm{out}_\textit{LVA}$} represent variables live before/after a \astnode{CFGNode}.
%represents variables live \emph{before} a CFG node, and \Acirc{$\textrm{out}_\textit{LVA}$} represents variables live \emph{after} the node.
%Both attributes are defined as circular, since their equations are interdependent.
Here, \Acirc{$\textrm{out}_\textit{LVA}$} reads
from \Asyn{succ} nodes, since we are implementing an on-demand
backward analysis.  \astnode{VarAccess} and \astnode{AssignExpr} override
\Asyn{use} and \Asyn{def}, respectively.  Since the CFG traverses
through the right-hand side of each assignment, this specification
suffices to capture the analysis of our Java language fragment.
Our full implementation for Java~7 takes up 38 lines of code.

% For our simplified language, the transfer function $f$ then becomes:
% \[
% \begin{array}{r}
% f(L, \vterminal{\vmetavar{e}})	= (L \setminus \textit{def}(\vmetavar{e}))\ \cup\ \textit{use}(\vmetavar{e}), \textrm{ where \phantom{fantomenzzzzz}}\\
% \begin{array}{ll|rr}


%  					&& \textit{def}(\vmetavar{e}) =	& \textit{use}(\vmetavar{e}) = \\
% \hline
%   \vmetavar{e} = \vterminal{\vmetavar{v} = \vmetavar{e'}}	&
%                                                                 	&\{ \vmetavar{v} \}	& \textit{use}(\vmetavar{e'}) \\
%   \vmetavar{e} = \vterminal{new()} & \textrm{or } \vmetavar{e} = \vterminal{null}
%                                                                         & \emptyset		& \emptyset \\
%   \vmetavar{e} = \vterminal{\vmetavar{v}} & \textrm{or }
%   \vmetavar{e} = \vterminal{\vmetavar{v}.f}
%                                                                         & \emptyset		& \{ \vmetavar{v} \}
% \end{array}
% \end{array}
% \]


% Listing~\ref{listing:lva} shows the Jastadd specification of LVA.
% In the implementation,  the sets \code{Use} and \code{Def} are represented by the synthesized attributes \code{LVA\_use} and \code{LVA\_def}, respectively.
% The attributes \code{LVA\_use} and \code{LVA\_def} are sets of variable declarations as it is a simple way to  uniquely identify a variable in the program.

% The circular attributes \code{LVA\_in} and \code{LVA\_out} are used to encode the mutual dependent environment expressed by  Equation~\ref{eq:LVA_IN} and Equation~\ref{eq:LVA_OUT}, respectively.
% The least-fix point is always guaranteed to exists since the analysis operates on the finite lattice of variable declarations  and all the used operations are monotone.
% \begin{lstlisting}[language=JastAdd, caption={\jastadd\ specification of LVA}, captionpos=b, label=listing:lva]
% interface Assignments {}
% AssignBitwiseExpr, AssignShiftExpr,AssignMultiplicativeExpr, AssignAdditiveExpr implements Assignments;

% syn BitSet CFGNode.LVA_def()  = emptyBitSet();
% eq VariableDeclarator.LVA_def() =
%   emptyBitSet().mutable().union(singletonValue());
% eq AssignExpr.LVA_def() = getVarDeclSet();
% eq UnaryIncDec.LVA_def() = getVarDeclSet();
% eq ImplicitAssignment.LVA_def() = emptyBitSet();

% syn BitSet CFGNode.LVA_use()  = emptyBitSet();
% eq VarAccess.LVA_use() = getVarDeclSet();
% eq UnaryIncDec.LVA_use() = getVarDeclSet();
% eq Assignments.LVA_use() = getVarDeclSet();


% syn BitSet CFGNode.LVA_out() {
%  BitSet res = emptyBitSet().mutable();
%  for (CFGNode e : succ()) {
%    res.add(e.LVA_in());
%  }
%  return res;
% }

% syn BitSet CFGNode.LVA_in() circular[emptyBitSet()] {
%  return LVA_use().union(LVA_out().compl(LVA_def()));
% }
% \end{lstlisting}

\subsection{Dead Assignment Analysis}\label{sec:daa}
We use dead assignment analysis (DAA) as a straightforward client analysis for LVA.
% Listing~\ref{listing:FOPDeadAssignement} shows one of the dead assignments
% that we detect in the FOP\footnote{src/java/org/apache/fop/fo/properties/CommonHyphenation.java:215} benchmark.
% An example of dead assignment is shown in
% The code has been taken and simplified from the
% It is not immediate to see that the assignment in line 6 is a dead assignment since all the assignments are related to the variable in line 4.
% \CR{simplified by whom?  How large was it?  Does your analysis detect this one?}
% The example shown is clearly a bug, and as a result of this bug, the \code{hashCode} function always returns 0.
% \CR{The readers will know what a dead assignment is, so you can keep the reminder short.}
% \begin{lstlisting}[language=JastAdd, caption={Example of dead assignment.}, captionpos=b, label=listing:FOPDeadAssignement]
% private int hash = 0;
% public int hashCode() {
%     if (hash == 0) {
%         int hash = 17;
%         hash = 37 * hash /* + complex operation */;
%     }
%     return hash;
% }
% \end{lstlisting}
% Once having the LVA analysis ready, implementing the DAA is straightforward.
% In fact, given an assignment represented by the \code{CFGNode} $\node$ is dead iff $\Pi_{\Out}[\node] \cap \Def(\node) = \emptyset$.
% Listing~\ref{listing:DAA} shows  the \jastadd\ specification of the DAA.
% The attribute \code{isDeadAssign}, is used to reduce the number of false positive.
Our implementation of DAA refines the results of LVA with a number of heuristics that we have
adopted from the SonarQube checker.  Specifically, these heuristics suppress warnings in code like the following:
\begin{lstlisting}[language=java]
  String status = ""; // WARNING: unused assignment
  if (...) status = "enabled";
  else     status = "disabled";
\end{lstlisting}
Here, the initial assignment to \code{status} reflects a defensive coding pattern
that ensures that all variables are initialised to some safe default.
We (optionally) suppress warnings like the above under two conditions:
(1) the assignment must be in a variable initialisation, and (2)
the initialiser must be a \emph{common default value}, i.e., one of
$\{ \code{null}, \code{1}, \code{0}, \code{-1}, \code{""}, \code{true}, \code{false} \}$.
Our DAA implementation takes up 62 lines of code.

% \begin{lstlisting}[language=JastAdd, caption={Dead Assignment Analsysis specification.
% The analysis depends on the result of LVA.}, captionpos=b, label=listing:DAA]
% syn boolean CFGNode.isDead() = !LVA_def().compl(LVA_out()).isEmpty();

% syn Boolean CFGNode.isDeadAssign() = false;
% eq VariableDeclarator.LVA_isDeadAssign() = isDead() && getDest().isLocalVar() && ...;

% eq AssignExpr.isDeadAssign() = isDead() && getDest().isLocalVar() && ...;

% \end{lstlisting}

% ----------------------------------------
% figure: LVA UML
\begin{figure}
  \centering
  \scalebox{1.2}{
  \begin{tikzpicture}\scriptsize

    \begin{interface2}[text width=4cm]{CFGNode}{\astnode{CFGNode}}{-2.2, 0}
      \attribute{\Acirc{$\textrm{in}_\textit{LVA}$} : $\mathcal{P}(\mathcal{A})$}
      \attribute{\Acirc{$\textrm{out}_\textit{LVA}$} : $\mathcal{P}(\mathcal{A})$}
      \attribute{\Asyn{def} : $\mathcal{P}(\mathcal{A})$}
      \attribute{\Asyn{use} : $\mathcal{P}(\mathcal{A})$}
      \operation{\Acirc{$\textrm{in}_\textit{LVA}$} = $(\Acirc{$\textrm{out}_\textit{LVA}$} \setminus \Asyn{def}) \cup \Asyn{use}$}
      \operation{\Acirc{$\textrm{out}_\textit{LVA}$} = $\bigcup \{ n\umlcode{.\Acirc{$\textrm{in}_\textit{LVA}$}} | n \in \Asyn{succ}\}$}
      \operation{\hl{{\mbox{\Asyn{def} = $\emptyset$}}}}
      \operation{\hl{{\mbox{\Asyn{use} = $\emptyset$}}}}
    \end{interface2}

    \begin{class2}[text width=3.8cm]{VarAccess}{\astnode{VarAccess}}{2.2, 0}
      \attribute{implements \astnode{CFGNode}}
      \operation{\Asyn{use} = \{ \Asyn{decl} \}}
    \end{class2}

    \begin{class2}[text width=3.8cm]{AssignExpr}{\astnode{AssignExpr} ::= lhs:\astnode{Expr} rhs:\astnode{Expr}}{2.2, -2}
      \attribute{implements \astnode{CFGNode}}
      \operation{\Asyn{def} = \{ lhs.\Asyn{decl} \}}
    \end{class2}

%     \begin{class}[text width=4cm]{TryStmt ::= Block CatchClause* \ldots}{-2.2, -1.65}  % [Finally:Block] /ExceptionHandler:Block/
% %      \attribute{AndOp ::= left:Expr right:Expr}
%       \operation{block.$\mInh$enclosingTry = this
%       \operation{...}
% %      \operation{left.$\mInh$nextNodes = left.$\mInh$nextNodesTT $\union$ left.$\mInh$nextNodesTT}
% %      \operation{right.$\mInh$nextNodesTT = $\mInh$nextNodesTT}
% %      \operation{right.$\mInh$nextNodesFF = $\mInh$nextNodesFF}
% %      \operation{right.$\mInh$nextNodes = right.$\mInh$nextNodesTT $\union$ right.$\mInh$nextNodesTT}
%     \end{class}

  \end{tikzpicture}
  }
  \caption{Partial implementation of our LVA.}
  \label{fig:LVAuml}
  
\end{figure}



%% SECTION INTRODUCTION
\begin{center}
\begin{figure}
\scalebox{1}{
  

% Pattern Info

\tikzset{
pattern size/.store in=\mcSize,
pattern size = 5pt,
pattern thickness/.store in=\mcThickness,
pattern thickness = 0.3pt,
pattern radius/.store in=\mcRadius,
pattern radius = 1pt}
\makeatletter
\pgfutil@ifundefined{pgf@pattern@name@_gmnpobo22}{
\pgfdeclarepatternformonly[\mcThickness,\mcSize]{_gmnpobo22}
{\pgfqpoint{0pt}{0pt}}
{\pgfpoint{\mcSize+\mcThickness}{\mcSize+\mcThickness}}
{\pgfpoint{\mcSize}{\mcSize}}
{
\pgfsetcolor{\tikz@pattern@color}
\pgfsetlinewidth{\mcThickness}
\pgfpathmoveto{\pgfqpoint{0pt}{0pt}}
\pgfpathlineto{\pgfpoint{\mcSize+\mcThickness}{\mcSize+\mcThickness}}
\pgfusepath{stroke}
}}
\makeatother

% Pattern Info

\tikzset{
pattern size/.store in=\mcSize,
pattern size = 5pt,
pattern thickness/.store in=\mcThickness,
pattern thickness = 0.3pt,
pattern radius/.store in=\mcRadius,
pattern radius = 1pt}
\makeatletter
\pgfutil@ifundefined{pgf@pattern@name@_yqrqflual}{
\pgfdeclarepatternformonly[\mcThickness,\mcSize]{_yqrqflual}
{\pgfqpoint{0pt}{0pt}}
{\pgfpoint{\mcSize+\mcThickness}{\mcSize+\mcThickness}}
{\pgfpoint{\mcSize}{\mcSize}}
{
\pgfsetcolor{\tikz@pattern@color}
\pgfsetlinewidth{\mcThickness}
\pgfpathmoveto{\pgfqpoint{0pt}{0pt}}
\pgfpathlineto{\pgfpoint{\mcSize+\mcThickness}{\mcSize+\mcThickness}}
\pgfusepath{stroke}
}}
\makeatother

% Pattern Info

\tikzset{
pattern size/.store in=\mcSize,
pattern size = 5pt,
pattern thickness/.store in=\mcThickness,
pattern thickness = 0.3pt,
pattern radius/.store in=\mcRadius,
pattern radius = 1pt}
\makeatletter
\pgfutil@ifundefined{pgf@pattern@name@_dyl7n5ruv}{
\pgfdeclarepatternformonly[\mcThickness,\mcSize]{_dyl7n5ruv}
{\pgfqpoint{0pt}{0pt}}
{\pgfpoint{\mcSize+\mcThickness}{\mcSize+\mcThickness}}
{\pgfpoint{\mcSize}{\mcSize}}
{
\pgfsetcolor{\tikz@pattern@color}
\pgfsetlinewidth{\mcThickness}
\pgfpathmoveto{\pgfqpoint{0pt}{0pt}}
\pgfpathlineto{\pgfpoint{\mcSize+\mcThickness}{\mcSize+\mcThickness}}
\pgfusepath{stroke}
}}
\makeatother

% Pattern Info

\tikzset{
pattern size/.store in=\mcSize,
pattern size = 5pt,
pattern thickness/.store in=\mcThickness,
pattern thickness = 0.3pt,
pattern radius/.store in=\mcRadius,
pattern radius = 1pt}
\makeatletter
\pgfutil@ifundefined{pgf@pattern@name@_pzazicklx}{
\pgfdeclarepatternformonly[\mcThickness,\mcSize]{_pzazicklx}
{\pgfqpoint{0pt}{0pt}}
{\pgfpoint{\mcSize+\mcThickness}{\mcSize+\mcThickness}}
{\pgfpoint{\mcSize}{\mcSize}}
{
\pgfsetcolor{\tikz@pattern@color}
\pgfsetlinewidth{\mcThickness}
\pgfpathmoveto{\pgfqpoint{0pt}{0pt}}
\pgfpathlineto{\pgfpoint{\mcSize+\mcThickness}{\mcSize+\mcThickness}}
\pgfusepath{stroke}
}}
\makeatother
\tikzset{every picture/.style={line width=0.75pt}} %set default line width to 0.75pt

\begin{tikzpicture}[x=0.75pt,y=0.75pt,yscale=-1,xscale=1]
%uncomment if require: \path (0,518); %set diagram left start at 0, and has height of 518

%Rounded Rect [id:dp7477417660418649]
\draw  [pattern=_gmnpobo22,pattern size=6pt,pattern thickness=0.75pt,pattern radius=0pt, pattern color={rgb, 255:red, 0; green, 0; blue, 0}] (276.64,86.01) .. controls (276.64,83.18) and (278.94,80.89) .. (281.77,80.89) -- (319.38,80.89) .. controls (322.21,80.89) and (324.5,83.18) .. (324.5,86.01) -- (324.5,101.38) .. controls (324.5,104.21) and (322.21,106.5) .. (319.38,106.5) -- (281.77,106.5) .. controls (278.94,106.5) and (276.64,104.21) .. (276.64,101.38) -- cycle ;
%Rounded Rect [id:dp21965449746672105]
\draw  [pattern=_yqrqflual,pattern size=6pt,pattern thickness=0.75pt,pattern radius=0pt, pattern color={rgb, 255:red, 0; green, 0; blue, 0}] (321.5,139.7) .. controls (321.5,136.83) and (323.83,134.5) .. (326.7,134.5) -- (389.8,134.5) .. controls (392.67,134.5) and (395,136.83) .. (395,139.7) -- (395,155.3) .. controls (395,158.17) and (392.67,160.5) .. (389.8,160.5) -- (326.7,160.5) .. controls (323.83,160.5) and (321.5,158.17) .. (321.5,155.3) -- cycle ;
%Straight Lines [id:da07807976749884471]
\draw [color={rgb, 255:red, 74; green, 144; blue, 226 }  ,draw opacity=1 ]   (240,107.33) -- (240,175.75) ;
\draw [shift={(240,178.75)}, rotate = 270] [fill={rgb, 255:red, 74; green, 144; blue, 226 }  ,fill opacity=1 ][line width=0.08]  [draw opacity=0] (8.93,-4.29) -- (0,0) -- (8.93,4.29) -- cycle    ;
%Straight Lines [id:da9553679898860882]
\draw [color={rgb, 255:red, 74; green, 144; blue, 226 }  ,draw opacity=1 ]   (302.45,181.5) -- (287.75,160.94) ;
\draw [shift={(286,158.5)}, rotate = 414.41999999999996] [fill={rgb, 255:red, 74; green, 144; blue, 226 }  ,fill opacity=1 ][line width=0.08]  [draw opacity=0] (8.93,-4.29) -- (0,0) -- (8.93,4.29) -- cycle    ;
%Straight Lines [id:da7038442102744196]
\draw [color={rgb, 255:red, 74; green, 144; blue, 226 }  ,draw opacity=1 ]   (257.5,192.5) -- (277.5,192.5) ;
\draw [shift={(280.5,192.5)}, rotate = 180] [fill={rgb, 255:red, 74; green, 144; blue, 226 }  ,fill opacity=1 ][line width=0.08]  [draw opacity=0] (8.93,-4.29) -- (0,0) -- (8.93,4.29) -- cycle    ;
%Curve Lines [id:da3694545951340712]
\draw [color={rgb, 255:red, 74; green, 144; blue, 226 }  ,draw opacity=1 ]   (299,144.5) .. controls (326.48,162.61) and (315,203.5) .. (321,212.5) .. controls (326.31,220.47) and (337.89,233.13) .. (344.64,238.71) ;
\draw [shift={(347,240.5)}, rotate = 213.69] [fill={rgb, 255:red, 74; green, 144; blue, 226 }  ,fill opacity=1 ][line width=0.08]  [draw opacity=0] (8.93,-4.29) -- (0,0) -- (8.93,4.29) -- cycle    ;
%Straight Lines [id:da5041645716131096]
\draw [color={rgb, 255:red, 74; green, 144; blue, 226 }  ,draw opacity=1 ]   (366.06,210.5) -- (366.45,230.5) ;
\draw [shift={(366,207.5)}, rotate = 88.87] [fill={rgb, 255:red, 74; green, 144; blue, 226 }  ,fill opacity=1 ][line width=0.08]  [draw opacity=0] (8.93,-4.29) -- (0,0) -- (8.93,4.29) -- cycle    ;
%Curve Lines [id:da4059666486839172]
\draw [color={rgb, 255:red, 74; green, 144; blue, 226 }  ,draw opacity=1 ]   (322.5,194.5) .. controls (307.5,222.5) and (309.5,218.5) .. (275.5,218.5) .. controls (243.2,218.5) and (242.49,223.01) .. (242.5,206.32) ;
\draw [shift={(242.5,203.5)}, rotate = 450] [fill={rgb, 255:red, 74; green, 144; blue, 226 }  ,fill opacity=1 ][line width=0.08]  [draw opacity=0] (8.93,-4.29) -- (0,0) -- (8.93,4.29) -- cycle    ;
%Curve Lines [id:da8929371477168765]
\draw [color={rgb, 255:red, 74; green, 144; blue, 226 }  ,draw opacity=1 ]   (299,144.5) .. controls (329.53,126.6) and (317.16,121.27) .. (342.94,96.45) ;
\draw [shift={(345,94.5)}, rotate = 497.08] [fill={rgb, 255:red, 74; green, 144; blue, 226 }  ,fill opacity=1 ][line width=0.08]  [draw opacity=0] (8.93,-4.29) -- (0,0) -- (8.93,4.29) -- cycle    ;
%Rounded Rect [id:dp6635143753739587]
\draw  [fill={rgb, 255:red, 245; green, 166; blue, 35 }  ,fill opacity=1 ] (70.37,87.5) .. controls (70.37,84.74) and (72.61,82.5) .. (75.37,82.5) -- (114.59,82.5) .. controls (117.35,82.5) and (119.59,84.74) .. (119.59,87.5) -- (119.59,102.5) .. controls (119.59,105.26) and (117.35,107.5) .. (114.59,107.5) -- (75.37,107.5) .. controls (72.61,107.5) and (70.37,105.26) .. (70.37,102.5) -- cycle ;
%Rounded Rect [id:dp7717035779626354]
\draw  [fill={rgb, 255:red, 208; green, 2; blue, 27 }  ,fill opacity=1 ] (54.75,138.22) .. controls (54.75,135.61) and (56.86,133.5) .. (59.47,133.5) -- (90.3,133.5) .. controls (92.91,133.5) and (95.02,135.61) .. (95.02,138.22) -- (95.02,152.39) .. controls (95.02,155) and (92.91,157.11) .. (90.3,157.11) -- (59.47,157.11) .. controls (56.86,157.11) and (54.75,155) .. (54.75,152.39) -- cycle ;
%Rounded Rect [id:dp1158519866234119]
\draw  [fill={rgb, 255:red, 245; green, 166; blue, 35 }  ,fill opacity=1 ] (123.52,137.35) .. controls (123.52,134.67) and (125.69,132.5) .. (128.37,132.5) -- (178.83,132.5) .. controls (181.51,132.5) and (183.68,134.67) .. (183.68,137.35) -- (183.68,151.91) .. controls (183.68,154.59) and (181.51,156.77) .. (178.83,156.77) -- (128.37,156.77) .. controls (125.69,156.77) and (123.52,154.59) .. (123.52,151.91) -- cycle ;
%Rounded Rect [id:dp3755457817115687]
\draw   (19.5,87.5) .. controls (19.5,84.74) and (21.74,82.5) .. (24.5,82.5) -- (44.5,82.5) .. controls (47.26,82.5) and (49.5,84.74) .. (49.5,87.5) -- (49.5,102.5) .. controls (49.5,105.26) and (47.26,107.5) .. (44.5,107.5) -- (24.5,107.5) .. controls (21.74,107.5) and (19.5,105.26) .. (19.5,102.5) -- cycle ;
%Straight Lines [id:da38984159395451967]
\draw [color={rgb, 255:red, 74; green, 144; blue, 226 }  ,draw opacity=1 ]   (49.5,95.5) -- (66.23,95.46) ;
\draw [shift={(69.22,95.45)}, rotate = 539.87] [fill={rgb, 255:red, 74; green, 144; blue, 226 }  ,fill opacity=1 ][line width=0.08]  [draw opacity=0] (8.93,-4.29) -- (0,0) -- (8.93,4.29) -- cycle    ;
%Rounded Rect [id:dp8107436910484124]
\draw   (29.07,186.09) .. controls (29.07,183.56) and (31.13,181.5) .. (33.66,181.5) -- (49.45,181.5) .. controls (51.99,181.5) and (54.04,183.56) .. (54.04,186.09) -- (54.04,199.87) .. controls (54.04,202.4) and (51.99,204.46) .. (49.45,204.46) -- (33.66,204.46) .. controls (31.13,204.46) and (29.07,202.4) .. (29.07,199.87) -- cycle ;
%Rounded Rect [id:dp19110661350325764]
\draw  [fill={rgb, 255:red, 208; green, 2; blue, 27 }  ,fill opacity=1 ] (120.91,184.75) .. controls (120.91,181.63) and (123.44,179.11) .. (126.56,179.11) -- (184.28,179.11) .. controls (187.4,179.11) and (189.92,181.63) .. (189.92,184.75) -- (189.92,201.67) .. controls (189.92,204.78) and (187.4,207.31) .. (184.28,207.31) -- (126.56,207.31) .. controls (123.44,207.31) and (120.91,204.78) .. (120.91,201.67) -- cycle ;
%Straight Lines [id:da12719703619254474]
\draw [color={rgb, 255:red, 74; green, 144; blue, 226 }  ,draw opacity=1 ]   (48.26,179.07) -- (63.47,158.11) ;
\draw [shift={(46.5,181.5)}, rotate = 305.96] [fill={rgb, 255:red, 74; green, 144; blue, 226 }  ,fill opacity=1 ][line width=0.08]  [draw opacity=0] (8.93,-4.29) -- (0,0) -- (8.93,4.29) -- cycle    ;
%Straight Lines [id:da8929035973372142]
\draw [color={rgb, 255:red, 74; green, 144; blue, 226 }  ,draw opacity=1 ]   (57.5,194.5) -- (79.5,194.5) ;
\draw [shift={(82.5,194.5)}, rotate = 180] [fill={rgb, 255:red, 74; green, 144; blue, 226 }  ,fill opacity=1 ][line width=0.08]  [draw opacity=0] (8.93,-4.29) -- (0,0) -- (8.93,4.29) -- cycle    ;
%Straight Lines [id:da5803808451550517]
\draw [color={rgb, 255:red, 74; green, 144; blue, 226 }  ,draw opacity=1 ]   (150.78,208.33) -- (150.54,227.5) ;
\draw [shift={(150.5,230.5)}, rotate = 270.72] [fill={rgb, 255:red, 74; green, 144; blue, 226 }  ,fill opacity=1 ][line width=0.08]  [draw opacity=0] (8.93,-4.29) -- (0,0) -- (8.93,4.29) -- cycle    ;
%Curve Lines [id:da8157636188823344]
\draw [color={rgb, 255:red, 74; green, 144; blue, 226 }  ,draw opacity=1 ]   (103.45,182.5) .. controls (104.44,170.68) and (84.63,119.74) .. (144.22,95.58) ;
\draw [shift={(147,94.5)}, rotate = 519.4] [fill={rgb, 255:red, 74; green, 144; blue, 226 }  ,fill opacity=1 ][line width=0.08]  [draw opacity=0] (8.93,-4.29) -- (0,0) -- (8.93,4.29) -- cycle    ;
%Straight Lines [id:da47377520853566013]
\draw  [dash pattern={on 0.84pt off 2.51pt}]  (97.17,108.33) -- (81.92,131.31) ;
\draw [shift={(80.81,132.98)}, rotate = 303.56] [fill={rgb, 255:red, 0; green, 0; blue, 0 }  ][line width=0.08]  [draw opacity=0] (12,-3) -- (0,0) -- (12,3) -- cycle    ;
%Straight Lines [id:da8784077658620598]
\draw  [dash pattern={on 0.84pt off 2.51pt}]  (74.36,159.35) -- (58.68,180.88) ;
\draw [shift={(57.5,182.5)}, rotate = 306.06] [fill={rgb, 255:red, 0; green, 0; blue, 0 }  ][line width=0.08]  [draw opacity=0] (12,-3) -- (0,0) -- (12,3) -- cycle    ;
%Straight Lines [id:da8903909735464935]
\draw  [dash pattern={on 0.84pt off 2.51pt}]  (84.25,158.11) -- (100.03,179.88) ;
\draw [shift={(101.2,181.5)}, rotate = 234.06] [fill={rgb, 255:red, 0; green, 0; blue, 0 }  ][line width=0.08]  [draw opacity=0] (12,-3) -- (0,0) -- (12,3) -- cycle    ;
%Straight Lines [id:da3046411955310566]
\draw  [dash pattern={on 0.84pt off 2.51pt}]  (98.17,107.33) -- (126.83,131.22) ;
\draw [shift={(128.37,132.5)}, rotate = 219.8] [fill={rgb, 255:red, 0; green, 0; blue, 0 }  ][line width=0.08]  [draw opacity=0] (12,-3) -- (0,0) -- (12,3) -- cycle    ;
%Straight Lines [id:da1883032511894267]
\draw  [dash pattern={on 0.84pt off 2.51pt}]  (153.34,157.77) -- (153.77,176.98) ;
\draw [shift={(153.81,178.98)}, rotate = 268.72] [fill={rgb, 255:red, 0; green, 0; blue, 0 }  ][line width=0.08]  [draw opacity=0] (12,-3) -- (0,0) -- (12,3) -- cycle    ;
%Straight Lines [id:da41794667202811797]
\draw  [dash pattern={on 0.84pt off 2.51pt}]  (158.38,208.35) -- (158.49,229.5) ;
\draw [shift={(158.5,231.5)}, rotate = 269.69] [fill={rgb, 255:red, 0; green, 0; blue, 0 }  ][line width=0.08]  [draw opacity=0] (12,-3) -- (0,0) -- (12,3) -- cycle    ;
%Straight Lines [id:da9815332509782984]
\draw [color={rgb, 255:red, 74; green, 144; blue, 226 }  ,draw opacity=1 ]   (87.5,108.5) -- (74.09,129.96) ;
\draw [shift={(72.5,132.5)}, rotate = 302.01] [fill={rgb, 255:red, 74; green, 144; blue, 226 }  ,fill opacity=1 ][line width=0.08]  [draw opacity=0] (8.93,-4.29) -- (0,0) -- (8.93,4.29) -- cycle    ;
%Straight Lines [id:da07789701257979376]
\draw [color={rgb, 255:red, 74; green, 144; blue, 226 }  ,draw opacity=1 ]   (108.04,187.09) -- (126.7,159.26) ;
\draw [shift={(128.37,156.77)}, rotate = 483.83] [fill={rgb, 255:red, 74; green, 144; blue, 226 }  ,fill opacity=1 ][line width=0.08]  [draw opacity=0] (8.93,-4.29) -- (0,0) -- (8.93,4.29) -- cycle    ;
%Straight Lines [id:da8051374521171675]
\draw [color={rgb, 255:red, 74; green, 144; blue, 226 }  ,draw opacity=1 ]   (145.81,157.98) -- (145.81,175.98) ;
\draw [shift={(145.81,178.98)}, rotate = 270] [fill={rgb, 255:red, 74; green, 144; blue, 226 }  ,fill opacity=1 ][line width=0.08]  [draw opacity=0] (8.93,-4.29) -- (0,0) -- (8.93,4.29) -- cycle    ;
%Straight Lines [id:da6629804008910536]
\draw    (205.5,259.5) -- (205.25,74) ;
%Shape: Rectangle [id:dp7380430162169602]
\draw   (122.66,30.14) -- (397,30.14) -- (397,69.84) -- (122.66,69.84) -- cycle ;
%Rounded Rect [id:dp8441557067769366]
\draw  [fill={rgb, 255:red, 208; green, 2; blue, 27 }  ,fill opacity=1 ] (213.13,38.02) .. controls (213.13,36.82) and (214.1,35.85) .. (215.3,35.85) -- (221.83,35.85) .. controls (223.03,35.85) and (224,36.82) .. (224,38.02) -- (224,46.28) .. controls (224,47.48) and (223.03,48.45) .. (221.83,48.45) -- (215.3,48.45) .. controls (214.1,48.45) and (213.13,47.48) .. (213.13,46.28) -- cycle ;
%Rounded Rect [id:dp5132742431212957]
\draw  [pattern=_dyl7n5ruv,pattern size=2.8499999999999996pt,pattern thickness=0.75pt,pattern radius=0pt, pattern color={rgb, 255:red, 0; green, 0; blue, 0}] (127.43,54.41) .. controls (127.43,53.3) and (128.34,52.39) .. (129.45,52.39) -- (135.51,52.39) .. controls (136.63,52.39) and (137.53,53.3) .. (137.53,54.41) -- (137.53,62.19) .. controls (137.53,63.31) and (136.63,64.21) .. (135.51,64.21) -- (129.45,64.21) .. controls (128.34,64.21) and (127.43,63.31) .. (127.43,62.19) -- cycle ;
%Rounded Rect [id:dp6862287065349623]
\draw  [fill={rgb, 255:red, 245; green, 166; blue, 35 }  ,fill opacity=1 ] (213.13,54.57) .. controls (213.13,53.37) and (214.1,52.39) .. (215.3,52.39) -- (221.83,52.39) .. controls (223.03,52.39) and (224,53.37) .. (224,54.57) -- (224,62.83) .. controls (224,64.03) and (223.03,65) .. (221.83,65) -- (215.3,65) .. controls (214.1,65) and (213.13,64.03) .. (213.13,62.83) -- cycle ;
%Straight Lines [id:da3730304449604416]
\draw [color={rgb, 255:red, 74; green, 144; blue, 226 }  ,draw opacity=1 ]   (297.46,41.39) -- (312.12,41.75) ;
\draw [shift={(315.12,41.82)}, rotate = 181.41] [fill={rgb, 255:red, 74; green, 144; blue, 226 }  ,fill opacity=1 ][line width=0.08]  [draw opacity=0] (8.93,-4.29) -- (0,0) -- (8.93,4.29) -- cycle    ;
%Straight Lines [id:da8667219441119757]
\draw  [dash pattern={on 0.84pt off 2.51pt}]  (297.72,57.64) -- (313.12,57.52) ;
\draw [shift={(315.12,57.51)}, rotate = 539.5799999999999] [fill={rgb, 255:red, 0; green, 0; blue, 0 }  ][line width=0.08]  [draw opacity=0] (12,-3) -- (0,0) -- (12,3) -- cycle    ;
%Rounded Rect [id:dp5497681885990634]
\draw  [fill={rgb, 255:red, 253; green, 177; blue, 49 }  ,fill opacity=1 ] (127.68,36.98) .. controls (127.68,36.35) and (128.19,35.85) .. (128.81,35.85) -- (132.2,35.85) .. controls (132.82,35.85) and (133.33,36.35) .. (133.33,36.98) -- (133.33,40.89) .. controls (133.33,41.51) and (132.82,42.02) .. (132.2,42.02) -- (128.81,42.02) .. controls (128.19,42.02) and (127.68,41.51) .. (127.68,40.89) -- cycle ;
%Rounded Rect [id:dp6079824039000589]
\draw  [fill={rgb, 255:red, 187; green, 0; blue, 23 }  ,fill opacity=1 ] (133.36,36.98) .. controls (133.36,36.35) and (133.86,35.85) .. (134.49,35.85) -- (137.87,35.85) .. controls (138.5,35.85) and (139,36.35) .. (139,36.98) -- (139,40.89) .. controls (139,41.51) and (138.5,42.02) .. (137.87,42.02) -- (134.49,42.02) .. controls (133.86,42.02) and (133.36,41.51) .. (133.36,40.89) -- cycle ;
%Rounded Rect [id:dp9168574021765797]
\draw  [fill={rgb, 255:red, 255; green, 255; blue, 255 }  ,fill opacity=1 ] (133.36,43.08) .. controls (133.36,42.45) and (133.86,41.95) .. (134.49,41.95) -- (137.87,41.95) .. controls (138.5,41.95) and (139,42.45) .. (139,43.08) -- (139,46.99) .. controls (139,47.61) and (138.5,48.12) .. (137.87,48.12) -- (134.49,48.12) .. controls (133.86,48.12) and (133.36,47.61) .. (133.36,46.99) -- cycle ;
%Rounded Rect [id:dp8062792968559922]
\draw  [pattern=_pzazicklx,pattern size=2.25pt,pattern thickness=0.75pt,pattern radius=0pt, pattern color={rgb, 255:red, 0; green, 0; blue, 0}] (127.71,43.08) .. controls (127.71,42.45) and (128.21,41.95) .. (128.84,41.95) -- (132.23,41.95) .. controls (132.85,41.95) and (133.36,42.45) .. (133.36,43.08) -- (133.36,46.99) .. controls (133.36,47.61) and (132.85,48.12) .. (132.23,48.12) -- (128.84,48.12) .. controls (128.21,48.12) and (127.71,47.61) .. (127.71,46.99) -- cycle ;
%Straight Lines [id:da20816198540483422]
\draw    (13.5,74) -- (397,74) ;
%Curve Lines [id:da7254383481508059]
\draw [color={rgb, 255:red, 74; green, 144; blue, 226 }  ,draw opacity=1 ]   (139.81,241.98) .. controls (117.81,240.98) and (28.81,234.98) .. (22.81,202.98) .. controls (16.99,171.94) and (23.11,169.62) .. (49.95,160.38) ;
\draw [shift={(52.5,159.5)}, rotate = 520.97] [fill={rgb, 255:red, 74; green, 144; blue, 226 }  ,fill opacity=1 ][line width=0.08]  [draw opacity=0] (8.93,-4.29) -- (0,0) -- (8.93,4.29) -- cycle    ;
%Rounded Rect [id:dp40911960487355714]
\draw   (147.5,86.5) .. controls (147.5,83.74) and (149.74,81.5) .. (152.5,81.5) -- (172.5,81.5) .. controls (175.26,81.5) and (177.5,83.74) .. (177.5,86.5) -- (177.5,101.5) .. controls (177.5,104.26) and (175.26,106.5) .. (172.5,106.5) -- (152.5,106.5) .. controls (149.74,106.5) and (147.5,104.26) .. (147.5,101.5) -- cycle ;
%Rounded Rect [id:dp8410242913474202]
\draw   (256.75,137.22) .. controls (256.75,134.61) and (258.86,132.5) .. (261.47,132.5) -- (292.3,132.5) .. controls (294.91,132.5) and (297.02,134.61) .. (297.02,137.22) -- (297.02,151.39) .. controls (297.02,154) and (294.91,156.11) .. (292.3,156.11) -- (261.47,156.11) .. controls (258.86,156.11) and (256.75,154) .. (256.75,151.39) -- cycle ;
%Rounded Rect [id:dp8142030870528841]
\draw   (223.5,86.5) .. controls (223.5,83.74) and (225.74,81.5) .. (228.5,81.5) -- (248.5,81.5) .. controls (251.26,81.5) and (253.5,83.74) .. (253.5,86.5) -- (253.5,101.5) .. controls (253.5,104.26) and (251.26,106.5) .. (248.5,106.5) -- (228.5,106.5) .. controls (225.74,106.5) and (223.5,104.26) .. (223.5,101.5) -- cycle ;
%Rounded Rect [id:dp7177335687520789]
\draw   (230.07,185.09) .. controls (230.07,182.56) and (232.13,180.5) .. (234.66,180.5) -- (250.45,180.5) .. controls (252.99,180.5) and (255.04,182.56) .. (255.04,185.09) -- (255.04,198.87) .. controls (255.04,201.4) and (252.99,203.46) .. (250.45,203.46) -- (234.66,203.46) .. controls (232.13,203.46) and (230.07,201.4) .. (230.07,198.87) -- cycle ;
%Rounded Rect [id:dp13476719163646977]
\draw   (322.91,184.75) .. controls (322.91,181.63) and (325.44,179.11) .. (328.56,179.11) -- (386.28,179.11) .. controls (389.4,179.11) and (391.92,181.63) .. (391.92,184.75) -- (391.92,201.67) .. controls (391.92,204.78) and (389.4,207.31) .. (386.28,207.31) -- (328.56,207.31) .. controls (325.44,207.31) and (322.91,204.78) .. (322.91,201.67) -- cycle ;
%Straight Lines [id:da9116520299771969]
\draw  [dash pattern={on 0.84pt off 2.51pt}]  (300.17,108.33) -- (284.16,130.87) ;
\draw [shift={(283,132.5)}, rotate = 305.39] [fill={rgb, 255:red, 0; green, 0; blue, 0 }  ][line width=0.08]  [draw opacity=0] (12,-3) -- (0,0) -- (12,3) -- cycle    ;
%Straight Lines [id:da020679763740367174]
\draw  [dash pattern={on 0.84pt off 2.51pt}]  (271.36,159.35) -- (251.79,181.02) ;
\draw [shift={(250.45,182.5)}, rotate = 312.08000000000004] [fill={rgb, 255:red, 0; green, 0; blue, 0 }  ][line width=0.08]  [draw opacity=0] (12,-3) -- (0,0) -- (12,3) -- cycle    ;
%Straight Lines [id:da43199665831248546]
\draw  [dash pattern={on 0.84pt off 2.51pt}]  (276.25,159.11) -- (291,179.71) ;
\draw [shift={(292.17,181.33)}, rotate = 234.39] [fill={rgb, 255:red, 0; green, 0; blue, 0 }  ][line width=0.08]  [draw opacity=0] (12,-3) -- (0,0) -- (12,3) -- cycle    ;
%Straight Lines [id:da8370636232517105]
\draw  [dash pattern={on 0.84pt off 2.51pt}]  (300.17,108.33) -- (336.31,131.42) ;
\draw [shift={(338,132.5)}, rotate = 212.57] [fill={rgb, 255:red, 0; green, 0; blue, 0 }  ][line width=0.08]  [draw opacity=0] (12,-3) -- (0,0) -- (12,3) -- cycle    ;
%Straight Lines [id:da031549326666119426]
\draw  [dash pattern={on 0.84pt off 2.51pt}]  (357,161.5) -- (357,176.75) ;
\draw [shift={(357,178.75)}, rotate = 270] [fill={rgb, 255:red, 0; green, 0; blue, 0 }  ][line width=0.08]  [draw opacity=0] (12,-3) -- (0,0) -- (12,3) -- cycle    ;
%Straight Lines [id:da5768883733441561]
\draw  [dash pattern={on 0.84pt off 2.51pt}]  (357.38,208.35) -- (357.49,227.5) ;
\draw [shift={(357.5,229.5)}, rotate = 269.65999999999997] [fill={rgb, 255:red, 0; green, 0; blue, 0 }  ][line width=0.08]  [draw opacity=0] (12,-3) -- (0,0) -- (12,3) -- cycle    ;
%Rounded Rect [id:dp31052448874186844]
\draw   (347.5,85.5) .. controls (347.5,82.74) and (349.74,80.5) .. (352.5,80.5) -- (372.5,80.5) .. controls (375.26,80.5) and (377.5,82.74) .. (377.5,85.5) -- (377.5,100.5) .. controls (377.5,103.26) and (375.26,105.5) .. (372.5,105.5) -- (352.5,105.5) .. controls (349.74,105.5) and (347.5,103.26) .. (347.5,100.5) -- cycle ;
%Rounded Rect [id:dp039949249090200256]
\draw   (282.07,186.09) .. controls (282.07,183.56) and (284.13,181.5) .. (286.66,181.5) -- (302.45,181.5) .. controls (304.99,181.5) and (307.04,183.56) .. (307.04,186.09) -- (307.04,199.87) .. controls (307.04,202.4) and (304.99,204.46) .. (302.45,204.46) -- (286.66,204.46) .. controls (284.13,204.46) and (282.07,202.4) .. (282.07,199.87) -- cycle ;
%Rounded Rect [id:dp29236322794607517]
\draw   (349.07,236.09) .. controls (349.07,233.56) and (351.13,231.5) .. (353.66,231.5) -- (369.45,231.5) .. controls (371.99,231.5) and (374.04,233.56) .. (374.04,236.09) -- (374.04,249.87) .. controls (374.04,252.4) and (371.99,254.46) .. (369.45,254.46) -- (353.66,254.46) .. controls (351.13,254.46) and (349.07,252.4) .. (349.07,249.87) -- cycle ;
%Rounded Rect [id:dp9083691680730595]
\draw   (83.07,187.09) .. controls (83.07,184.56) and (85.13,182.5) .. (87.66,182.5) -- (103.45,182.5) .. controls (105.99,182.5) and (108.04,184.56) .. (108.04,187.09) -- (108.04,200.87) .. controls (108.04,203.4) and (105.99,205.46) .. (103.45,205.46) -- (87.66,205.46) .. controls (85.13,205.46) and (83.07,203.4) .. (83.07,200.87) -- cycle ;
%Rounded Rect [id:dp28250262123636416]
\draw   (142.07,236.09) .. controls (142.07,233.56) and (144.13,231.5) .. (146.66,231.5) -- (162.45,231.5) .. controls (164.99,231.5) and (167.04,233.56) .. (167.04,236.09) -- (167.04,249.87) .. controls (167.04,252.4) and (164.99,254.46) .. (162.45,254.46) -- (146.66,254.46) .. controls (144.13,254.46) and (142.07,252.4) .. (142.07,249.87) -- cycle ;

% Text Node
\draw  [fill={rgb, 255:red, 255; green, 255; blue, 255 }  ,fill opacity=1 ]  (330,139) -- (389.51,139) -- (389.51,155) -- (330,155) -- cycle  ;
\draw (330.51,141.75) node [anchor=north west][inner sep=0.75pt]  [font=\footnotesize,color={rgb, 255:red, 255; green, 255; blue, 255 }  ,opacity=1 ]  {$\mathtt{\textcolor[rgb]{0,0,0}{ExprStmt}}$};
% Text Node
\draw (54.75,138.95) node [anchor=north west][inner sep=0.75pt]  [font=\footnotesize,color={rgb, 255:red, 255; green, 255; blue, 255 }  ,opacity=1 ]  {$\mathtt{LessOp}$};
% Text Node
\draw (125.52,138.35) node [anchor=north west][inner sep=0.75pt]  [font=\footnotesize,color={rgb, 255:red, 255; green, 255; blue, 255 }  ,opacity=1 ]  {$\mathtt{ExprStmt}$};
% Text Node
\draw (26,91) node [anchor=north west][inner sep=0.75pt]  [font=\footnotesize]  {$\cdots$};
% Text Node
\draw (155,90.21) node [anchor=north west][inner sep=0.75pt]  [font=\footnotesize]   {$\cdots$};
% Text Node
\draw (34.56,186.77) node [anchor=north west][inner sep=0.75pt]  [font=\footnotesize]  {$\mathtt{p} 1$};
% Text Node
\draw (130.46,187.73) node [anchor=north west][inner sep=0.75pt]  [font=\footnotesize,color={rgb, 255:red, 255; green, 255; blue, 255 }  ,opacity=1 ]  {$\mathtt{PostInc}$};
% Text Node
\draw (80.52,88.84) node [anchor=north west][inner sep=0.75pt]  [font=\footnotesize,color={rgb, 255:red, 255; green, 255; blue, 255 }  ,opacity=1 ]  {$\mathtt{While}$};
% Text Node
\draw  [fill={rgb, 255:red, 255; green, 255; blue, 255 }  ,fill opacity=1 ]  (282.77,85) -- (317.77,85) -- (317.77,100) -- (282.77,100) -- cycle  ;
\draw (285.77,88.29) node [anchor=north west][inner sep=0.75pt]  [font=\footnotesize,color={rgb, 255:red, 255; green, 255; blue, 255 }  ,opacity=1 ]  {$\mathtt{\textcolor[rgb]{0,0,0}{While}}$};
% Text Node
\draw (10.5,240.87) node [anchor=north west][inner sep=0.75pt]   [align=left] {\ParentFirst};
% Text Node
\draw (209,240.87) node [anchor=north west][inner sep=0.75pt]   [align=left] {\ASTUnrestricted};
% Text Node
\draw (144.57,53.81) node [anchor=north west][inner sep=0.75pt]  [font=\footnotesize]  {$\mathtt{Not\ in\ CFG}$};
% Text Node
\draw (231.64,36.38) node [anchor=north west][inner sep=0.75pt]  [font=\footnotesize]  {$\mathtt{Misplaced}$};
% Text Node
\draw (229.94,52.81) node [anchor=north west][inner sep=0.75pt]  [font=\footnotesize]  {$\mathtt{Redundant}$};
% Text Node
\draw (317.76,35.79) node [anchor=north west][inner sep=0.75pt]  [font=\footnotesize]  {$\mathtt{succ}$};
% Text Node
\draw (315.69,52.04) node [anchor=north west][inner sep=0.75pt]  [font=\footnotesize]  {$\mathtt{Child\ relation}$};
% Text Node
\draw (144.32,36.51) node [anchor=north west][inner sep=0.75pt]  [font=\footnotesize]  {$\mathtt{AST\ node}$};
% Text Node
\draw (120.17,15.11) node [anchor=north west][inner sep=0.75pt]  [font=\footnotesize] [align=left] {\textit{Legend}};
% Text Node
\draw (256.75,137.95) node [anchor=north west][inner sep=0.75pt]  [font=\footnotesize,color={rgb, 255:red, 0; green, 0; blue, 0 }  ,opacity=1 ]  {$\mathtt{LessOp}$};
% Text Node
\draw (231,90) node [anchor=north west][inner sep=0.75pt]  [font=\footnotesize]   {$\cdots$};
% Text Node
\draw (355,90) node [anchor=north west][inner sep=0.75pt]  [font=\footnotesize]  {$\cdots$};
% Text Node
\draw (235.56,185.77) node [anchor=north west][inner sep=0.75pt]  [font=\footnotesize]  {$\mathtt{p} 1$};
% Text Node
\draw (332.46,187.73) node [anchor=north west][inner sep=0.75pt]  [font=\footnotesize,color={rgb, 255:red, 0; green, 0; blue, 0 }  ,opacity=1 ]  {$\mathtt{PostInc}$};
% Text Node
\draw (287.56,186.77) node [anchor=north west][inner sep=0.75pt]  [font=\footnotesize]  {$\mathtt{p} 2$};
% Text Node
\draw (354.56,236.77) node [anchor=north west][inner sep=0.75pt]  [font=\footnotesize]  {$\mathtt{p} 1$};
% Text Node
\draw (88.56,187.77) node [anchor=north west][inner sep=0.75pt]  [font=\footnotesize]  {$\mathtt{p} 2$};
% Text Node
\draw (147.56,236.77) node [anchor=north west][inner sep=0.75pt]  [font=\footnotesize]  {$\mathtt{p} 1$};
% Text Node
\draw (10.5,20.87) node [anchor=north west][inner sep=0.75pt]   [align=left]{
\begin{lstlisting}[language=java,frame=none]
while(p1<p2){
  p1++;
}
\end{lstlisting}};


\end{tikzpicture}

 }
\caption{In the \ParentFirst\  CFG (left) a parent always precedes its children,
resulting in redundant and misplaced nodes.
The \ASTUnrestricted\ CFG (right) is correct and minimal.}
  \label{fig:semanticsvssyntax}
\end{figure}
\end{center}
%
%\subsection{Reference Attribute Grammars}
Attribute grammars, originally introduced by Knuth~\cite{knuth1968semantics}, 
are declarative specifications that decorate AST nodes with attributes.
Each AST node type can declare attributes and define their values through equations.
There are two main kinds of attributes: \textbf{\emph{synthesised attributes}}, 
defined in the same node, and \textbf{\emph{inherited attributes}}, defined in
a parent or an ancestor node.
Synthesised attributes are useful for propagating information upwards in an AST,
e.g.\@ for basic type analysis of expressions.
Inherited attributes are useful for propagating information downwards,
e.g., for environment information.

Reference Attribute Grammars (RAGs) \cite{hedin2000rags} extend Knuth's
attribute grammars with \textbf{\emph{reference attributes}}, whose values are references to other AST nodes.
Attributes that compute references to AST nodes can declaratively construct graphs that are superimposed on the AST, e.g., {\CFG}s, so that RAGs can propagate information directly along these graph references.

For our implementation, we have used the JastAdd metacompilation system~\cite{hedin2003jastadd}, which supports RAGs as well as the following attribute grammar extensions that we use here:

\begin{description}
\item[Higher-order attributes] (HOAs)~\cite{vogt1989higher} have a value that is a fresh AST subtree, which can itself have attributes.
  HOAs are useful for reifying implicit structures not available in the AST constructed by the parser.
  We use HOAs to reify, for example, control flow for unchecked exceptions and implicit \code{null} assignments.
\item[Circular attributes] are attributes whose equations may transitively depend on their own values~\cite{magnusson2007circular}.
  They support declarative fixpoint computations and can e.g.\@ express {data flow} properties on top of a \CFG.
\item[Collection attributes] are attributes that aggregate any number of \emph{contributions} from anywhere in the AST, or from a bounded AST region~\cite{magnusson2007extending}.
  They simplify e.g.\@ error reporting and the computation of the predecessor relation from the successor relation in a \CFG.
\item[Node type interfaces] are similar to Java interfaces and can be mixed into AST node types.
  They declare e.g.\@ attributes and equations,
  and enable language-independent plugin components in attribute grammars~\cite{fors2020patterns}.
\item[Attribution aspects] are modules that use inter-type declarations to declare a set of attributes, equations, collection contributions, etc.\@ for specific node types~\cite{hedin2003jastadd},
  and mix in interfaces to existing node types.
  They provide a modular extension mechanism for RAGs.
\item[On-demand evaluation,] where attributes are evaluated only if they are used, and with optional caching that prevents reevaluation of attributes used more than once~\cite{jourdan84}.
JastAdd exclusively uses this evaluation strategy.
\end{description}

\subsection{RAG frameworks for control flow}

% What is new in comparison to earlier work
Our work is the second to construct {\CFG}s in a RAG framework, following the earlier {\jastaddjintraflow}~\cite{10.1016/j.scico.2012.02.002}.
{\jastaddjintraflow} constructs \emph{\ParentFirst} CFGs, in the sense that all AST nodes involved in the {\CFG} computation are also part of the {\CFG} and impose their nesting structure, so that the {\CFG} must always pass through all of a node's ancestors before it can reach the node itself.
By contrast, our \intracfg{} framework is \emph{\ASTUnrestricted}, in that the resulting CFG need not follow the syntactic nesting structure.

Figure~\ref{fig:semanticsvssyntax} illustrates this difference between the two approaches for a \code{while} loop in Java.
The left (\ParentFirst) {\CFG} from {\jastaddjintraflow} first flows through the \astnode{While} node to reach the loop condition.
However, the {\CFG} already encodes the flow properties of \astnode{While}, so this flow is unnecessary for data flow analysis.
The same holds for \astnode{ExprStmt}.
We therefore consider these nodes \emph{redundant} for the {\CFG}.
By contrast, our system's {\ASTUnrestricted} {\CFG} on the right skips these two nodes entirely.


The second, more severe concern is that the control flow
in the left {\CFG} in Figure~\ref{fig:semanticsvssyntax} cannot follow Java's evaluation order due to the {\ParentFirst} constraint:
flow passes through the \astnode{PostUnaryInc} node, which represents an update, before passing through the node's subexpression \code{p1}.
This flow would represent an inversion of the actual order of evaluation: the nodes are \emph{misplaced} in the {\CFG}.
Typical client analyses on such a flawed {\CFG} must add additional checks to compensate or otherwise sacrifice soundness or precision in programs where \code{p1} also has a side effect.
By contrast, our {\ASTUnrestricted} {\CFG} on the right addresses this limitation and accurately reflects Java's control flow.

We note that recent work on program analysis~\cite{szaboincrementalizing,helm2020modular} has asserted that attribute grammars restrict computations to be tightly bound to the AST structure.
Our work demonstrates that this generalization does not hold, and that RAGs are an effective framework for efficiently deriving precise {\CFG}s that deviate from the AST structure and for expressing client analyses directly in terms of such derived structures.


\subsection{The {\intracfg} framework}
%%JASTADD
{\intracfg} is our new RAG framework for constructing intraprocedural {\ASTUnrestricted} CFGs, superimposing the graph on the AST.
Figure~\ref{fig:intracfgframework} shows the framework as a UML class diagram.
{\intracfg} is language-independent, and includes interfaces that AST types in an abstract grammar can mix in and specialise to compute the {\CFG} for a particular language.
The figure shows five types:
the \astnode{CFGRoot} interface is intended for subroutines, e.g., methods and constructors, to represent a local {\CFG} with a unique entry and exit node.
  We represent the latter as synthetic AST node types \astnode{Entry} and \astnode{Exit}.
The \astnode{CFGNode} interface marks nodes in the {\CFG}, and each node has reference attributes \Abase{succ} and \Abase{pred} to represent the successor and predecessor edges.
The \astnode{CFGSupport} interface marks AST nodes in a location that may contain \astnode{CFGNode}s.
All \astnode{CFGNode}s are \astnode{CFGSupport} nodes, but \astnode{CFGSupport} nodes that are not \astnode{CFGNode}s can help steer the construction of the {\CFG}.


\begin{figure}
  \begin{tikzpicture}\scriptsize
    \begin{interface2}[text width=5cm]{CFGSupport}{\astnode{CFGSupport}}{2.3, 0}
      \attribute{\Asyn{firstNodes} : $\mathcal{P}({\astnode{CFGNode}})$}
      \attribute{\Ainh{nextNodes} : $\mathcal{P}({\astnode{CFGNode}})$}
      \attribute{\Ainh{nextNodesTT} : $\mathcal{P}({\astnode{CFGNode}})$ \hfill\dfapi}
      \attribute{\Ainh{nextNodesFF} : $\mathcal{P}({\astnode{CFGNode}})$ \hfill\dfapi}
      \operation{\hl{{\mbox{\Asyn{firstNodes} = $\emptyset$}}}}
    \end{interface2}
    \begin{interface2}[text width=2.7cm]{CFGRoot}{\astnode{CFGRoot}}{-2.3 , 0}
%      \inherit{\astnode{CFGSupport}}
      \attribute{\Ahoa{entry} : \astnode{Entry} \hfill\dfapi}
      \attribute{\Ahoa{exit} : \astnode{Exit} \hfill\dfapi}
      \operation{\Ahoa{entry} = new \astnode{Entry}}
      \operation{\Ahoa{exit} = new \astnode{Exit}}
      \operation{*.\Ainh{entry} = \Ahoa{entry}}
      \operation{*.\Ainh{exit} = \Ahoa{exit}}
      \operation{\hl{{\mbox{*.\Ainh{nextNodes} = $\emptyset$}}}}
      \operation{\hl{{\mbox{*.\Ainh{nextNodesTT} = $\emptyset$}}}}
      \operation{\hl{{\mbox{*.\Ainh{nextNodesFF} = $\emptyset$}}}}
    \end{interface2}
    \begin{interface2}[text width=5.8cm]{CFGNode}{\astnode{CFGNode}}{2.3, -2.8}
      \inherit{CFGSupport}
      \attribute{\Asyn{succ} : $\mathcal{P}(\texttt{\astnode{CFGNode}})$ \hfill\dfapi}
      \attribute{\Asyn{pred} : $\mathcal{P}(\texttt{\astnode{CFGNode}})$ \hfill\dfapi}
      \attribute{\Ainh{entry} : \astnode{Entry}}
      \attribute{\Ainh{exit} : \astnode{Exit}}
      \attribute{\Acoll{succInv} : $\mathcal{P}(\texttt{\astnode{CFGNode}})$}
      \attribute{\Acirc{reachable} : boolean}
      \operation{\hl{{\mbox{\Asyn{firstNodes} = $\{$this$\}$}}}}
      \operation{\hl{{\mbox{\Asyn{succ} = \Ainh{nextNodes}}}}}
      \operation{$s \in \Asyn{succ} \implies $this$ \: \in s.\Acoll{succInv}$}
      \operation{\Acirc{reachable} = }
      \operation{\phantom{nn}\Ainh{entry}$ \in \Acoll{succInv}  \: \lor \: \exists p \in \Acoll{succInv}: p.\Acirc{reachable}$}
      \operation{\Asyn{pred} = $\{ p \mid p \in \Acoll{succInv} \land \: p.\Acirc{reachable}\}$}
%      \operation{$\mSyn\texttt{pred} = \{ p \mid p \in \mColl\texttt{succInv} \and p.\mCirc\texttt{isReachable}\}$}
    \end{interface2}
    \begin{class2}[text width=1.7cm]{Entry}{\texttt{\astnode{Entry}}}{-2.3, -3.55}%{1, -6.1}
      \inherit{CFGNode}
      %\attribute{implements \astnode{CFGNode}}
      %\attribute{Entry ::= $\epsilon$}
    \end{class2}
    \begin{class2}[text width=1.7cm]{Exit}{\astnode{Exit}}{-2.3, -4.1}%{3.6, -6.1}
      \inherit{CFGNode}
      %\attribute{implements \astnode{CFGNode}}
      %\attribute{Exit ::= $\epsilon$}
    \end{class2}

    \node[draw] at (-2.3, -5.85) {
      \begin{tabular}{@{}r@{\ }p{1.5cm}@{}}
        \multicolumn{2}{@{}l}{\textbf{Attribute markers}}\\
        \hline
        \textcolor{ATGsym}{\mSyn{}} & synthesized \\
        \textcolor{ATGsym}{\mInh{}} & inherited \\
        \textcolor{ATGsym}{\mHOA{}} & higher-order \\
        \textcolor{ATGsym}{\mColl{}} & collection \\
        \textcolor{ATGsym}{\mCirc{}} & circular \\
        % \hl{{\mbox{\phantom{foo}}}} & default equations \\
        \hfill\dfapi & for client data flow analyses \\
      \end{tabular}
    };
  \end{tikzpicture}
  \caption{The {\intracfg} framework with interfaces \mbox{\astnode{CFGRoot},} \astnode{CFGSupport},  {\astnode{CFGNode}}, and synthetic AST types \astnode{Entry}, \mbox{\astnode{Exit}.} \hl{Highlighted} attribute equations are default equations, intended for overriding.}
  \label{fig:intracfgframework}
\end{figure}


Figure~\ref{fig:intracfgframework} also shows the AST node types' attributes and their types (middle boxes), as well the defining equations (bottom boxes).
%We use a concise pseudo notation for attributes and equations, where $\mSyn$ indicates a synthesised attribute, $\mInh$ an inherited attribute, $\mHOA$ a higher-order attribute (HOA), $\mColl$ a collection attribute, and $\mCirc$ a circular attribute.
Here, we write $\mathcal{P}(\astnode{CFGNode})$ for the type of sets over \astnode{CFGNode}s.
We optionally prefix attribute names with
\textcolor{ATGsym}{\mSyn},
\textcolor{ATGsym}{\mInh},
\textcolor{ATGsym}{\mHOA},
\textcolor{ATGsym}{\mColl}, or
\textcolor{ATGsym}{\mCirc} to highlight the AST traversal underlying their computation.
For the different kinds of attributes, we use the following equations, for attributes \Abase{x} and expressions $e$:
\begin{description}
\item[Synthesised attributes:]
  $\Asyn{x} = e$ defines attribute $\Asyn{x}$ for the local AST node (which we call \code{this}).
%  If $e$ constructs a fresh AST fragment, $\Ahoa{x}$ is a \textbf{HOA}.
\item[Inherited attributes:] $c.\Ainh{x} = e$
  gives AST child node $c$ and its descendants access to $e$ through \Ainh{x}, where $e$ is evaluated in the context of the \code{this} node ($c$'s parent).
  We use the wildcard $*$ for $c$ to broadcast to all children, $*.\Ainh{x} = e$.
\item[Higher-order attributes:] $\Ahoa{x} = e$ where $e$ must construct a fresh AST subtree.
\item[Circular attributes:]
  $\Acirc{x} = e$, where $e$ computes a fixpoint.
  In this paper, boolean circular attributes start at \emph{false} and monotonically grow with $\lor$,
  while set-typed circular attributes start at $\emptyset$ and monotonically grow with $\cup$.
\item[Collection attributes] have no equations, but \emph{contributions}.
  We write $P \implies e \in n.\Acoll{x}$ to contribute the value of expression $e$ to collection attribute \Acoll{x} in node $n$ if $P$ holds.
  In this paper, all collection attributes are sets.
\end{description}

This pseudocode translates straightforwardly to more verbose JastAdd code that uses Java for the right-hand sides in our equations. {\intracfg} is 45 LOC of JastAdd code.%
\footnote{%
\url{https://github.com/lu-cs-sde/IntraJSCAM2021/}%

}

The equations in the framework define some of the attributes, and provide default definitions for others.
To specialise the framework to a particular language, the default equations can be overridden for specific AST node types to capture the control flow of the language.

Client analyses can then use attributes marked as  {\dfapi} in Figure~\ref{fig:intracfgframework}, such as, \Asyn{succ} and \Asyn{pred}, to analyze the {\CFG}.
Since {\CFG} nodes are also AST nodes, it is easy for these analyses to also access syntactic information and attributes from, e.g., type analysis,
as we illustrate in Section~\ref{sec:analysis}.

\begin{figure}
  \begin{tikzpicture}\scriptsize
    \begin{class2}[text width=4cm]{MethodDecl}{\astnode{MethodDecl} ::= ... b:\astnode{Block}}{-2.2, 0}
      \attribute{implements \astnode{CFGRoot}}
%      \attribute{MethodDecl ::= ... b:Block}
      \operation{entry.\Ainh{nextNodes} = b.\Asyn{firstNodes}}
      \operation{b.\Ainh{nextNodes} = $\{\Ainh{exit}\}$}
    \end{class2}
    \begin{class2}[text width=3.8cm]{EQOp}{\astnode{EQOp} ::= left:\astnode{Expr} right:\astnode{Expr}}{-2.2, -1.75}
      \attribute{implements \astnode{CFGNode}}
%      \attribute{EQOp ::= left:Expr right:Expr}
      \operation{\Asyn{firstNodes} = left.\Asyn{firstNodes}}
      \operation{left.\Ainh{nextNodes} = right.\Asyn{firstNodes}}
      \operation{right.\Ainh{nextNodes} = $\{$this$\}$}
    \end{class2}
    \begin{class2}[text width=4cm]{AndOp}{\astnode{AndOp} ::= left:\astnode{Expr} right:\astnode{Expr}}{2.2, 0}
      \attribute{implements \astnode{CFGSupport}}
%      \attribute{AndOp ::= left:Expr right:Expr}
      \operation{\Asyn{firstNodes} = left.\Asyn{firstNodes}}
      \operation{left.\Ainh{nextNodesTT} = right.\Asyn{firstNodes}}
      \operation{left.\Ainh{nextNodesFF} = \Ainh{nextNodesFF}}
      \operation{...}
%      \operation{left.\Ainh{nextNodes} = left.\Ainh{nextNodes}TT $\union$ left.\Ainh{nextNodes}TT}
%      \operation{right.\Ainh{nextNodes}TT = \Ainh{nextNodes}TT}
%      \operation{right.\Ainh{nextNodes}FF = \Ainh{nextNodes}FF}
%      \operation{right.\Ainh{nextNodes} = right.\Ainh{nextNodes}TT $\union$ right.\Ainh{nextNodes}TT}
    \end{class2}
    \begin{class2}[text width=3.8cm]{ReturnStmt}{\astnode{ReturnStmt} ::= [e:\astnode{Expr}]}{2.2, -2.2}
      \attribute{implements \astnode{CFGNode}}
%      \attribute{ReturnStmt ::= [e:Expr]}
      \operation{...}
      \operation{\Asyn{succ} = $\{\Ainh{exit}\}$}
    \end{class2}
  \end{tikzpicture}
  \caption{Example application of the {\intracfg} framework.}
  \label{fig:exampleuml}
\end{figure}

\begin{lstlisting}[language=JastAdd, caption=JastAdd translation of \astnode{EQOp} in Figure~\ref{fig:exampleuml}., label=listing:jastadd-eqop, captionpos=b,float=t]
EQOp ::= Left:Expr Right:Expr;   // Abstract grammar
EQOp implements CFGNode;
eq EQOp.firstNodes() = getLeft().firstNodes();
eq EQOp.getLeft().nextNodes() = getRight().firstNodes();
eq EQOp.getRight().nextNodes() = SmallSet<CFGNode>.singleton(this);
\end{lstlisting}

\subsection{Computing the successor attributes}

To compute the \Asyn{succ} attributes, we use the helper attributes \Asyn{firstNodes} and \Ainh{nextNodes}.
Given an AST subtree $t$, its \Asyn{firstNodes} contain the first \astnode{CFGNode} \emph{within or after} $t$ that should be executed, if such a node exists.
If not, \Asyn{firstNodes} is empty.
The framework in Figure~\ref{fig:intracfgframework} shows the default definitions for this attribute: the empty set for a \astnode{CFGSupport} node, and the node itself for a \astnode{CFGNode}.

The \Ainh{nextNodes} attribute contains the \astnode{CFGNode}s that are \emph{outside} $t$, and that would immediately follow the last executed \astnode{CFGNode} within $t$, disregarding abrupt execution flow like returns and exceptions.
By default, the \Asyn{succ} attribute is defined as equal to \Ainh{nextNodes}.

\begin{figure*}
  \begin{lstlisting}[language=JastAdd]
    void foo(int p1, int p2, boolean b1){
      if (p1==p2 && b1) p1 = 0;
    }
    \end{lstlisting}
\centering
\scalebox{0.7}{
	

\tikzset{every picture/.style={line width=0.75pt}} %set default line width to 0.75pt

\begin{tikzpicture}[x=0.75pt,y=0.75pt,yscale=-1,xscale=1]
%uncomment if require: \path (0,806); %set diagram left start at 0, and has height of 806

%Rounded Rect [id:dp8057613875937755]
\draw  [fill={rgb, 255:red, 224; green, 219; blue, 219 }  ,fill opacity=1 ] (96.67,62.72) .. controls (96.67,59.18) and (99.54,56.3) .. (103.09,56.3) -- (172.41,56.3) .. controls (175.95,56.3) and (178.83,59.18) .. (178.83,62.72) -- (178.83,81.98) .. controls (178.83,85.52) and (175.95,88.4) .. (172.41,88.4) -- (103.09,88.4) .. controls (99.54,88.4) and (96.67,85.52) .. (96.67,81.98) -- cycle ;
%Rounded Rect [id:dp9321852876702529]
\draw  [fill={rgb, 255:red, 224; green, 219; blue, 219 }  ,fill opacity=1 ] (96.67,118.89) .. controls (96.67,115.34) and (99.54,112.47) .. (103.09,112.47) -- (172.41,112.47) .. controls (175.95,112.47) and (178.83,115.34) .. (178.83,118.89) -- (178.83,138.15) .. controls (178.83,141.69) and (175.95,144.57) .. (172.41,144.57) -- (103.09,144.57) .. controls (99.54,144.57) and (96.67,141.69) .. (96.67,138.15) -- cycle ;
%Rounded Rect [id:dp5527002642013513]
\draw  [fill={rgb, 255:red, 224; green, 219; blue, 219 }  ,fill opacity=1 ] (62.44,176.1) .. controls (62.44,173.31) and (64.7,171.05) .. (67.49,171.05) -- (109.34,171.05) .. controls (112.13,171.05) and (114.39,173.31) .. (114.39,176.1) -- (114.39,191.27) .. controls (114.39,194.06) and (112.13,196.32) .. (109.34,196.32) -- (67.49,196.32) .. controls (64.7,196.32) and (62.44,194.06) .. (62.44,191.27) -- cycle ;
%Rounded Rect [id:dp223278190872505]
\draw   (35.05,219.43) .. controls (35.05,216.64) and (37.32,214.37) .. (40.11,214.37) -- (78.73,214.37) .. controls (81.52,214.37) and (83.78,216.64) .. (83.78,219.43) -- (83.78,234.6) .. controls (83.78,237.39) and (81.52,239.65) .. (78.73,239.65) -- (40.11,239.65) .. controls (37.32,239.65) and (35.05,237.39) .. (35.05,234.6) -- cycle ;
%Rounded Rect [id:dp5657207837294708]
\draw   (97.48,219.43) .. controls (97.48,216.64) and (99.74,214.37) .. (102.53,214.37) -- (124.64,214.37) .. controls (127.43,214.37) and (129.69,216.64) .. (129.69,219.43) -- (129.69,234.6) .. controls (129.69,237.39) and (127.43,239.65) .. (124.64,239.65) -- (102.53,239.65) .. controls (99.74,239.65) and (97.48,237.39) .. (97.48,234.6) -- cycle ;
%Straight Lines [id:da6219776930622651]
\draw  [dash pattern={on 0.84pt off 2.51pt}]  (123.02,144.57) -- (90.61,169.04) ;
\draw [shift={(89.02,170.24)}, rotate = 322.94] [fill={rgb, 255:red, 0; green, 0; blue, 0 }  ][line width=0.08]  [draw opacity=0] (12,-3) -- (0,0) -- (12,3) -- cycle    ;
%Straight Lines [id:da7821070677058984]
\draw  [dash pattern={on 0.84pt off 2.51pt}]  (81.77,198.33) -- (64.01,212.33) ;
\draw [shift={(62.44,213.57)}, rotate = 321.74] [fill={rgb, 255:red, 0; green, 0; blue, 0 }  ][line width=0.08]  [draw opacity=0] (12,-3) -- (0,0) -- (12,3) -- cycle    ;
%Straight Lines [id:da14017532215712158]
\draw  [dash pattern={on 0.84pt off 2.51pt}]  (95.33,197.21) -- (110.83,212.65) ;
\draw [shift={(112.25,214.06)}, rotate = 224.89] [fill={rgb, 255:red, 0; green, 0; blue, 0 }  ][line width=0.08]  [draw opacity=0] (12,-3) -- (0,0) -- (12,3) -- cycle    ;
%Straight Lines [id:da9326975142197422]
\draw  [dash pattern={on 0.84pt off 2.51pt}]  (151.04,144.57) -- (168.63,163.88) ;
\draw [shift={(169.97,165.36)}, rotate = 227.68] [fill={rgb, 255:red, 0; green, 0; blue, 0 }  ][line width=0.08]  [draw opacity=0] (12,-3) -- (0,0) -- (12,3) -- cycle    ;
%Straight Lines [id:da10726145269729981]
\draw  [dash pattern={on 0.84pt off 2.51pt}]  (139.36,91.61) -- (139.36,108.87) ;
\draw [shift={(139.36,110.87)}, rotate = 270] [fill={rgb, 255:red, 0; green, 0; blue, 0 }  ][line width=0.08]  [draw opacity=0] (12,-3) -- (0,0) -- (12,3) -- cycle    ;
%Straight Lines [id:da8780944343405684]
\draw  [dash pattern={on 0.84pt off 2.51pt}]  (138.42,37.04) -- (138.54,51.09) ;
\draw [shift={(138.55,53.09)}, rotate = 269.52] [fill={rgb, 255:red, 0; green, 0; blue, 0 }  ][line width=0.08]  [draw opacity=0] (12,-3) -- (0,0) -- (12,3) -- cycle    ;
%Curve Lines [id:da2616418220552038]
\draw [color={rgb, 255:red, 126; green, 211; blue, 33 }  ,draw opacity=1 ][line width=1.5]  [dash pattern={on 5.63pt off 4.5pt}]  (95.87,133.73) .. controls (88.62,143.6) and (44.66,133.23) .. (32.98,160.51) .. controls (21.3,187.79) and (22.92,189.4) .. (16.87,207.45) .. controls (11.29,224.15) and (14.57,243.34) .. (23.06,256.92) ;
\draw [shift={(25.23,260.11)}, rotate = 233.57] [fill={rgb, 255:red, 126; green, 211; blue, 33 }  ,fill opacity=1 ][line width=0.08]  [draw opacity=0] (11.61,-5.58) -- (0,0) -- (11.61,5.58) -- cycle    ;
%Curve Lines [id:da7836156464726156]
\draw [color={rgb, 255:red, 126; green, 211; blue, 33 }  ,draw opacity=1 ][line width=1.5]  [dash pattern={on 5.63pt off 4.5pt}]  (109.72,240.5) .. controls (83.68,273.41) and (139.02,269.33) .. (120.49,243.77) ;
\draw [shift={(118.18,240.9)}, rotate = 408.36] [fill={rgb, 255:red, 126; green, 211; blue, 33 }  ,fill opacity=1 ][line width=0.08]  [draw opacity=0] (11.61,-5.58) -- (0,0) -- (11.61,5.58) -- cycle    ;
%Curve Lines [id:da7516463381612492]
\draw [color={rgb, 255:red, 126; green, 211; blue, 33 }  ,draw opacity=1 ][line width=1.5]  [dash pattern={on 5.63pt off 4.5pt}]  (169.35,292.3) .. controls (143.31,325.2) and (198.66,321.13) .. (180.12,295.56) ;
\draw [shift={(177.81,292.7)}, rotate = 408.36] [fill={rgb, 255:red, 126; green, 211; blue, 33 }  ,fill opacity=1 ][line width=0.08]  [draw opacity=0] (11.61,-5.58) -- (0,0) -- (11.61,5.58) -- cycle    ;
%Curve Lines [id:da5051478668127928]
\draw [color={rgb, 255:red, 126; green, 211; blue, 33 }  ,draw opacity=1 ][line width=1.5]  [dash pattern={on 5.63pt off 4.5pt}]  (220.1,292.47) .. controls (194.06,325.38) and (249.4,321.3) .. (230.87,295.74) ;
\draw [shift={(228.55,292.87)}, rotate = 408.36] [fill={rgb, 255:red, 126; green, 211; blue, 33 }  ,fill opacity=1 ][line width=0.08]  [draw opacity=0] (11.61,-5.58) -- (0,0) -- (11.61,5.58) -- cycle    ;
%Rounded Rect [id:dp26569584116703093]
\draw   (20.15,265.17) .. controls (20.15,262.37) and (22.42,260.11) .. (25.21,260.11) -- (47.32,260.11) .. controls (50.11,260.11) and (52.37,262.37) .. (52.37,265.17) -- (52.37,280.33) .. controls (52.37,283.12) and (50.11,285.39) .. (47.32,285.39) -- (25.21,285.39) .. controls (22.42,285.39) and (20.15,283.12) .. (20.15,280.33) -- cycle ;
%Rounded Rect [id:dp5910830009507467]
\draw   (67.67,265.17) .. controls (67.67,262.37) and (69.94,260.11) .. (72.73,260.11) -- (94.84,260.11) .. controls (97.63,260.11) and (99.89,262.37) .. (99.89,265.17) -- (99.89,280.33) .. controls (99.89,283.12) and (97.63,285.39) .. (94.84,285.39) -- (72.73,285.39) .. controls (69.94,285.39) and (67.67,283.12) .. (67.67,280.33) -- cycle ;
%Straight Lines [id:da58248592664115]
\draw  [dash pattern={on 0.84pt off 2.51pt}]  (51.97,244.06) -- (34.21,258.07) ;
\draw [shift={(32.64,259.31)}, rotate = 321.74] [fill={rgb, 255:red, 0; green, 0; blue, 0 }  ][line width=0.08]  [draw opacity=0] (12,-3) -- (0,0) -- (12,3) -- cycle    ;
%Straight Lines [id:da3743490707917462]
\draw  [dash pattern={on 0.84pt off 2.51pt}]  (65.53,242.95) -- (81.03,258.39) ;
\draw [shift={(82.45,259.8)}, rotate = 224.89] [fill={rgb, 255:red, 0; green, 0; blue, 0 }  ][line width=0.08]  [draw opacity=0] (12,-3) -- (0,0) -- (12,3) -- cycle    ;
%Curve Lines [id:da6779048206808959]
\draw [color={rgb, 255:red, 126; green, 211; blue, 33 }  ,draw opacity=1 ][line width=1.5]  [dash pattern={on 5.63pt off 4.5pt}]  (33.2,286.41) .. controls (7.16,319.32) and (62.51,315.24) .. (43.97,289.68) ;
\draw [shift={(41.66,286.81)}, rotate = 408.36] [fill={rgb, 255:red, 126; green, 211; blue, 33 }  ,fill opacity=1 ][line width=0.08]  [draw opacity=0] (11.61,-5.58) -- (0,0) -- (11.61,5.58) -- cycle    ;
%Curve Lines [id:da6817465603054699]
\draw [color={rgb, 255:red, 126; green, 211; blue, 33 }  ,draw opacity=1 ][line width=1.5]  [dash pattern={on 5.63pt off 4.5pt}]  (79.92,286.24) .. controls (53.88,319.14) and (109.22,315.07) .. (90.69,289.5) ;
\draw [shift={(88.37,286.64)}, rotate = 408.36] [fill={rgb, 255:red, 126; green, 211; blue, 33 }  ,fill opacity=1 ][line width=0.08]  [draw opacity=0] (11.61,-5.58) -- (0,0) -- (11.61,5.58) -- cycle    ;
%Curve Lines [id:da39978819515421726]
\draw [color={rgb, 255:red, 126; green, 211; blue, 33 }  ,draw opacity=1 ][line width=1.5]  [dash pattern={on 5.63pt off 4.5pt}]  (62.84,181.32) .. controls (60.37,183.38) and (21.3,184.98) .. (20.9,195.82) ;
%Curve Lines [id:da036977447679392816]
\draw [color={rgb, 255:red, 126; green, 211; blue, 33 }  ,draw opacity=1 ][line width=1.5]  [dash pattern={on 5.63pt off 4.5pt}]  (35.05,225.85) .. controls (32.58,227.91) and (16.07,227.91) .. (15.67,238.74) ;
%Rounded Rect [id:dp18652134945636833]
\draw  [fill={rgb, 255:red, 119; green, 119; blue, 119 }  ,fill opacity=1 ] (96.67,8.16) .. controls (96.67,4.61) and (99.54,1.74) .. (103.09,1.74) -- (172.41,1.74) .. controls (175.95,1.74) and (178.83,4.61) .. (178.83,8.16) -- (178.83,27.42) .. controls (178.83,30.96) and (175.95,33.84) .. (172.41,33.84) -- (103.09,33.84) .. controls (99.54,33.84) and (96.67,30.96) .. (96.67,27.42) -- cycle ;
%Rounded Rect [id:dp4639599267646033]
\draw  [dash pattern={on 4.5pt off 4.5pt}] (10.5,40.81) .. controls (10.5,38.01) and (12.76,35.75) .. (15.56,35.75) -- (57.4,35.75) .. controls (60.19,35.75) and (62.45,38.01) .. (62.45,40.81) -- (62.45,55.97) .. controls (62.45,58.76) and (60.19,61.03) .. (57.4,61.03) -- (15.56,61.03) .. controls (12.76,61.03) and (10.5,58.76) .. (10.5,55.97) -- cycle ;
%Rounded Rect [id:dp48845970858003995]
\draw  [dash pattern={on 4.5pt off 4.5pt}] (209.23,39.2) .. controls (209.23,36.41) and (211.49,34.15) .. (214.28,34.15) -- (256.12,34.15) .. controls (258.92,34.15) and (261.18,36.41) .. (261.18,39.2) -- (261.18,54.37) .. controls (261.18,57.16) and (258.92,59.42) .. (256.12,59.42) -- (214.28,59.42) .. controls (211.49,59.42) and (209.23,57.16) .. (209.23,54.37) -- cycle ;
%Straight Lines [id:da9744934035962726]
\draw  [dash pattern={on 0.84pt off 2.51pt}]  (93.85,18.59) -- (64.09,39.65) ;
\draw [shift={(62.45,40.81)}, rotate = 324.72] [fill={rgb, 255:red, 0; green, 0; blue, 0 }  ][line width=0.08]  [draw opacity=0] (12,-3) -- (0,0) -- (12,3) -- cycle    ;
%Straight Lines [id:da5212293254106946]
\draw  [dash pattern={on 0.84pt off 2.51pt}]  (207.73,39.88) -- (181.24,16.58) ;
\draw [shift={(209.23,41.2)}, rotate = 221.34] [fill={rgb, 255:red, 0; green, 0; blue, 0 }  ][line width=0.08]  [draw opacity=0] (12,-3) -- (0,0) -- (12,3) -- cycle    ;
%Curve Lines [id:da8790196815807736]
\draw [color={rgb, 255:red, 126; green, 211; blue, 33 }  ,draw opacity=1 ][line width=1.5]  [dash pattern={on 5.63pt off 4.5pt}]  (30.39,62.23) .. controls (4.36,95.14) and (59.7,91.06) .. (41.17,65.5) ;
\draw [shift={(38.85,62.63)}, rotate = 408.36] [fill={rgb, 255:red, 126; green, 211; blue, 33 }  ,fill opacity=1 ][line width=0.08]  [draw opacity=0] (11.61,-5.58) -- (0,0) -- (11.61,5.58) -- cycle    ;
%Curve Lines [id:da1696045511646027]
\draw [color={rgb, 255:red, 126; green, 211; blue, 33 }  ,draw opacity=1 ][line width=1.5]  [dash pattern={on 5.63pt off 4.5pt}]  (229.12,61.43) .. controls (203.08,94.34) and (258.43,90.26) .. (239.9,64.69) ;
\draw [shift={(237.58,61.83)}, rotate = 408.36] [fill={rgb, 255:red, 126; green, 211; blue, 33 }  ,fill opacity=1 ][line width=0.08]  [draw opacity=0] (11.61,-5.58) -- (0,0) -- (11.61,5.58) -- cycle    ;
%Rounded Rect [id:dp6945718021564854]
\draw  [fill={rgb, 255:red, 255; green, 255; blue, 255 }  ,fill opacity=1 ] (157.11,220.43) .. controls (157.11,216.89) and (159.98,214.02) .. (163.53,214.02) -- (232.85,214.02) .. controls (236.39,214.02) and (239.27,216.89) .. (239.27,220.43) -- (239.27,239.69) .. controls (239.27,243.24) and (236.39,246.11) .. (232.85,246.11) -- (163.53,246.11) .. controls (159.98,246.11) and (157.11,243.24) .. (157.11,239.69) -- cycle ;
%Rounded Rect [id:dp09638150891615105]
\draw   (159.52,272.03) .. controls (159.52,269.24) and (161.79,266.97) .. (164.58,266.97) -- (186.69,266.97) .. controls (189.48,266.97) and (191.74,269.24) .. (191.74,272.03) -- (191.74,287.19) .. controls (191.74,289.99) and (189.48,292.25) .. (186.69,292.25) -- (164.58,292.25) .. controls (161.79,292.25) and (159.52,289.99) .. (159.52,287.19) -- cycle ;
%Rounded Rect [id:dp4364849194998165]
\draw   (207.05,272.03) .. controls (207.05,269.24) and (209.31,266.97) .. (212.1,266.97) -- (234.21,266.97) .. controls (237,266.97) and (239.27,269.24) .. (239.27,272.03) -- (239.27,287.19) .. controls (239.27,289.99) and (237,292.25) .. (234.21,292.25) -- (212.1,292.25) .. controls (209.31,292.25) and (207.05,289.99) .. (207.05,287.19) -- cycle ;
%Straight Lines [id:da5227770003346777]
\draw  [dash pattern={on 0.84pt off 2.51pt}]  (192.15,247.72) -- (173.48,264.82) ;
\draw [shift={(172.01,266.17)}, rotate = 317.49] [fill={rgb, 255:red, 0; green, 0; blue, 0 }  ][line width=0.08]  [draw opacity=0] (12,-3) -- (0,0) -- (12,3) -- cycle    ;
%Straight Lines [id:da2203379712109298]
\draw  [dash pattern={on 0.84pt off 2.51pt}]  (204.9,249.81) -- (220.4,265.25) ;
\draw [shift={(221.82,266.66)}, rotate = 224.89] [fill={rgb, 255:red, 0; green, 0; blue, 0 }  ][line width=0.08]  [draw opacity=0] (12,-3) -- (0,0) -- (12,3) -- cycle    ;
%Rounded Rect [id:dp10352261889629344]
\draw  [fill={rgb, 255:red, 224; green, 219; blue, 219 }  ,fill opacity=1 ] (156.84,172.37) .. controls (156.84,169.58) and (159.1,167.31) .. (161.89,167.31) -- (229.4,167.31) .. controls (232.19,167.31) and (234.46,169.58) .. (234.46,172.37) -- (234.46,187.53) .. controls (234.46,190.32) and (232.19,192.59) .. (229.4,192.59) -- (161.89,192.59) .. controls (159.1,192.59) and (156.84,190.32) .. (156.84,187.53) -- cycle ;
%Straight Lines [id:da016930349705861603]
\draw  [dash pattern={on 0.84pt off 2.51pt}]  (195.64,194.33) -- (195.76,210.41) ;
\draw [shift={(195.77,212.41)}, rotate = 269.57] [fill={rgb, 255:red, 0; green, 0; blue, 0 }  ][line width=0.08]  [draw opacity=0] (12,-3) -- (0,0) -- (12,3) -- cycle    ;
%Curve Lines [id:da3109220860170444]
\draw [color={rgb, 255:red, 126; green, 211; blue, 33 }  ,draw opacity=1 ][line width=1.5]  [dash pattern={on 5.63pt off 4.5pt}]  (234.81,179.88) .. controls (260.8,200.6) and (266.98,259.03) .. (243.6,278.4) ;
\draw [shift={(240.52,280.61)}, rotate = 328.28] [fill={rgb, 255:red, 126; green, 211; blue, 33 }  ,fill opacity=1 ][line width=0.08]  [draw opacity=0] (11.61,-5.58) -- (0,0) -- (11.61,5.58) -- cycle    ;
%Curve Lines [id:da5939567801888644]
\draw [color={rgb, 255:red, 126; green, 211; blue, 33 }  ,draw opacity=1 ][line width=1.5]  [dash pattern={on 5.63pt off 4.5pt}]  (240.52,228.61) .. controls (253.83,234.49) and (263.08,242.04) .. (254.52,262.61) ;
%Curve Lines [id:da12463656400835332]
\draw [color={rgb, 255:red, 126; green, 211; blue, 33 }  ,draw opacity=1 ][line width=1.5]  [dash pattern={on 5.63pt off 4.5pt}]  (95.87,73.95) .. controls (83.78,80.37) and (48.75,126.11) .. (37.07,153.39) ;
%Rounded Rect [id:dp8964386224394802]
\draw  [fill={rgb, 255:red, 224; green, 219; blue, 219 }  ,fill opacity=1 ] (378.41,67.16) .. controls (378.41,63.32) and (381.53,60.2) .. (385.38,60.2) -- (453.52,60.2) .. controls (457.37,60.2) and (460.49,63.32) .. (460.49,67.16) -- (460.49,88.07) .. controls (460.49,91.92) and (457.37,95.04) .. (453.52,95.04) -- (385.38,95.04) .. controls (381.53,95.04) and (378.41,91.92) .. (378.41,88.07) -- cycle ;
%Rounded Rect [id:dp5069498651180213]
\draw  [fill={rgb, 255:red, 224; green, 219; blue, 219 }  ,fill opacity=1 ] (377.17,125.64) .. controls (377.17,121.79) and (380.29,118.67) .. (384.14,118.67) -- (452.27,118.67) .. controls (456.12,118.67) and (459.24,121.79) .. (459.24,125.64) -- (459.24,146.54) .. controls (459.24,150.39) and (456.12,153.51) .. (452.27,153.51) -- (384.14,153.51) .. controls (380.29,153.51) and (377.17,150.39) .. (377.17,146.54) -- cycle ;
%Rounded Rect [id:dp8860038624305825]
\draw  [fill={rgb, 255:red, 224; green, 219; blue, 219 }  ,fill opacity=1 ] (342.97,187.74) .. controls (342.97,184.71) and (345.43,182.25) .. (348.46,182.25) -- (389.38,182.25) .. controls (392.41,182.25) and (394.87,184.71) .. (394.87,187.74) -- (394.87,204.21) .. controls (394.87,207.24) and (392.41,209.69) .. (389.38,209.69) -- (348.46,209.69) .. controls (345.43,209.69) and (342.97,207.24) .. (342.97,204.21) -- cycle ;
%Rounded Rect [id:dp804459423293255]
\draw   (315.62,234.78) .. controls (315.62,231.75) and (318.07,229.29) .. (321.1,229.29) -- (358.81,229.29) .. controls (361.84,229.29) and (364.3,231.75) .. (364.3,234.78) -- (364.3,251.24) .. controls (364.3,254.27) and (361.84,256.73) .. (358.81,256.73) -- (321.1,256.73) .. controls (318.07,256.73) and (315.62,254.27) .. (315.62,251.24) -- cycle ;
%Rounded Rect [id:dp29501739153223683]
\draw   (377.97,234.78) .. controls (377.97,231.75) and (380.43,229.29) .. (383.46,229.29) -- (404.67,229.29) .. controls (407.7,229.29) and (410.16,231.75) .. (410.16,234.78) -- (410.16,251.24) .. controls (410.16,254.27) and (407.7,256.73) .. (404.67,256.73) -- (383.46,256.73) .. controls (380.43,256.73) and (377.97,254.27) .. (377.97,251.24) -- cycle ;
%Straight Lines [id:da18622648311378742]
\draw  [dash pattern={on 0.84pt off 2.51pt}]  (398.49,153.51) -- (370.97,180) ;
\draw [shift={(369.53,181.38)}, rotate = 316.1] [fill={rgb, 255:red, 0; green, 0; blue, 0 }  ][line width=0.08]  [draw opacity=0] (12,-3) -- (0,0) -- (12,3) -- cycle    ;
%Straight Lines [id:da13499732735240277]
\draw  [dash pattern={on 0.84pt off 2.51pt}]  (362.28,211.87) -- (344.49,227.12) ;
\draw [shift={(342.97,228.42)}, rotate = 319.4] [fill={rgb, 255:red, 0; green, 0; blue, 0 }  ][line width=0.08]  [draw opacity=0] (12,-3) -- (0,0) -- (12,3) -- cycle    ;
%Straight Lines [id:da2251932570253401]
\draw  [dash pattern={on 0.84pt off 2.51pt}]  (375.83,210.66) -- (391.37,227.49) ;
\draw [shift={(392.73,228.95)}, rotate = 227.27] [fill={rgb, 255:red, 0; green, 0; blue, 0 }  ][line width=0.08]  [draw opacity=0] (12,-3) -- (0,0) -- (12,3) -- cycle    ;
%Straight Lines [id:da14913119830371502]
\draw  [dash pattern={on 0.84pt off 2.51pt}]  (431.48,153.51) -- (448.81,171.49) ;
\draw [shift={(450.2,172.93)}, rotate = 226.06] [fill={rgb, 255:red, 0; green, 0; blue, 0 }  ][line width=0.08]  [draw opacity=0] (12,-3) -- (0,0) -- (12,3) -- cycle    ;
%Straight Lines [id:da6513337159436552]
\draw  [dash pattern={on 0.84pt off 2.51pt}]  (419.71,97.76) -- (419.98,114.32) ;
\draw [shift={(420.01,116.32)}, rotate = 269.06] [fill={rgb, 255:red, 0; green, 0; blue, 0 }  ][line width=0.08]  [draw opacity=0] (12,-3) -- (0,0) -- (12,3) -- cycle    ;
%Straight Lines [id:da9223575107615766]
\draw  [dash pattern={on 0.84pt off 2.51pt}]  (419.17,40.92) -- (419.9,54.57) ;
\draw [shift={(420.01,56.57)}, rotate = 266.92] [fill={rgb, 255:red, 0; green, 0; blue, 0 }  ][line width=0.08]  [draw opacity=0] (12,-3) -- (0,0) -- (12,3) -- cycle    ;
%Straight Lines [id:da18307285644805127]
\draw [color={rgb, 255:red, 245; green, 166; blue, 35 }  ,draw opacity=1 ][line width=1.5]    (364.3,243.16) -- (366.35,243.16) -- (374.35,243.16) ;
\draw [shift={(377.97,243.16)}, rotate = 180] [fill={rgb, 255:red, 245; green, 166; blue, 35 }  ,fill opacity=1 ][line width=0.08]  [draw opacity=0] (11.61,-5.58) -- (0,0) -- (11.61,5.58) -- cycle    ;
%Curve Lines [id:da8167539622752814]
\draw [color={rgb, 255:red, 245; green, 166; blue, 35 }  ,draw opacity=1 ][line width=1.5]    (334.12,227.55) .. controls (331.72,226.79) and (330.77,225.2) .. (331.26,222.77) .. controls (331.78,220.39) and (330.88,218.88) .. (328.57,218.24) .. controls (326.28,217.64) and (325.45,216.22) .. (326.06,213.97) .. controls (326.71,211.76) and (325.93,210.42) .. (323.72,209.93) .. controls (321.53,209.47) and (320.64,207.89) .. (321.03,205.2) .. controls (321.82,203.2) and (321.01,201.74) .. (318.6,200.82) .. controls (316.53,200.51) and (315.81,199.16) .. (316.42,196.77) .. controls (317.11,194.48) and (316.34,192.99) .. (314.13,192.3) .. controls (311.95,191.64) and (311.3,190.29) .. (312.18,188.25) .. controls (312.97,185.93) and (312.27,184.33) .. (310.09,183.45) .. controls (307.97,182.57) and (307.41,181.01) .. (308.42,178.77) .. controls (309.62,176.84) and (309.25,175.24) .. (307.3,173.95) .. controls (305.49,172.22) and (305.48,170.56) .. (307.25,168.99) .. controls (309.22,167.98) and (309.77,166.48) .. (308.9,164.47) .. controls (308.77,161.89) and (309.87,160.61) .. (312.2,160.64) .. controls (314.51,160.87) and (315.87,159.79) .. (316.28,157.4) .. controls (316.7,155.06) and (318,154.15) .. (320.17,154.67) .. controls (322.7,154.95) and (324.18,153.92) .. (324.61,151.58) .. controls (324.78,149.4) and (326.1,148.43) .. (328.59,148.66) .. controls (330.86,149.01) and (332.12,148.01) .. (332.36,145.67) .. controls (332.58,143.3) and (333.9,142.16) .. (336.33,142.25) .. controls (338.5,142.52) and (339.7,141.39) .. (339.95,138.84) .. controls (339.82,136.61) and (341.06,135.33) .. (343.68,134.99) .. controls (346.01,134.9) and (347.11,133.68) .. (346.96,131.31) -- (348.07,130) ;
%Curve Lines [id:da8025253559104815]
\draw [color={rgb, 255:red, 245; green, 166; blue, 35 }  ,draw opacity=1 ][line width=1.5]    (488.69,180.6) .. controls (486.96,178.77) and (486.9,176.91) .. (488.51,175.03) .. controls (490.13,173.2) and (490.09,171.44) .. (488.39,169.75) .. controls (486.7,168.08) and (486.67,166.42) .. (488.31,164.75) .. controls (489.97,163.15) and (489.96,161.58) .. (488.29,160.04) .. controls (486.63,158.52) and (486.64,157.03) .. (488.31,155.58) .. controls (490.01,153.5) and (490.04,151.77) .. (488.41,150.38) .. controls (486.8,148.36) and (486.86,146.75) .. (488.57,145.55) .. controls (490.32,143.82) and (490.42,142.05) .. (488.85,140.23) .. controls (487.31,138.45) and (487.44,136.84) .. (489.23,135.4) .. controls (491.09,133.61) and (491.27,131.92) .. (489.78,130.34) .. controls (488.38,128.33) and (488.64,126.63) .. (490.55,125.24) .. controls (492.48,124.04) and (492.78,122.56) .. (491.45,120.79) .. controls (490.26,118.7) and (490.75,116.97) .. (492.9,115.61) .. controls (494.94,114.84) and (495.49,113.4) .. (494.55,111.31) .. controls (493.84,108.96) and (494.62,107.45) .. (496.91,106.8) .. controls (499.25,106.25) and (500.18,104.89) .. (499.69,102.71) .. controls (499.43,100.32) and (500.49,98.95) .. (502.88,98.62) .. controls (505.17,98.39) and (506.13,97.11) .. (505.78,94.78) .. controls (505.37,92.43) and (506.28,91.04) .. (508.52,90.61) .. controls (510.83,89.92) and (511.67,88.37) .. (511.04,85.94) .. controls (510.21,83.77) and (510.9,82.21) .. (513.11,81.25) -- (514.74,76.87) -- (516.89,69.76) ;
\draw [shift={(517.86,65.97)}, rotate = 463.61] [fill={rgb, 255:red, 245; green, 166; blue, 35 }  ,fill opacity=1 ][line width=0.08]  [draw opacity=0] (11.61,-5.58) -- (0,0) -- (11.61,5.58) -- cycle    ;
%Straight Lines [id:da3899459507328076]
\draw [color={rgb, 255:red, 245; green, 166; blue, 35 }  ,draw opacity=1 ][line width=1.5]    (460.89,77.14) .. controls (461.52,74.87) and (462.98,74.05) .. (465.25,74.68) .. controls (467.52,75.31) and (468.97,74.49) .. (469.6,72.22) .. controls (470.23,69.95) and (471.68,69.13) .. (473.95,69.76) .. controls (476.22,70.39) and (477.68,69.57) .. (478.31,67.3) .. controls (478.94,65.03) and (480.39,64.21) .. (482.66,64.84) -- (485.12,63.44) -- (492.08,59.51) ;
\draw [shift={(495.56,57.54)}, rotate = 510.52] [fill={rgb, 255:red, 245; green, 166; blue, 35 }  ,fill opacity=1 ][line width=0.08]  [draw opacity=0] (11.61,-5.58) -- (0,0) -- (11.61,5.58) -- cycle    ;
%Curve Lines [id:da21452681283679098]
\draw [color={rgb, 255:red, 245; green, 166; blue, 35 }  ,draw opacity=1 ][line width=1.5]    (520.06,242.07) .. controls (519.03,239.92) and (519.58,238.3) .. (521.7,237.23) .. controls (523.81,236.18) and (524.33,234.62) .. (523.26,232.53) .. controls (522.68,228.96) and (523.42,226.69) .. (525.49,225.71) .. controls (527.56,224.75) and (528.03,223.29) .. (526.9,221.32) .. controls (525.77,219.39) and (526.22,217.97) .. (528.25,217.06) .. controls (530.28,216.16) and (530.71,214.78) .. (529.54,212.92) .. controls (528.36,211.08) and (528.96,209.08) .. (531.35,206.92) .. controls (533.34,206.1) and (533.72,204.82) .. (532.49,203.08) .. controls (531.26,201.36) and (531.79,199.51) .. (534.09,197.52) .. controls (536.04,196.77) and (536.37,195.58) .. (535.09,193.95) .. controls (534.11,191.21) and (534.57,189.49) .. (536.48,188.8) .. controls (538.67,187.02) and (539.09,185.39) .. (537.75,183.9) .. controls (536.65,181.41) and (537.03,179.85) .. (538.9,179.23) .. controls (541,177.61) and (541.35,176.13) .. (539.94,174.78) .. controls (538.73,172.53) and (539.14,170.67) .. (541.16,169.19) .. controls (543.15,167.74) and (543.49,166) .. (542.18,163.97) .. controls (540.69,162.84) and (540.97,161.22) .. (543.02,159.11) .. controls (544.9,157.82) and (545.12,156.32) .. (543.68,154.6) .. controls (542.21,152.98) and (542.4,151.25) .. (544.27,149.4) .. controls (546.04,148.24) and (546.16,146.67) .. (544.62,144.69) .. controls (543.02,142.92) and (543.06,141.23) .. (544.73,139.62) .. controls (546.36,138.01) and (546.29,136.51) .. (544.53,135.14) .. controls (542.69,133.58) and (542.46,131.87) .. (543.83,130) .. controls (545.1,128.07) and (544.66,126.46) .. (542.51,125.18) .. controls (540.38,124.36) and (539.7,122.91) .. (540.49,120.84) .. controls (540.87,118.39) and (539.86,117) .. (537.46,116.67) .. controls (535.16,116.68) and (533.99,115.54) .. (533.95,113.25) .. controls (533.76,110.92) and (532.5,109.88) .. (530.17,110.14) .. controls (527.82,110.37) and (526.54,109.29) .. (526.33,106.9) .. controls (526.26,104.53) and (525.12,103.39) .. (522.92,103.46) .. controls (520.41,102.99) and (519.36,101.58) .. (519.78,99.23) .. controls (520.4,96.94) and (519.65,95.51) .. (517.53,94.94) .. controls (515.37,94.05) and (514.8,92.52) .. (515.83,90.35) .. controls (516.97,88.2) and (516.52,86.38) .. (514.49,84.88) -- (513.86,80.97) ;
%Rounded Rect [id:dp6245224447542773]
\draw   (300.73,284.43) .. controls (300.73,281.4) and (303.19,278.94) .. (306.22,278.94) -- (327.43,278.94) .. controls (330.46,278.94) and (332.92,281.4) .. (332.92,284.43) -- (332.92,300.89) .. controls (332.92,303.92) and (330.46,306.38) .. (327.43,306.38) -- (306.22,306.38) .. controls (303.19,306.38) and (300.73,303.92) .. (300.73,300.89) -- cycle ;
%Rounded Rect [id:dp42951758335930534]
\draw   (348.2,284.43) .. controls (348.2,281.4) and (350.66,278.94) .. (353.69,278.94) -- (374.9,278.94) .. controls (377.93,278.94) and (380.39,281.4) .. (380.39,284.43) -- (380.39,300.89) .. controls (380.39,303.92) and (377.93,306.38) .. (374.9,306.38) -- (353.69,306.38) .. controls (350.66,306.38) and (348.2,303.92) .. (348.2,300.89) -- cycle ;
%Straight Lines [id:da8501241208461399]
\draw  [dash pattern={on 0.84pt off 2.51pt}]  (332.51,261.52) -- (314.72,276.77) ;
\draw [shift={(313.2,278.07)}, rotate = 319.4] [fill={rgb, 255:red, 0; green, 0; blue, 0 }  ][line width=0.08]  [draw opacity=0] (12,-3) -- (0,0) -- (12,3) -- cycle    ;
%Straight Lines [id:da46373934261066674]
\draw  [dash pattern={on 0.84pt off 2.51pt}]  (346.06,260.31) -- (361.6,277.14) ;
\draw [shift={(362.96,278.6)}, rotate = 227.27] [fill={rgb, 255:red, 0; green, 0; blue, 0 }  ][line width=0.08]  [draw opacity=0] (12,-3) -- (0,0) -- (12,3) -- cycle    ;
%Straight Lines [id:da2913829767130839]
\draw [color={rgb, 255:red, 245; green, 166; blue, 35 }  ,draw opacity=1 ][line width=1.5]    (333.72,291.57) -- (336.07,291.58) -- (344.07,291.62) ;
\draw [shift={(348.07,291.64)}, rotate = 180.26] [fill={rgb, 255:red, 245; green, 166; blue, 35 }  ,fill opacity=1 ][line width=0.08]  [draw opacity=0] (11.61,-5.58) -- (0,0) -- (11.61,5.58) -- cycle    ;
%Straight Lines [id:da1365205158433368]
\draw [color={rgb, 255:red, 245; green, 166; blue, 35 }  ,draw opacity=1 ][line width=1.5]    (372.1,278.94) .. controls (369.75,278.79) and (368.65,277.54) .. (368.8,275.19) .. controls (368.95,272.84) and (367.84,271.59) .. (365.49,271.44) -- (363.61,269.3) -- (358.32,263.3) ;
\draw [shift={(355.67,260.3)}, rotate = 408.62] [fill={rgb, 255:red, 245; green, 166; blue, 35 }  ,fill opacity=1 ][line width=0.08]  [draw opacity=0] (11.61,-5.58) -- (0,0) -- (11.61,5.58) -- cycle    ;
%Curve Lines [id:da22111636927235767]
\draw [color={rgb, 255:red, 245; green, 166; blue, 35 }  ,draw opacity=1 ][line width=1.5]    (362.69,180.77) .. controls (360.03,180.02) and (358.95,178.58) .. (359.46,176.45) .. controls (359.67,173.83) and (358.71,172.43) .. (356.6,172.26) .. controls (354.15,171.5) and (353.32,170.15) .. (354.1,168.21) .. controls (354.67,165.84) and (353.81,164.28) .. (351.54,163.53) .. controls (349.3,162.72) and (348.61,161.22) .. (349.46,159.04) .. controls (350.45,157.01) and (349.91,155.58) .. (347.84,154.75) .. controls (345.67,153.34) and (345.23,151.75) .. (346.51,149.97) .. controls (347.82,147.96) and (347.54,146.25) .. (345.67,144.83) .. controls (343.9,143.24) and (343.86,141.63) .. (345.54,140.01) .. controls (347.36,138.33) and (347.57,136.65) .. (346.18,134.96) .. controls (345.05,132.71) and (345.52,131.16) .. (347.59,130.31) .. controls (349.8,129.46) and (350.6,127.89) .. (349.98,125.6) .. controls (349.65,123.15) and (350.59,121.87) .. (352.81,121.76) .. controls (355.21,121.62) and (356.35,120.46) .. (356.23,118.28) .. controls (356.62,115.79) and (357.93,114.75) .. (360.16,115.17) .. controls (362.63,115.53) and (364.09,114.61) .. (364.55,112.4) .. controls (365.18,110.17) and (366.77,109.37) .. (369.33,109.99) .. controls (371.46,110.88) and (372.97,110.26) .. (373.87,108.12) .. controls (374.9,105.98) and (376.49,105.44) .. (378.63,106.51) .. controls (380.7,107.64) and (382.35,107.19) .. (383.56,105.15) .. controls (384.45,103.23) and (385.93,102.9) .. (387.99,104.16) .. controls (389.99,105.49) and (391.71,105.19) .. (393.14,103.27) .. controls (394.67,101.38) and (396.4,101.15) .. (398.35,102.6) .. controls (399.8,104.13) and (401.33,104) .. (402.92,102.21) .. controls (404.58,100.45) and (406.31,100.37) .. (408.11,101.98) .. controls (409.83,103.62) and (411.54,103.61) .. (413.23,101.96) .. controls (414.93,100.33) and (416.6,100.35) .. (418.25,102.02) .. controls (419.93,103.67) and (421.57,103.64) .. (423.18,101.93) .. controls (424.72,100.2) and (426.33,100.13) .. (428.01,101.71) .. controls (429.72,103.27) and (431.3,103.16) .. (432.75,101.37) .. controls (434.52,99.52) and (436.27,99.34) .. (437.98,100.83) .. controls (439.71,102.3) and (441.42,102.08) .. (443.09,100.17) .. controls (444.32,98.3) and (445.81,98.06) .. (447.54,99.47) .. controls (449.67,100.79) and (451.48,100.46) .. (452.98,98.48) .. controls (454.07,96.56) and (455.66,96.23) .. (457.77,97.48) .. controls (459.54,98.79) and (461.09,98.43) .. (462.44,96.4) .. controls (463.74,94.37) and (465.43,93.94) .. (467.52,95.12) .. controls (469.29,96.37) and (470.78,95.97) .. (471.99,93.91) .. controls (473.15,91.84) and (474.76,91.38) .. (476.83,92.51) .. controls (478.9,93.63) and (480.47,93.15) .. (481.56,91.08) .. controls (482.91,88.91) and (484.59,88.38) .. (486.62,89.49) .. controls (488.63,90.6) and (490.13,90.12) .. (491.1,88.05) .. controls (492.35,85.89) and (493.95,85.37) .. (495.91,86.48) .. controls (498.13,87.51) and (499.69,87.01) .. (500.58,84.97) .. controls (501.75,82.85) and (503.4,82.33) .. (505.54,83.41) .. controls (507.63,84.51) and (509.24,84.03) .. (510.35,81.96) -- (513.86,80.97) ;
%Curve Lines [id:da27837816309222707]
\draw [color={rgb, 255:red, 245; green, 166; blue, 35 }  ,draw opacity=1 ][line width=1.5]    (399.32,259.64) .. controls (401.43,261) and (401.96,262.8) .. (400.92,265.04) .. controls (399.97,267.33) and (400.49,268.74) .. (402.48,269.26) .. controls (404.69,270.18) and (405.41,271.76) .. (404.62,274) .. controls (403.95,276.3) and (404.75,277.77) .. (407.04,278.41) .. controls (409.3,278.88) and (410.19,280.24) .. (409.71,282.49) .. controls (409.35,284.79) and (410.32,286.04) .. (412.63,286.25) .. controls (415.23,286.68) and (416.46,288.02) .. (416.32,290.27) .. controls (416.31,292.56) and (417.44,293.6) .. (419.71,293.39) .. controls (422.31,293.36) and (423.71,294.47) .. (423.91,296.7) .. controls (424.23,298.95) and (425.71,299.93) .. (428.35,299.64) .. controls (430.49,298.97) and (431.82,299.72) .. (432.33,301.9) .. controls (432.96,304.09) and (434.56,304.87) .. (437.15,304.24) .. controls (439.2,303.3) and (440.62,303.89) .. (441.42,306) .. controls (442.31,308.11) and (444.02,308.7) .. (446.53,307.79) .. controls (448.49,306.65) and (449.99,307.09) .. (451.02,309.12) .. controls (452.13,311.13) and (453.64,311.52) .. (455.57,310.27) .. controls (457.96,309.08) and (459.75,309.45) .. (460.96,311.39) .. controls (462.24,313.31) and (463.8,313.57) .. (465.63,312.18) .. controls (467.42,310.75) and (468.99,310.97) .. (470.32,312.83) .. controls (472.24,314.72) and (474.07,314.92) .. (475.81,313.42) .. controls (477.51,311.89) and (479.07,312.02) .. (480.5,313.79) .. controls (481.97,315.56) and (483.78,315.66) .. (485.95,314.1) .. controls (487.56,312.5) and (489.1,312.56) .. (490.58,314.27) .. controls (492.07,315.97) and (493.85,316) .. (495.9,314.36) .. controls (497.4,312.7) and (498.89,312.7) .. (500.38,314.37) .. controls (502.35,316.02) and (504.06,316.01) .. (505.49,314.32) .. controls (507.36,312.61) and (509.01,312.58) .. (510.45,314.21) .. controls (512.34,315.82) and (513.94,315.78) .. (515.24,314.07) .. controls (516.95,312.34) and (518.7,312.28) .. (520.47,313.89) .. controls (522.2,315.5) and (523.84,315.44) .. (525.41,313.73) .. controls (526.92,312.02) and (528.63,311.92) .. (530.53,313.44) .. controls (532.8,314.58) and (534.29,314.09) .. (535.01,311.97) .. controls (535.48,309.72) and (536.88,308.66) .. (539.22,308.81) .. controls (541.51,308.72) and (542.55,307.36) .. (542.36,304.71) .. controls (541.69,302.51) and (542.46,301) .. (544.69,300.17) .. controls (546.76,299.48) and (547.32,297.94) .. (546.38,295.55) .. controls (545.19,293.68) and (545.6,292.21) .. (547.59,291.16) .. controls (549.57,289.95) and (549.95,288.1) .. (548.74,285.59) .. controls (547.28,284.2) and (547.5,282.78) .. (549.41,281.33) .. controls (551.29,279.75) and (551.5,277.96) .. (550.04,275.95) .. controls (548.55,273.95) and (548.68,272.39) .. (550.43,271.26) .. controls (552.22,269.43) and (552.32,267.81) .. (550.72,266.41) .. controls (549.14,264.36) and (549.2,262.69) .. (550.91,261.4) .. controls (552.62,259.39) and (552.65,257.68) .. (551,256.28) .. controls (549.34,254.19) and (549.34,252.45) .. (551,251.04) .. controls (552.65,249.61) and (552.63,247.83) .. (550.93,245.71) .. controls (549.23,244.32) and (549.19,242.89) .. (550.81,241.4) .. controls (552.4,239.16) and (552.34,237.35) .. (550.61,235.96) .. controls (548.88,234.58) and (548.79,232.75) .. (550.35,230.48) .. controls (551.93,228.92) and (551.85,227.45) .. (550.1,226.08) .. controls (548.35,224.72) and (548.23,222.89) .. (549.75,220.58) .. controls (551.31,219) and (551.21,217.54) .. (549.44,216.2) .. controls (547.67,214.87) and (547.53,213.05) .. (549.02,210.74) .. controls (550.57,209.16) and (550.45,207.72) .. (548.67,206.41) .. controls (546.83,204.4) and (546.68,202.62) .. (548.21,201.06) .. controls (549.75,199.5) and (549.59,197.74) .. (547.74,195.79) .. controls (545.95,194.56) and (545.83,193.18) .. (547.36,191.65) .. controls (548.83,189.46) and (548.68,187.77) .. (546.89,186.58) .. controls (545.04,184.75) and (544.89,183.11) .. (546.42,181.64) .. controls (547.9,179.56) and (547.72,177.66) .. (545.88,175.93) .. controls (544.11,174.85) and (543.97,173.33) .. (545.46,171.36) .. controls (546.96,169.45) and (546.8,167.72) .. (544.99,166.15) .. controls (543.19,164.62) and (543.05,162.99) .. (544.58,161.25) .. controls (546.12,159.58) and (546.01,158.06) .. (544.24,156.69) .. controls (542.45,154.91) and (542.35,153.09) .. (543.92,151.23) -- (543.86,149.97) ;
%Curve Lines [id:da723553659441138]
\draw [color={rgb, 255:red, 245; green, 166; blue, 35 }  ,draw opacity=1 ][line width=1.5]    (305.21,61.73) .. controls (306.24,63.88) and (305.68,65.56) .. (303.53,66.79) .. controls (301.51,67.7) and (301.01,69.37) .. (302.02,71.81) .. controls (303.19,73.8) and (302.75,75.38) .. (300.7,76.53) .. controls (298.65,77.76) and (298.21,79.45) .. (299.38,81.59) .. controls (300.7,83.17) and (300.33,84.66) .. (298.28,86.07) .. controls (296.21,87.55) and (295.85,89.11) .. (297.18,90.76) .. controls (298.38,93.06) and (298.02,94.69) .. (296.1,95.65) .. controls (294.05,97.33) and (293.7,99.02) .. (295.05,100.73) .. controls (296.27,103.14) and (295.93,104.9) .. (294.02,105.99) .. controls (292.11,107.12) and (291.84,108.56) .. (293.21,110.31) .. controls (294.46,112.8) and (294.14,114.65) .. (292.25,115.85) .. controls (290.35,117.08) and (290.1,118.59) .. (291.5,120.38) .. controls (292.91,122.17) and (292.61,124.09) .. (290.62,126.16) .. controls (288.74,127.49) and (288.52,129.05) .. (289.95,130.86) .. controls (291.38,132.66) and (291.17,134.25) .. (289.32,135.63) .. controls (287.46,137.03) and (287.26,138.64) .. (288.73,140.45) .. controls (290.2,142.26) and (290.01,143.88) .. (288.18,145.33) .. controls (286.35,146.8) and (286.19,148.44) .. (287.68,150.25) .. controls (289.19,152.05) and (289.04,153.7) .. (287.23,155.21) .. controls (285.44,156.74) and (285.31,158.41) .. (286.84,160.2) .. controls (288.38,161.99) and (288.27,163.66) .. (286.5,165.22) .. controls (284.74,166.8) and (284.65,168.48) .. (286.22,170.25) .. controls (287.81,172) and (287.73,173.68) .. (286,175.29) .. controls (284.28,176.92) and (284.23,178.61) .. (285.85,180.34) .. controls (287.48,182.05) and (287.46,183.73) .. (285.77,185.38) .. controls (284.1,187.05) and (284.09,188.73) .. (285.76,190.41) .. controls (287.45,192.06) and (287.47,193.73) .. (285.83,195.42) .. controls (284.2,197.13) and (284.25,198.8) .. (285.97,200.41) .. controls (287.7,201.99) and (287.78,203.64) .. (286.2,205.36) .. controls (284.63,207.11) and (284.74,208.75) .. (286.51,210.28) .. controls (288.3,211.77) and (288.43,213.4) .. (286.91,215.15) .. controls (285.4,216.93) and (285.57,218.54) .. (287.4,219.97) .. controls (289.24,221.36) and (289.43,222.95) .. (287.98,224.74) .. controls (286.67,227.33) and (286.95,229.28) .. (288.84,230.59) .. controls (290.74,231.84) and (291.01,233.38) .. (289.65,235.19) .. controls (288.31,237.04) and (288.61,238.55) .. (290.55,239.72) .. controls (292.51,240.83) and (292.94,242.67) .. (291.84,245.24) .. controls (290.61,247.12) and (290.99,248.56) .. (292.99,249.55) .. controls (295,250.48) and (295.54,252.23) .. (294.6,254.79) .. controls (293.49,256.7) and (293.96,258.06) .. (296.01,258.86) .. controls (298.32,260.23) and (298.96,261.87) .. (297.95,263.78) .. controls (296.98,265.73) and (297.69,267.3) .. (300.08,268.49) -- (300.99,270.32) -- (304.43,276.43) ;
\draw [shift={(306.06,278.94)}, rotate = 236] [fill={rgb, 255:red, 245; green, 166; blue, 35 }  ,fill opacity=1 ][line width=0.08]  [draw opacity=0] (11.61,-5.58) -- (0,0) -- (11.61,5.58) -- cycle    ;
%Rounded Rect [id:dp3021340239448299]
\draw  [fill={rgb, 255:red, 119; green, 119; blue, 119 }  ,fill opacity=1 ] (376.66,8.69) .. controls (376.66,4.84) and (379.78,1.72) .. (383.63,1.72) -- (451.76,1.72) .. controls (455.61,1.72) and (458.73,4.84) .. (458.73,8.69) -- (458.73,29.59) .. controls (458.73,33.44) and (455.61,36.56) .. (451.76,36.56) -- (383.63,36.56) .. controls (379.78,36.56) and (376.66,33.44) .. (376.66,29.59) -- cycle ;
%Rounded Rect [id:dp5064479506206062]
\draw  [dash pattern={on 4.5pt off 4.5pt}] (293.13,39.82) .. controls (293.13,36.79) and (295.59,34.33) .. (298.62,34.33) -- (339.54,34.33) .. controls (342.57,34.33) and (345.03,36.79) .. (345.03,39.82) -- (345.03,56.28) .. controls (345.03,59.31) and (342.57,61.77) .. (339.54,61.77) -- (298.62,61.77) .. controls (295.59,61.77) and (293.13,59.31) .. (293.13,56.28) -- cycle ;
%Rounded Rect [id:dp09980644629223867]
\draw  [dash pattern={on 4.5pt off 4.5pt}] (495.56,41.08) .. controls (495.56,38.05) and (498.02,35.59) .. (501.05,35.59) -- (541.98,35.59) .. controls (545.01,35.59) and (547.46,38.05) .. (547.46,41.08) -- (547.46,57.54) .. controls (547.46,60.57) and (545.01,63.03) .. (541.98,63.03) -- (501.05,63.03) .. controls (498.02,63.03) and (495.56,60.57) .. (495.56,57.54) -- cycle ;
%Straight Lines [id:da7456391575971809]
\draw  [dash pattern={on 0.84pt off 2.51pt}]  (373.84,20.01) -- (346.68,38.69) ;
\draw [shift={(345.03,39.82)}, rotate = 325.49] [fill={rgb, 255:red, 0; green, 0; blue, 0 }  ][line width=0.08]  [draw opacity=0] (12,-3) -- (0,0) -- (12,3) -- cycle    ;
%Straight Lines [id:da30269366753637605]
\draw  [dash pattern={on 0.84pt off 2.51pt}]  (493.91,39.96) -- (461.14,17.83) ;
\draw [shift={(495.56,41.08)}, rotate = 214.03] [fill={rgb, 255:red, 0; green, 0; blue, 0 }  ][line width=0.08]  [draw opacity=0] (12,-3) -- (0,0) -- (12,3) -- cycle    ;
%Rounded Rect [id:dp4229439036357451]
\draw   (450.39,284.46) .. controls (450.39,281.43) and (452.85,278.98) .. (455.88,278.98) -- (477.09,278.98) .. controls (480.12,278.98) and (482.58,281.43) .. (482.58,284.46) -- (482.58,300.93) .. controls (482.58,303.96) and (480.12,306.41) .. (477.09,306.41) -- (455.88,306.41) .. controls (452.85,306.41) and (450.39,303.96) .. (450.39,300.93) -- cycle ;
%Rounded Rect [id:dp21074606617817526]
\draw   (497.86,284.46) .. controls (497.86,281.43) and (500.32,278.98) .. (503.35,278.98) -- (524.56,278.98) .. controls (527.59,278.98) and (530.05,281.43) .. (530.05,284.46) -- (530.05,300.93) .. controls (530.05,303.96) and (527.59,306.41) .. (524.56,306.41) -- (503.35,306.41) .. controls (500.32,306.41) and (497.86,303.96) .. (497.86,300.93) -- cycle ;
%Straight Lines [id:da15662271380993031]
\draw  [dash pattern={on 0.84pt off 2.51pt}]  (495.72,260.35) -- (511.26,277.17) ;
\draw [shift={(512.62,278.64)}, rotate = 227.27] [fill={rgb, 255:red, 0; green, 0; blue, 0 }  ][line width=0.08]  [draw opacity=0] (12,-3) -- (0,0) -- (12,3) -- cycle    ;
%Rounded Rect [id:dp3996008923066795]
\draw  [fill={rgb, 255:red, 224; green, 219; blue, 219 }  ,fill opacity=1 ] (447.71,179.42) .. controls (447.71,176.39) and (450.17,173.93) .. (453.2,173.93) -- (519.76,173.93) .. controls (522.79,173.93) and (525.24,176.39) .. (525.24,179.42) -- (525.24,195.88) .. controls (525.24,198.91) and (522.79,201.37) .. (519.76,201.37) -- (453.2,201.37) .. controls (450.17,201.37) and (447.71,198.91) .. (447.71,195.88) -- cycle ;
%Straight Lines [id:da7605048314362032]
\draw  [dash pattern={on 0.84pt off 2.51pt}]  (486.47,203.26) -- (486.59,220.74) ;
\draw [shift={(486.6,222.74)}, rotate = 269.61] [fill={rgb, 255:red, 0; green, 0; blue, 0 }  ][line width=0.08]  [draw opacity=0] (12,-3) -- (0,0) -- (12,3) -- cycle    ;
%Curve Lines [id:da5386744782075881]
\draw [color={rgb, 255:red, 245; green, 166; blue, 35 }  ,draw opacity=1 ][line width=1.5]    (460.05,134.35) .. controls (459.68,131.88) and (460.61,130.42) .. (462.83,129.97) .. controls (465.12,129.24) and (465.89,127.76) .. (465.12,125.54) .. controls (464.21,123.47) and (464.84,121.98) .. (467.01,121.07) .. controls (469.2,119.96) and (469.79,118.37) .. (468.76,116.32) .. controls (467.78,114.13) and (468.38,112.54) .. (470.55,111.55) .. controls (472.73,110.7) and (473.4,109.24) .. (472.56,107.17) .. controls (472.01,104.78) and (472.9,103.35) .. (475.23,102.89) .. controls (477.56,102.64) and (478.63,101.38) .. (478.46,99.11) .. controls (478.59,96.68) and (479.83,95.57) .. (482.17,95.77) .. controls (484.52,96.08) and (485.97,95.03) .. (486.52,92.63) .. controls (486.85,90.46) and (488.22,89.63) .. (490.63,90.14) .. controls (492.81,90.85) and (494.25,90.08) .. (494.96,87.84) -- (498.86,85.97) ;
%Straight Lines [id:da7080244436371879]
\draw    (272.64,0.62) -- (272.5,321.54) ;
%Straight Lines [id:da8233492061543983]
\draw    (574.45,0.62) -- (574.45,321.54) ;
%Rounded Rect [id:dp0685325195398302]
\draw  [fill={rgb, 255:red, 119; green, 119; blue, 119 }  ,fill opacity=1 ] (670.96,9.04) .. controls (670.96,5.02) and (674.23,1.76) .. (678.25,1.76) -- (743.49,1.76) .. controls (747.51,1.76) and (750.77,5.02) .. (750.77,9.04) -- (750.77,30.9) .. controls (750.77,34.92) and (747.51,38.18) .. (743.49,38.18) -- (678.25,38.18) .. controls (674.23,38.18) and (670.96,34.92) .. (670.96,30.9) -- cycle ;
%Rounded Rect [id:dp15595139562242855]
\draw  [fill={rgb, 255:red, 224; green, 219; blue, 219 }  ,fill opacity=1 ] (671.75,70.96) .. controls (671.75,66.94) and (675.01,63.67) .. (679.03,63.67) -- (744.27,63.67) .. controls (748.29,63.67) and (751.55,66.94) .. (751.55,70.96) -- (751.55,92.81) .. controls (751.55,96.83) and (748.29,100.1) .. (744.27,100.1) -- (679.03,100.1) .. controls (675.01,100.1) and (671.75,96.83) .. (671.75,92.81) -- cycle ;
%Rounded Rect [id:dp5961377463738089]
\draw  [fill={rgb, 255:red, 224; green, 219; blue, 219 }  ,fill opacity=1 ] (672.53,131.96) .. controls (672.53,127.94) and (675.79,124.68) .. (679.81,124.68) -- (745.05,124.68) .. controls (749.08,124.68) and (752.34,127.94) .. (752.34,131.96) -- (752.34,153.82) .. controls (752.34,157.84) and (749.08,161.1) .. (745.05,161.1) -- (679.81,161.1) .. controls (675.79,161.1) and (672.53,157.84) .. (672.53,153.82) -- cycle ;
%Rounded Rect [id:dp3105316328306651]
\draw  [fill={rgb, 255:red, 224; green, 219; blue, 219 }  ,fill opacity=1 ] (639.28,196.88) .. controls (639.28,193.72) and (641.84,191.15) .. (645.01,191.15) -- (684.01,191.15) .. controls (687.17,191.15) and (689.74,193.72) .. (689.74,196.88) -- (689.74,214.09) .. controls (689.74,217.26) and (687.17,219.83) .. (684.01,219.83) -- (645.01,219.83) .. controls (641.84,219.83) and (639.28,217.26) .. (639.28,214.09) -- cycle ;
%Rounded Rect [id:dp08411975073113154]
\draw   (612.67,246.05) .. controls (612.67,242.88) and (615.24,240.32) .. (618.41,240.32) -- (654.27,240.32) .. controls (657.44,240.32) and (660.01,242.88) .. (660.01,246.05) -- (660.01,263.26) .. controls (660.01,266.43) and (657.44,269) .. (654.27,269) -- (618.41,269) .. controls (615.24,269) and (612.67,266.43) .. (612.67,263.26) -- cycle ;
%Rounded Rect [id:dp8847982400152133]
\draw   (673.31,246.05) .. controls (673.31,242.88) and (675.88,240.32) .. (679.05,240.32) -- (698.87,240.32) .. controls (702.04,240.32) and (704.61,242.88) .. (704.61,246.05) -- (704.61,263.26) .. controls (704.61,266.43) and (702.04,269) .. (698.87,269) -- (679.05,269) .. controls (675.88,269) and (673.31,266.43) .. (673.31,263.26) -- cycle ;
%Straight Lines [id:da1772278585458008]
\draw  [dash pattern={on 0.84pt off 2.51pt}]  (693.26,161.1) -- (675.68,184.31) ;
\draw [shift={(674.48,185.91)}, rotate = 307.14] [fill={rgb, 255:red, 0; green, 0; blue, 0 }  ][line width=0.08]  [draw opacity=0] (12,-3) -- (0,0) -- (12,3) -- cycle    ;
%Straight Lines [id:da8897186365830865]
\draw  [dash pattern={on 0.84pt off 2.51pt}]  (658.05,222.11) -- (640.75,238.05) ;
\draw [shift={(639.28,239.41)}, rotate = 317.35] [fill={rgb, 255:red, 0; green, 0; blue, 0 }  ][line width=0.08]  [draw opacity=0] (12,-3) -- (0,0) -- (12,3) -- cycle    ;
%Straight Lines [id:da07728227281817646]
\draw  [dash pattern={on 0.84pt off 2.51pt}]  (671.23,220.84) -- (686.36,238.45) ;
\draw [shift={(687.66,239.96)}, rotate = 229.32999999999998] [fill={rgb, 255:red, 0; green, 0; blue, 0 }  ][line width=0.08]  [draw opacity=0] (12,-3) -- (0,0) -- (12,3) -- cycle    ;
%Straight Lines [id:da9801659479367559]
\draw  [dash pattern={on 0.84pt off 2.51pt}]  (712.43,101.46) -- (712.43,120.86) ;
\draw [shift={(712.43,122.86)}, rotate = 270] [fill={rgb, 255:red, 0; green, 0; blue, 0 }  ][line width=0.08]  [draw opacity=0] (12,-3) -- (0,0) -- (12,3) -- cycle    ;
%Straight Lines [id:da8677453790043593]
\draw  [dash pattern={on 0.84pt off 2.51pt}]  (712.3,40.91) -- (712.42,58.03) ;
\draw [shift={(712.43,60.03)}, rotate = 269.61] [fill={rgb, 255:red, 0; green, 0; blue, 0 }  ][line width=0.08]  [draw opacity=0] (12,-3) -- (0,0) -- (12,3) -- cycle    ;
%Rounded Rect [id:dp09576756738899128]
\draw   (742.17,296.62) .. controls (742.17,293.46) and (744.73,290.89) .. (747.9,290.89) -- (767.73,290.89) .. controls (770.89,290.89) and (773.46,293.46) .. (773.46,296.62) -- (773.46,313.83) .. controls (773.46,317) and (770.89,319.57) .. (767.73,319.57) -- (747.9,319.57) .. controls (744.73,319.57) and (742.17,317) .. (742.17,313.83) -- cycle ;
%Rounded Rect [id:dp5104991889369999]
\draw   (788.33,296.62) .. controls (788.33,293.46) and (790.9,290.89) .. (794.06,290.89) -- (813.89,290.89) .. controls (817.06,290.89) and (819.63,293.46) .. (819.63,296.62) -- (819.63,313.83) .. controls (819.63,317) and (817.06,319.57) .. (813.89,319.57) -- (794.06,319.57) .. controls (790.9,319.57) and (788.33,317) .. (788.33,313.83) -- cycle ;
%Straight Lines [id:da7003228707991531]
\draw  [dash pattern={on 0.84pt off 2.51pt}]  (773.85,269.04) -- (755.66,288.52) ;
\draw [shift={(754.29,289.98)}, rotate = 313.05] [fill={rgb, 255:red, 0; green, 0; blue, 0 }  ][line width=0.08]  [draw opacity=0] (12,-3) -- (0,0) -- (12,3) -- cycle    ;
%Straight Lines [id:da9672168635693303]
\draw  [dash pattern={on 0.84pt off 2.51pt}]  (786.25,271.41) -- (801.37,289.02) ;
\draw [shift={(802.68,290.54)}, rotate = 229.32999999999998] [fill={rgb, 255:red, 0; green, 0; blue, 0 }  ][line width=0.08]  [draw opacity=0] (12,-3) -- (0,0) -- (12,3) -- cycle    ;
%Curve Lines [id:da24516223021639916]
\draw [color={rgb, 255:red, 74; green, 144; blue, 226 }  ,draw opacity=1 ][line width=1.5]    (599.5,63.99) .. controls (580.01,119.57) and (576.56,226.7) .. (597.24,290.25) ;
\draw [shift={(598.2,293.13)}, rotate = 251.01] [fill={rgb, 255:red, 74; green, 144; blue, 226 }  ,fill opacity=1 ][line width=0.08]  [draw opacity=0] (11.61,-5.58) -- (0,0) -- (11.61,5.58) -- cycle    ;
%Straight Lines [id:da6786754445793188]
\draw [color={rgb, 255:red, 74; green, 144; blue, 226 }  ,draw opacity=1 ][fill={rgb, 255:red, 0; green, 0; blue, 0 }  ,fill opacity=1 ][line width=1.5]    (659.82,256.42) -- (669.65,256.41) ;
\draw [shift={(673.65,256.4)}, rotate = 539.94] [fill={rgb, 255:red, 74; green, 144; blue, 226 }  ,fill opacity=1 ][line width=0.08]  [draw opacity=0] (11.61,-5.58) -- (0,0) -- (11.61,5.58) -- cycle    ;
%Straight Lines [id:da6089027286659577]
\draw [color={rgb, 255:red, 74; green, 144; blue, 226 }  ,draw opacity=1 ][fill={rgb, 255:red, 0; green, 0; blue, 0 }  ,fill opacity=1 ][line width=1.5]    (798.61,271.75) -- (814.06,290.89) ;
\draw [shift={(796.09,268.64)}, rotate = 51.09] [fill={rgb, 255:red, 74; green, 144; blue, 226 }  ,fill opacity=1 ][line width=0.08]  [draw opacity=0] (11.61,-5.58) -- (0,0) -- (11.61,5.58) -- cycle    ;
%Curve Lines [id:da077506208655449]
\draw [color={rgb, 255:red, 74; green, 144; blue, 226 }  ,draw opacity=1 ][line width=1.5]    (660.01,246.05) .. controls (673.31,234.85) and (698.83,209.55) .. (718.83,185.55) .. controls (738.63,161.79) and (764.88,178.03) .. (791.97,71.45) ;
\draw [shift={(792.79,68.18)}, rotate = 463.96] [fill={rgb, 255:red, 74; green, 144; blue, 226 }  ,fill opacity=1 ][line width=0.08]  [draw opacity=0] (11.61,-5.58) -- (0,0) -- (11.61,5.58) -- cycle    ;
%Curve Lines [id:da7213601273577571]
\draw [color={rgb, 255:red, 74; green, 144; blue, 226 }  ,draw opacity=1 ][line width=1.5]    (818.16,237.86) .. controls (834.08,174.18) and (834.82,212.17) .. (811.15,70.34) ;
\draw [shift={(810.79,68.18)}, rotate = 440.55] [fill={rgb, 255:red, 74; green, 144; blue, 226 }  ,fill opacity=1 ][line width=0.08]  [draw opacity=0] (11.61,-5.58) -- (0,0) -- (11.61,5.58) -- cycle    ;
%Rounded Rect [id:dp8275444816273247]
\draw   (598.2,297.95) .. controls (598.2,294.78) and (600.77,292.22) .. (603.93,292.22) -- (623.76,292.22) .. controls (626.93,292.22) and (629.5,294.78) .. (629.5,297.95) -- (629.5,315.16) .. controls (629.5,318.33) and (626.93,320.9) .. (623.76,320.9) -- (603.93,320.9) .. controls (600.77,320.9) and (598.2,318.33) .. (598.2,315.16) -- cycle ;
%Rounded Rect [id:dp8002622268985179]
\draw   (644.36,297.95) .. controls (644.36,294.78) and (646.93,292.22) .. (650.1,292.22) -- (669.92,292.22) .. controls (673.09,292.22) and (675.66,294.78) .. (675.66,297.95) -- (675.66,315.16) .. controls (675.66,318.33) and (673.09,320.9) .. (669.92,320.9) -- (650.1,320.9) .. controls (646.93,320.9) and (644.36,318.33) .. (644.36,315.16) -- cycle ;
%Straight Lines [id:da7732085686827401]
\draw  [dash pattern={on 0.84pt off 2.51pt}]  (629.1,274) -- (611.8,289.95) ;
\draw [shift={(610.33,291.3)}, rotate = 317.35] [fill={rgb, 255:red, 0; green, 0; blue, 0 }  ][line width=0.08]  [draw opacity=0] (12,-3) -- (0,0) -- (12,3) -- cycle    ;
%Straight Lines [id:da1285346837964736]
\draw  [dash pattern={on 0.84pt off 2.51pt}]  (642.28,272.74) -- (657.41,290.35) ;
\draw [shift={(658.71,291.86)}, rotate = 229.32999999999998] [fill={rgb, 255:red, 0; green, 0; blue, 0 }  ][line width=0.08]  [draw opacity=0] (12,-3) -- (0,0) -- (12,3) -- cycle    ;
%Straight Lines [id:da26351102476457344]
\draw [color={rgb, 255:red, 74; green, 144; blue, 226 }  ,draw opacity=1 ][fill={rgb, 255:red, 0; green, 0; blue, 0 }  ,fill opacity=1 ][line width=1.5]    (629.39,309.31) -- (639.65,309.31) ;
\draw [shift={(643.65,309.31)}, rotate = 180] [fill={rgb, 255:red, 74; green, 144; blue, 226 }  ,fill opacity=1 ][line width=0.08]  [draw opacity=0] (11.61,-5.58) -- (0,0) -- (11.61,5.58) -- cycle    ;
%Straight Lines [id:da28047527387958604]
\draw [color={rgb, 255:red, 74; green, 144; blue, 226 }  ,draw opacity=1 ][line width=1.5]    (649.29,292.22) -- (639.54,275.2) ;
\draw [shift={(637.55,271.73)}, rotate = 420.19] [fill={rgb, 255:red, 74; green, 144; blue, 226 }  ,fill opacity=1 ][line width=0.08]  [draw opacity=0] (11.61,-5.58) -- (0,0) -- (11.61,5.58) -- cycle    ;
%Rounded Rect [id:dp736002922585473]
\draw  [dash pattern={on 4.5pt off 4.5pt}] (593.77,41.05) .. controls (593.77,37.88) and (596.33,35.31) .. (599.5,35.31) -- (638.5,35.31) .. controls (641.66,35.31) and (644.23,37.88) .. (644.23,41.05) -- (644.23,58.26) .. controls (644.23,61.43) and (641.66,63.99) .. (638.5,63.99) -- (599.5,63.99) .. controls (596.33,63.99) and (593.77,61.43) .. (593.77,58.26) -- cycle ;
%Rounded Rect [id:dp5790452356616052]
\draw  [dash pattern={on 4.5pt off 4.5pt}] (780.2,41.23) .. controls (780.2,38.06) and (782.77,35.49) .. (785.94,35.49) -- (824.93,35.49) .. controls (828.1,35.49) and (830.67,38.06) .. (830.67,41.23) -- (830.67,58.44) .. controls (830.67,61.6) and (828.1,64.17) .. (824.93,64.17) -- (785.94,64.17) .. controls (782.77,64.17) and (780.2,61.6) .. (780.2,58.44) -- cycle ;
%Straight Lines [id:da8401550921446078]
\draw  [dash pattern={on 0.84pt off 2.51pt}]  (668.23,20.88) -- (645.76,39.76) ;
\draw [shift={(644.23,41.05)}, rotate = 319.95] [fill={rgb, 255:red, 0; green, 0; blue, 0 }  ][line width=0.08]  [draw opacity=0] (12,-3) -- (0,0) -- (12,3) -- cycle    ;
%Straight Lines [id:da1764734219226035]
\draw  [dash pattern={on 0.84pt off 2.51pt}]  (778.67,39.95) -- (753.12,18.6) ;
\draw [shift={(780.2,41.23)}, rotate = 219.87] [fill={rgb, 255:red, 0; green, 0; blue, 0 }  ][line width=0.08]  [draw opacity=0] (12,-3) -- (0,0) -- (12,3) -- cycle    ;
%Rounded Rect [id:dp5583650771894602]
\draw  [fill={rgb, 255:red, 224; green, 219; blue, 219 }  ,fill opacity=1 ] (739.56,183.19) .. controls (739.56,180.02) and (742.13,177.45) .. (745.29,177.45) -- (809.22,177.45) .. controls (812.39,177.45) and (814.95,180.02) .. (814.95,183.19) -- (814.95,200.4) .. controls (814.95,203.56) and (812.39,206.13) .. (809.22,206.13) -- (745.29,206.13) .. controls (742.13,206.13) and (739.56,203.56) .. (739.56,200.4) -- cycle ;
%Straight Lines [id:da6709607705036338]
\draw  [dash pattern={on 0.84pt off 2.51pt}]  (777.24,208.11) -- (777.36,225.97) ;
\draw [shift={(777.37,227.97)}, rotate = 269.62] [fill={rgb, 255:red, 0; green, 0; blue, 0 }  ][line width=0.08]  [draw opacity=0] (12,-3) -- (0,0) -- (12,3) -- cycle    ;
%Curve Lines [id:da6119123592261212]
\draw [color={rgb, 255:red, 245; green, 166; blue, 35 }  ,draw opacity=1 ][line width=1.5]    (396.32,192.64) .. controls (398.61,193.49) and (399.42,195.08) .. (398.73,197.39) .. controls (398.06,199.7) and (398.86,201.22) .. (401.13,201.93) .. controls (403.39,202.58) and (404.19,204.03) .. (403.54,206.27) .. controls (402.9,208.5) and (403.7,209.88) .. (405.95,210.4) .. controls (408.17,210.86) and (409.17,212.49) .. (408.95,215.3) .. controls (408.36,217.45) and (409.16,218.69) .. (411.36,219) .. controls (413.53,219.25) and (414.53,220.7) .. (414.36,223.37) .. controls (413.81,225.46) and (414.8,226.83) .. (417.34,227.46) .. controls (419.45,227.49) and (420.45,228.76) .. (420.32,231.29) .. controls (420.23,233.8) and (421.21,234.99) .. (423.28,234.86) .. controls (425.7,235.11) and (426.88,236.43) .. (426.81,238.83) .. controls (426.78,241.22) and (427.94,242.43) .. (430.3,242.46) .. controls (432.61,242.39) and (433.77,243.49) .. (433.76,245.78) .. controls (433.81,248.06) and (435.14,249.22) .. (437.75,249.27) .. controls (439.92,248.89) and (441.22,249.93) .. (441.67,252.39) .. controls (441.8,254.56) and (443.08,255.48) .. (445.52,255.16) .. controls (447.87,254.72) and (449.3,255.65) .. (449.83,257.94) .. controls (450.4,260.21) and (451.8,261.01) .. (454.01,260.34) .. controls (456.14,259.57) and (457.66,260.35) .. (458.57,262.66) .. controls (459.2,264.79) and (460.66,265.44) .. (462.95,264.61) .. controls (465.13,263.7) and (466.68,264.31) .. (467.59,266.44) .. controls (468.52,268.55) and (470.12,269.12) .. (472.39,268.14) .. controls (474.29,267.01) and (475.77,267.5) .. (476.84,269.6) .. controls (478.07,271.76) and (479.67,272.29) .. (481.62,271.19) .. controls (483.77,270.2) and (485.37,270.83) .. (486.4,273.07) .. controls (487.06,275.26) and (488.45,276.05) .. (490.57,275.42) -- (491.86,276.46) ;
%Shape: Rectangle [id:dp06931307882957338]
\draw   (863.19,151.35) -- (996.86,151.35) -- (996.86,322) -- (863.19,322) -- cycle ;
%Rounded Rect [id:dp9373431563537398]
\draw  [fill={rgb, 255:red, 255; green, 255; blue, 255 }  ,fill opacity=1 ] (866.46,256.51) .. controls (866.46,255) and (867.68,253.78) .. (869.19,253.78) -- (877.41,253.78) .. controls (878.92,253.78) and (880.14,255) .. (880.14,256.51) -- (880.14,266.2) .. controls (880.14,267.71) and (878.92,268.94) .. (877.41,268.94) -- (869.19,268.94) .. controls (867.68,268.94) and (866.46,267.71) .. (866.46,266.2) -- cycle ;
%Rounded Rect [id:dp6252996753118361]
\draw  [fill={rgb, 255:red, 119; green, 119; blue, 119 }  ,fill opacity=1 ] (866.01,235.47) .. controls (866.01,234.07) and (867.14,232.93) .. (868.55,232.93) -- (876.18,232.93) .. controls (877.58,232.93) and (878.72,234.07) .. (878.72,235.47) -- (878.72,244.6) .. controls (878.72,246) and (877.58,247.14) .. (876.18,247.14) -- (868.55,247.14) .. controls (867.14,247.14) and (866.01,246) .. (866.01,244.6) -- cycle ;
%Rounded Rect [id:dp8224272608445956]
\draw  [fill={rgb, 255:red, 224; green, 219; blue, 219 }  ,fill opacity=1 ] (866.46,276.41) .. controls (866.46,274.9) and (867.68,273.67) .. (869.19,273.67) -- (877.41,273.67) .. controls (878.92,273.67) and (880.14,274.9) .. (880.14,276.41) -- (880.14,286.1) .. controls (880.14,287.61) and (878.92,288.84) .. (877.41,288.84) -- (869.19,288.84) .. controls (867.68,288.84) and (866.46,287.61) .. (866.46,286.1) -- cycle ;
%Rounded Rect [id:dp9255577029535439]
\draw  [fill={rgb, 255:red, 255; green, 255; blue, 255 }  ,fill opacity=1 ][dash pattern={on 0.84pt off 2.51pt}] (866.28,215.41) .. controls (866.28,214.73) and (866.83,214.18) .. (867.51,214.18) -- (871.2,214.18) .. controls (871.88,214.18) and (872.43,214.73) .. (872.43,215.41) -- (872.43,219.66) .. controls (872.43,220.34) and (871.88,220.89) .. (871.2,220.89) -- (867.51,220.89) .. controls (866.83,220.89) and (866.28,220.34) .. (866.28,219.66) -- cycle ;
%Rounded Rect [id:dp8408932387571381]
\draw  [fill={rgb, 255:red, 224; green, 219; blue, 219 }  ,fill opacity=1 ] (872.46,215.11) .. controls (872.46,214.43) and (873.01,213.88) .. (873.69,213.88) -- (877.38,213.88) .. controls (878.06,213.88) and (878.61,214.43) .. (878.61,215.11) -- (878.61,219.36) .. controls (878.61,220.04) and (878.06,220.59) .. (877.38,220.59) -- (873.69,220.59) .. controls (873.01,220.59) and (872.46,220.04) .. (872.46,219.36) -- cycle ;
%Rounded Rect [id:dp16021779953884874]
\draw  [fill={rgb, 255:red, 255; green, 255; blue, 255 }  ,fill opacity=1 ] (872.46,221.74) .. controls (872.46,221.06) and (873.01,220.51) .. (873.69,220.51) -- (877.38,220.51) .. controls (878.06,220.51) and (878.61,221.06) .. (878.61,221.74) -- (878.61,225.99) .. controls (878.61,226.67) and (878.06,227.22) .. (877.38,227.22) -- (873.69,227.22) .. controls (873.01,227.22) and (872.46,226.67) .. (872.46,225.99) -- cycle ;
%Rounded Rect [id:dp4261784856501417]
\draw  [fill={rgb, 255:red, 119; green, 119; blue, 119 }  ,fill opacity=1 ] (866.3,220.51) .. controls (866.3,220.51) and (866.3,220.51) .. (866.3,220.51) -- (872.46,220.51) .. controls (872.46,220.51) and (872.46,220.51) .. (872.46,220.51) -- (872.46,227.22) .. controls (872.46,227.22) and (872.46,227.22) .. (872.46,227.22) -- (866.3,227.22) .. controls (866.3,227.22) and (866.3,227.22) .. (866.3,227.22) -- cycle ;
%Rounded Rect [id:dp920962013677702]
\draw  [fill={rgb, 255:red, 255; green, 255; blue, 255 }  ,fill opacity=1 ][dash pattern={on 0.84pt off 2.51pt}] (867.39,300.1) .. controls (867.39,298.59) and (868.62,297.36) .. (870.13,297.36) -- (878.34,297.36) .. controls (879.85,297.36) and (881.08,298.59) .. (881.08,300.1) -- (881.08,309.79) .. controls (881.08,311.3) and (879.85,312.52) .. (878.34,312.52) -- (870.13,312.52) .. controls (868.62,312.52) and (867.39,311.3) .. (867.39,309.79) -- cycle ;
%Straight Lines [id:da2386760978227387]
\draw [color={rgb, 255:red, 74; green, 144; blue, 226 }  ,draw opacity=1 ][fill={rgb, 255:red, 0; green, 0; blue, 0 }  ,fill opacity=1 ][line width=1.5]    (867.78,198.02) -- (890.4,198.04) ;
\draw [shift={(894.4,198.04)}, rotate = 180.05] [fill={rgb, 255:red, 74; green, 144; blue, 226 }  ,fill opacity=1 ][line width=0.08]  [draw opacity=0] (11.61,-5.58) -- (0,0) -- (11.61,5.58) -- cycle    ;
%Straight Lines [id:da41035588779934484]
\draw [color={rgb, 255:red, 245; green, 166; blue, 35 }  ,draw opacity=1 ][fill={rgb, 255:red, 0; green, 0; blue, 0 }  ,fill opacity=1 ][line width=1.5]    (866.54,178.3) .. controls (868.21,176.63) and (869.87,176.63) .. (871.54,178.3) .. controls (873.21,179.97) and (874.87,179.97) .. (876.54,178.31) .. controls (878.21,176.64) and (879.87,176.64) .. (881.54,178.31) -- (882.17,178.31) -- (890.17,178.32) ;
\draw [shift={(894.17,178.32)}, rotate = 180.04] [fill={rgb, 255:red, 245; green, 166; blue, 35 }  ,fill opacity=1 ][line width=0.08]  [draw opacity=0] (11.61,-5.58) -- (0,0) -- (11.61,5.58) -- cycle    ;
%Straight Lines [id:da9353020233445941]
\draw [color={rgb, 255:red, 126; green, 211; blue, 33 }  ,draw opacity=1 ][fill={rgb, 255:red, 0; green, 0; blue, 0 }  ,fill opacity=1 ][line width=1.5]  [dash pattern={on 5.63pt off 4.5pt}]  (867.08,158.56) -- (880.29,158.58) -- (890.71,158.58) ;
\draw [shift={(894.71,158.58)}, rotate = 180.03] [fill={rgb, 255:red, 126; green, 211; blue, 33 }  ,fill opacity=1 ][line width=0.08]  [draw opacity=0] (11.61,-5.58) -- (0,0) -- (11.61,5.58) -- cycle    ;
%Curve Lines [id:da40873579465177934]
\draw [color={rgb, 255:red, 74; green, 144; blue, 226 }  ,draw opacity=1 ][line width=1.5]    (705,257) .. controls (723.05,274.01) and (748.68,264.5) .. (785.48,294.26) ;
\draw [shift={(788.33,296.62)}, rotate = 220.4] [fill={rgb, 255:red, 74; green, 144; blue, 226 }  ,fill opacity=1 ][line width=0.08]  [draw opacity=0] (11.61,-5.58) -- (0,0) -- (11.61,5.58) -- cycle    ;
%Rounded Rect [id:dp47587316702499083]
\draw  [fill={rgb, 255:red, 255; green, 255; blue, 255 }  ,fill opacity=1 ] (444.54,230.09) .. controls (444.54,226.54) and (447.41,223.67) .. (450.96,223.67) -- (520.28,223.67) .. controls (523.82,223.67) and (526.7,226.54) .. (526.7,230.09) -- (526.7,249.34) .. controls (526.7,252.89) and (523.82,255.76) .. (520.28,255.76) -- (450.96,255.76) .. controls (447.41,255.76) and (444.54,252.89) .. (444.54,249.34) -- cycle ;
%Rounded Rect [id:dp7554097198860851]
\draw  [fill={rgb, 255:red, 255; green, 255; blue, 255 }  ,fill opacity=1 ] (736,237.86) .. controls (736,234.31) and (738.88,231.44) .. (742.42,231.44) -- (811.74,231.44) .. controls (815.29,231.44) and (818.16,234.31) .. (818.16,237.86) -- (818.16,257.12) .. controls (818.16,260.66) and (815.29,263.54) .. (811.74,263.54) -- (742.42,263.54) .. controls (738.88,263.54) and (736,260.66) .. (736,257.12) -- cycle ;
%Straight Lines [id:da6237703211481653]
\draw [color={rgb, 255:red, 245; green, 166; blue, 35 }  ,draw opacity=1 ][line width=1.5]    (524.72,278.98) .. controls (522.37,278.83) and (521.27,277.57) .. (521.42,275.22) .. controls (521.57,272.87) and (520.46,271.62) .. (518.11,271.47) -- (516.23,269.33) -- (510.94,263.33) ;
\draw [shift={(508.3,260.33)}, rotate = 408.62] [fill={rgb, 255:red, 245; green, 166; blue, 35 }  ,fill opacity=1 ][line width=0.08]  [draw opacity=0] (11.61,-5.58) -- (0,0) -- (11.61,5.58) -- cycle    ;
%Curve Lines [id:da27799061143931836]
\draw [color={rgb, 255:red, 245; green, 166; blue, 35 }  ,draw opacity=1 ][line width=1.5]    (411.32,241.64) .. controls (413.93,242.01) and (415.21,243.36) .. (415.15,245.68) .. controls (415.16,248.01) and (416.33,249.09) .. (418.64,248.94) .. controls (420.89,248.68) and (422.22,249.79) .. (422.65,252.28) .. controls (422.86,254.53) and (424.07,255.43) .. (426.28,254.96) .. controls (428.72,254.6) and (430.1,255.51) .. (430.43,257.68) .. controls (431.14,260.05) and (432.69,260.94) .. (435.08,260.37) .. controls (437.09,259.54) and (438.49,260.25) .. (439.29,262.5) .. controls (440.18,264.75) and (441.74,265.44) .. (443.98,264.59) .. controls (446.15,263.67) and (447.71,264.28) .. (448.66,266.42) .. controls (449.67,268.55) and (451.21,269.08) .. (453.3,268.02) .. controls (455.32,266.91) and (457,267.42) .. (458.33,269.55) .. controls (459.4,271.58) and (460.89,271.99) .. (462.81,270.78) .. controls (464.97,269.61) and (466.57,270) .. (467.61,271.97) .. controls (468.94,274) and (470.62,274.39) .. (472.66,273.15) .. controls (474.66,271.9) and (476.26,272.26) .. (477.47,274.25) .. controls (478.62,276.23) and (480.26,276.62) .. (482.37,275.43) .. controls (484.46,274.25) and (486.08,274.69) .. (487.22,276.75) -- (487.22,276.75) -- (494.59,279.53) ;
\draw [shift={(497.86,281.46)}, rotate = 215.74] [fill={rgb, 255:red, 245; green, 166; blue, 35 }  ,fill opacity=1 ][line width=0.08]  [draw opacity=0] (11.61,-5.58) -- (0,0) -- (11.61,5.58) -- cycle    ;
%Curve Lines [id:da8614685572167997]
\draw [color={rgb, 255:red, 245; green, 166; blue, 35 }  ,draw opacity=1 ][line width=1.5]    (482.67,293.33) .. controls (484.78,294.38) and (485.37,295.98) .. (484.43,298.12) .. controls (483.63,300.37) and (484.36,301.85) .. (486.62,302.54) .. controls (488.89,303.04) and (489.81,304.41) .. (489.38,306.66) .. controls (489.26,309.07) and (490.42,310.28) .. (492.86,310.31) .. controls (495.1,309.91) and (496.49,310.82) .. (497.04,313.05) -- (497.67,313.33) ;
%Straight Lines [id:da26292308817099974]
\draw    (853.45,0.62) -- (853.45,323.54) ;
%Curve Lines [id:da14447198361207314]
\draw [color={rgb, 255:red, 74; green, 144; blue, 226 }  ,draw opacity=1 ][line width=1.5]    (704.61,246.05) .. controls (717.91,234.85) and (702.83,194.55) .. (728.83,176.55) ;
%Straight Lines [id:da34766493159530876]
\draw  [dash pattern={on 0.84pt off 2.51pt}]  (482.98,258.07) -- (464.28,276.69) ;
\draw [shift={(462.86,278.11)}, rotate = 315.12] [fill={rgb, 255:red, 0; green, 0; blue, 0 }  ][line width=0.08]  [draw opacity=0] (12,-3) -- (0,0) -- (12,3) -- cycle    ;
%Straight Lines [id:da3327881591237285]
\draw  [dash pattern={on 0.84pt off 2.51pt}]  (725.34,161.1) -- (744.3,177.25) ;
\draw [shift={(745.83,178.55)}, rotate = 220.43] [fill={rgb, 255:red, 0; green, 0; blue, 0 }  ][line width=0.08]  [draw opacity=0] (12,-3) -- (0,0) -- (12,3) -- cycle    ;

% Text Node
\draw (68.12,176.18) node [anchor=north west][inner sep=0.75pt]    {$\mathtt{AndOp}$};
% Text Node
\draw (104.79,219.49) node [anchor=north west][inner sep=0.75pt]    {$\mathtt{b_{1}}$};
% Text Node
\draw (29.29,266.84) node [anchor=north west][inner sep=0.75pt]    {$\mathtt{p_{1}}$};
% Text Node
\draw (75.47,268.35) node [anchor=north west][inner sep=0.75pt]    {$\mathtt{p_{2}}$};
% Text Node
\draw (39.6,220.77) node [anchor=north west][inner sep=0.75pt]    {$\mathtt{EQOp}$};
% Text Node
\draw (16.35,38.93) node [anchor=north west][inner sep=0.75pt]    {$\mathtt{Entry}$};
% Text Node
\draw (219.68,38.76) node [anchor=north west][inner sep=0.75pt]    {$\mathtt{Exit}$};
% Text Node
\draw (167.79,273.07) node [anchor=north west][inner sep=0.75pt]    {$\mathtt{p_{1}}$};
% Text Node
\draw (218.35,274.35) node [anchor=north west][inner sep=0.75pt]    {$\mathtt{0}$};
% Text Node
\draw (163.88,172.45) node [anchor=north west][inner sep=0.75pt]    {$\mathtt{ExprStmt}$};
% Text Node
\draw (101.7,9.38) node [anchor=north west][inner sep=0.75pt]  [color={rgb, 255:red, 255; green, 255; blue, 255 }  ,opacity=1 ]  {$\mathtt{MethodDecl}$};
% Text Node
\draw (114.89,121.27) node [anchor=north west][inner sep=0.75pt]    {$\mathtt{IfStmt}$};
% Text Node
\draw (161.85,215.96) node [anchor=north west][inner sep=0.75pt]    {$\mathtt{AssignExpr}$};
% Text Node
\draw (118.12,64.75) node [anchor=north west][inner sep=0.75pt]    {$\mathtt{Block}$};
% Text Node
\draw (347.72,187.83) node [anchor=north west][inner sep=0.75pt]    {$\mathtt{AndOp}$};
% Text Node
\draw (386.66,236.36) node [anchor=north west][inner sep=0.75pt]    {$\mathtt{b_{1}}$};
% Text Node
\draw (309.85,288.52) node [anchor=north west][inner sep=0.75pt]    {$\mathtt{p_{1}}$};
% Text Node
\draw (356.89,288.95) node [anchor=north west][inner sep=0.75pt]    {$\mathtt{p_{2}}$};
% Text Node
\draw (318.51,237.95) node [anchor=north west][inner sep=0.75pt]    {$\mathtt{EQOp}$};
% Text Node
\draw (299.4,37.51) node [anchor=north west][inner sep=0.75pt]    {$\mathtt{Entry}$};
% Text Node
\draw (505.99,41.16) node [anchor=north west][inner sep=0.75pt]    {$\mathtt{Exit}$};
% Text Node
\draw (457.21,286.71) node [anchor=north west][inner sep=0.75pt]    {$\mathtt{p_{1}}$};
% Text Node
\draw (509.15,287.36) node [anchor=north west][inner sep=0.75pt]    {$\text{0}$};
% Text Node
\draw (453.26,181.02) node [anchor=north west][inner sep=0.75pt]    {$\mathtt{ExprStmt}$};
% Text Node
\draw (307.47,146.51) node [anchor=north west][inner sep=0.75pt]  [font=\tiny,rotate=-325.1] [align=left] {False};
% Text Node
\draw (362.54,251.83) node [anchor=north west][inner sep=0.75pt]  [font=\tiny,rotate=-1.51] [align=left] {True};
% Text Node
\draw (382.45,10.67) node [anchor=north west][inner sep=0.75pt]  [color={rgb, 255:red, 255; green, 255; blue, 255 }  ,opacity=1 ]  {$\mathtt{MethodDecl}$};
% Text Node
\draw (399.08,128.6) node [anchor=north west][inner sep=0.75pt]    {$\mathtt{IfStmt}$};
% Text Node
\draw (399.81,69.15) node [anchor=north west][inner sep=0.75pt]    {$\mathtt{Block}$};
% Text Node
\draw (411.45,202.83) node [anchor=north west][inner sep=0.75pt]  [font=\tiny,rotate=-53.12] [align=left] {True};
% Text Node
\draw (645.18,198.25) node [anchor=north west][inner sep=0.75pt]    {$\mathtt{AndOp}$};
% Text Node
\draw (679.7,244.16) node [anchor=north west][inner sep=0.75pt]    {$\mathtt{b_{1}}$};
% Text Node
\draw (750.14,298.94) node [anchor=north west][inner sep=0.75pt]    {$\mathtt{p_{1}}$};
% Text Node
\draw (799.16,299.88) node [anchor=north west][inner sep=0.75pt]    {$\mathtt{0}$};
% Text Node
\draw (606.18,298.31) node [anchor=north west][inner sep=0.75pt]    {$\mathtt{p_{1}}$};
% Text Node
\draw (650.75,299.98) node [anchor=north west][inner sep=0.75pt]    {$\mathtt{p_{2}}$};
% Text Node
\draw (617.32,246.51) node [anchor=north west][inner sep=0.75pt]    {$\mathtt{EQOp}$};
% Text Node
\draw (598.78,41.44) node [anchor=north west][inner sep=0.75pt]    {$\mathtt{Entry}$};
% Text Node
\draw (789.97,41.62) node [anchor=north west][inner sep=0.75pt]    {$\mathtt{Exit}$};
% Text Node
\draw (748.48,185.74) node [anchor=north west][inner sep=0.75pt]    {$\mathtt{ExprStmt}$};
% Text Node
\draw (674.79,10.55) node [anchor=north west][inner sep=0.75pt]  [color={rgb, 255:red, 255; green, 255; blue, 255 }  ,opacity=1 ]  {$\mathtt{MethodDecl}$};
% Text Node
\draw (691.11,135.36) node [anchor=north west][inner sep=0.75pt]    {$\mathtt{IfStmt}$};
% Text Node
\draw (692.8,73.37) node [anchor=north west][inner sep=0.75pt]    {$\mathtt{Block}$};
% Text Node
\draw (897.18,191.86) node [anchor=north west][inner sep=0.75pt]   [align=left] {\Asyn{succ}};
% Text Node
\draw (888.39,233.23) node [anchor=north west][inner sep=0.75pt]    {$\mathtt{CFGRoot}$};
% Text Node
\draw (889.51,254.07) node [anchor=north west][inner sep=0.75pt]    {$\mathtt{CFGNode}$};
% Text Node
\draw (888.32,276.08) node [anchor=north west][inner sep=0.75pt]    {$\mathtt{CFGSupport}$};
% Text Node
\draw (888.44,213.33) node [anchor=north west][inner sep=0.75pt]    {$\mathtt{AST\ node}$};
% Text Node
\draw (859.35,136.83) node [anchor=north west][inner sep=0.75pt]   [align=left] {\textit{Legend}};
% Text Node
\draw (891.95,297.82) node [anchor=north west][inner sep=0.75pt]    {$\mathtt{HOA}$};
% Text Node
\draw (896.86,171.33) node [anchor=north west][inner sep=0.75pt]   [align=left] {\Ainh{nextNodes}};
% Text Node
\draw (897.18,152.4) node [anchor=north west][inner sep=0.75pt]   [align=left] {\Asyn{firstNodes}};
% Text Node
\draw (190.61,228.99) node [anchor=north west][inner sep=0.75pt]    {$\mathtt{p_{1}}$};
% Text Node
\draw (449.28,225.61) node [anchor=north west][inner sep=0.75pt]    {$\mathtt{AssignExpr}$};
% Text Node
\draw (478.04,238.64) node [anchor=north west][inner sep=0.75pt]    {$\mathtt{p_{1}}$};
% Text Node
\draw (740.74,233.39) node [anchor=north west][inner sep=0.75pt]    {$\mathtt{AssignExpr}$};
% Text Node
\draw (769.51,246.41) node [anchor=north west][inner sep=0.75pt]    {$\mathtt{p_{1}}$};
% Text Node
\draw (864,12.4) node [anchor=north west][inner sep=0.75pt]  [scale=1.45]  {
\begin{lstlisting}[language=java,frame=none]
void
foo(int p1, int p2, boolean b1){
  if(p1==p2 && b1){
    p1 = 0;
  }
}
\end{lstlisting}};
% Text Node
\draw (421.95,260.64) node [anchor=north west][inner sep=0.75pt]  [font=\tiny,rotate=-29.59] [align=left] {True};
% Text Node
\draw (411.46,293.93) node [anchor=north west][inner sep=0.75pt]  [font=\tiny,rotate=-34.82] [align=left] {False};
% Text Node
\draw (345.98,174.35) node [anchor=north west][inner sep=0.75pt]  [font=\tiny,rotate=-239.9] [align=left] {False};


\end{tikzpicture}

}
\caption{Visualization of the attributes \Asyn{firstNodes}, \Ainh{nextNodes} and \Asyn{succ}. For boolean expressions (\astnode{AndOp} and \astnode{EQOp}), the subsets \Ainh{nextNodesTT} and \Ainh{nextNodesFF} are shown instead of \Ainh{nextNodes}, marked by True and False, respectively.}
\label{fig:RunningExample}
\end{figure*}


Figure~\ref{fig:exampleuml} shows how the framework can be specialised to some example AST node types to define the desired \CFG.
JastAdd expresses these additions in a modular attribution aspect.
For illustration, we again encode the JastAdd specification into UML and include the abstract syntax of each node type.
Listing~\ref{listing:jastadd-eqop} also illustrates how the pseudocode can be translated to JastAdd code.

Here, \astnode{MethodDecl} exemplifies a \astnode{CFGRoot}.
It defines the flow between its \Ahoa{entry} and \Ahoa{exit} HOAs and its children.
\astnode{EQOp} exemplifies a \astnode{CFGNode}.
It defines a pre-order flow: \code{left}, then \code{right}, then the node itself.
Each type defines its own synthesised attributes as well as the inherited attributes of its children and HOAs.

All \astnode{CFGNode}s have immediate access to the \astnode{Entry} and \astnode{Exit} nodes of the \CFG, through
the inherited \Ainh{entry} and \Ainh{exit} attributes declared in \astnode{CFGNode} and defined by the nearest \astnode{CFGRoot} ancestor (Figure~\ref{fig:intracfgframework}).
This allows e.g., the \astnode{ReturnStmt} to point its \Asyn{succ} edge directly to the \astnode{Exit} node.

For boolean expressions that affect control-flow, {\intracfg} supports path-sensitive analysis, splitting the successor set into two disjoint sets for the \emph{true} and \emph{false} branches.
We provide attributes \Ainh{nextNodesTT} and \Ainh{nextNodesFF}, respectively, to capture these branches.
The \astnode{AndOp} type illustrates how these attributes can capture short-circuit evaluation on the left child.
These attributes are relevant only for boolean branches, and must ensure the following property:\\
\code{$\;\;\;\;\;\;\;\Ainh{nextNodesTT} \: \cup \Ainh{nextNodesFF} \;=\; \Ainh{nextNodes}$}

Figure~\ref{fig:RunningExample} illustrates these attributes on a small program in a language with methods, statements, and expressions.
Here, \astnode{MethodDecl} is a \astnode{CFGRoot} and thus automatically has fresh \astnode{Entry} and \astnode{Exit} nodes.
Nodes in the control flow, e.g., identifiers and the equality-check operator, \astnode{EQOp}, are \astnode{CFGNode}s, and thus have the \Asyn{succ} attribute.
Nodes that do not belong to the control-flow but live in AST locations below a \astnode{CFGRoot} that may contribute to control flow are \astnode{CFGSupport} nodes.
The left-hand-side variable of the assignment \code{p1 = 0} (i.e., $\texttt{p}_1$) is not part of the flow (cf.\@ Section~\ref{sec:stmtexp}).
%For instance, \code{AndOp} is extended with \code{CFGSupport} in order to define the \code{nextNodes}, \code{nextNodesTT}, \code{nextNodesFF} attributes of its children, which are in turn used for defining the successor attribute.

\subsection{Computing predecessors}

To support both forward and backward analyses, we provide a predecessor attribute that captures the inverse of the successor attribute \Asyn{succ}.
However, \Asyn{succ} is also defined for \astnode{CFGNode}s that are not reachable from \astnode{Entry} by following \Asyn{succ} (i.e., that are ``dead code'').
Our framework therefore computes predecessor edges \Asyn{pred} by not only inverting \Asyn{succ} into a collection attribute \Acoll{succInv}, but also by filtering out such ``dead'' nodes from \Acoll{succInv} with a boolean circular attribute \Acirc{reachable} (Figure~\ref{fig:intracfgframework}).

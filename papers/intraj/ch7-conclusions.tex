We presented {\intracfg}, a RAG-based declarative language-independent framework for constructing intraprocedural CFGs.
{\intracfg} superimposes CFGs on the AST, allowing client analyses to take advantage of other AST attributes, such as type information and precise source information.
We validated our approach by implementing {\intraj}, an application of {\intracfg} to Java~7,
and demonstrated how {\intracfg} overcomes the limitations of an earlier RAG-based framework, {\jastaddjintraflow} (JJI), by allowing the CFG to not be constrained by the AST structure.
Compared to JJI, {\intraj} can faithfully capture execution order and improve {\CFG} conciseness and precision, removing  more than 30\% of the {\CFG} edges in our benchmarks.
We evaluated \intraj{} by implementing two data flow analyses: Null Pointer Exception Analysis (NPA) and Dead Assignment Analysis (DAA), comparing both to JJI (for DAA), and to the highly tuned commercial tool SonarQube (SQ) (for DAA and NPA).
Our results show that the {\intraj}-based analyses offer precision that is comparable to that of JJI and SQ.
%For DDA, {\intraj}'s analyses are sometimes more accurate than the other tools.
Compared to JJI, {\intraj} pays some overhead for computing more precise {\CFG} but can amortise this effort for larger programs by speeding up client analyses, outperforming JJI.
Compared to SQ, {\intraj}'s NPA analysis is substantially faster, although this is likely due to SQ's more advanced interprocedural analysis.
{\intraj}'s DAA analysis seems slower than SQ's, but SQ has a much larger baseline, which might include computations that we would attribute to the analysis for {\intraj}.
Overall, we find that our results demonstrate that {\intraj}-based data flow analyses are practical, that {\intraj} enables precise data flow analyses on Java source code,
and that {\intracfg} is effective for constructing {\CFG}s for Java-like languages.
Moreover, we demonstrate for the first time how RAGs can build and exploit graph structures over an AST without being restricted by the AST's structure.
